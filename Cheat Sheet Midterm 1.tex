\documentclass[8pt,reqno,twocolumn]{amsart}
\usepackage[textwidth=8in,textheight=10in]{geometry}

%%%%%%%%%%%%%%%%%%%%%%%%%%%%%%%%%%%%%%%%%%%%%%%%%%%
%								Packages									         %
%%%%%%%%%%%%%%%%%%%%%%%%%%%%%%%%%%%%%%%%%%%%%%%%%%%

\usepackage[T1]{fontenc}

\usepackage{amsmath}							
\usepackage{amssymb}
\usepackage{amsthm}
\usepackage{amscd}
\usepackage{amsfonts}
\usepackage{stmaryrd}
\usepackage{algorithm, algorithmic}
\usepackage{ wasysym }
\usepackage{caption}
\usepackage{subcaption}
\usepackage{euler}
\renewcommand{\rmdefault}{pplx}
\usepackage{extarrows}
\usepackage[colorlinks, linktocpage, citecolor = red, linkcolor = blue]{hyperref}
\usepackage{color}
\usepackage{tikz}									
\usepackage{fullpage}
\usepackage{esdiff}
\usepackage{xcolor}
\usepackage{soul}
\usepackage[shortlabels]{enumitem}

\linespread{1.1}

%%%%%%%%%%%%%%%%%%%%%%%%%%%%%%%%%%%%%%%%%%%%%%%%%%%
%								Theorems 								         %
%%%%%%%%%%%%%%%%%%%%%%%%%%%%%%%%%%%%%%%%%%%%%%%%%%%

\newtheorem*{maintheorem}{Main Theorem}

\newtheorem{theorem}{Theorem}[section]
\newtheorem*{theorem*}{Theorem}
\newtheorem*{lemma}{Lemma}
\newtheorem*{proposition}{Proposition}
\newtheorem*{corollary}{Corollary} 
\newtheorem*{conjecture}{Conjecture} 

\theoremstyle{definition}					                  		
\newtheorem*{definition}{Definition}
\newtheorem*{question}{Question}
\newtheorem*{example}{Example}
\newtheorem*{ogexample}{Original Example}
\newtheorem*{construction}{Construction}


\newtheorem{remark}[theorem]{Remark}
\newtheorem{remarks}[theorem]{Remarks}

\renewcommand{\algorithmicrequire}{\textbf{Input:}}
\renewcommand{\algorithmicensure}{\textbf{Output:}}

%%%%%%%%%%%%%%%%%%%%%%%%%%%%%%%%%%%%%%%%%%%%%%%%%%%
%								Operators									         %
%%%%%%%%%%%%%%%%%%%%%%%%%%%%%%%%%%%%%%%%%%%%%%%%%%%

\newcommand{\R}{\mathbb{R}}
\newcommand{\Z}{\mathbb{Z}}
\newcommand{\C}{\mathbb{C}}
\newcommand{\Q}{\mathbb{Q}}
\newcommand{\N}{\mathbb{N}}
\newcommand{\F}{\mathbb{F}}
\renewcommand{\tt}{\mathcal{T}}
\newcommand{\bb}{\mathcal{B}}
\newcommand{\nul}{\mathrm{null}}
\newcommand{\range}{\mathrm{range}}
\newcommand{\spa}{\mathrm{span}}
\renewcommand*{\epsilon}{\varepsilon}

\begin{document}
\subsection*{Abbott Definitions}
\begin{enumerate}
    \item Perfect: closed and contains no isolated points
    \item Isolated Point: not a limit point
    \item Separated: $A$ and $B$ are separated iff $\overline{A} \cap B = \emptyset$ and $A \cap \overline{B} = \emptyset$
    \item Disconnected: $E$ is disconnected if $E = A \cup B$ and $A, B$ are separated
    \item Totally Disconnected: $A$ is totally disconnected if its only connected sets are singletons and the empty set
    \item Nowhere Dense: no open intervals
\end{enumerate}
\begin{lemma}
 The following are equivalent: (1) $\tt \subset \tt'$. (2) $\forall x \in X$, $B \in \bb$ s.t $x \in B$, $\exists B' \in \bb'$ s.y $x \in B' \subset B$.
\end{lemma}
\subsection*{Product Topology}
\begin{definition}
    Let $X$,$Y$ be topological spaces. The product topology on $X \times Y$ is the topology having the basis $\bb = {U \times V}$ where $U$ is a open set in $X$ and $V$ is a open set in $Y$.
\end{definition}

\begin{theorem*}
    If $\bb$ is a basis for the topology $X$ and $\mathcal{C}$ is a basis for the topology $Y$, then $\mathcal{D} = \{U\times V: U \in \bb, V \in \mathcal{C}\}$ is a basis for the product topology $X \times Y$.
\end{theorem*}
\begin{theorem*}
    The collection $\mathcal{S} = \{\pi_1^{-1}(U): U\text{ open in $X$}\} \cup \{ \pi_2^{-1}(V): V \text{ open in $Y$}\}$ is a subbasis for the product topology on $X \times Y$.
\end{theorem*}

\subsection*{Subspace Topology}
\begin{definition}
    If $X$ is a space with topology $\tt$ and $Y \subset X$, then $\tt_Y = \{Y \cap U : U \in \tt\}$ is a topology on $Y$.
\end{definition}
\begin{lemma}
    Let $Y$ be a subspace of $X$. If $U$ is open in $Y$, and $Y$ is open in $X$, then $U$ is open in $X$ (and similarly for closed sets).
\end{lemma}
\begin{lemma}
    If $\bb$ is a basis for the topology on $X$, then $\bb_y = \{Y \cap B : B \in \bb_x\}$ is a basis for the subspace topology on $Y$.
\end{lemma}
\begin{theorem*}
    Let $A$ subspace of $X$, $B$ subspace of $Y$. The product topology on $A \times B$ is the same as the subspace topology on $A \times B$ inherited from $X \times Y$.
\end{theorem*}
\subsection*{Closed Sets and Limit Points}
\begin{theorem*}
    Let $Y$ be a subspace of $X$. $A$ is closed in $Y$ iff $A = C \cap Y$, where $C$ is closed in $X$.
\end{theorem*}
\begin{theorem*}
    Let $\overline{A}$ be the closure of $A$ in $X$. Then the closure of $A$ in $Y$ is $\overline{A} \cap Y$.
\end{theorem*}
\begin{theorem*}
    Let $A$ be a subset of a topological space $X$. Then $x \in \overline{A}$ iff every open set $U$ containing $x$ intersects $A$. Alternatively, every basis element containing $x$ intersects $A$. 
\end{theorem*}
\begin{definition}[Limit Point]
    $x$ is a limit point of $A$ if every neighborhood $U$ containing $x$ intersects $A$ at a point other than $x$.
\end{definition}

\subsection*{Hausdorff Spaces}
\begin{definition}[Convergence]
    The sequence $(x_n) \rightarrow x$ in $X$ if, for all neighborhoods $U$ of $x$, $\exists N$ such that $\forall n \geq N, x_n \in U$. 
\end{definition}
\begin{definition}[Hausdorff Space]
    For every distinct pair of points $x_1, x_2$, there are neighborhoods $U_1$ and $U_2$ of $x_1$ and $x_2$ that are disjoint.
\end{definition}
\begin{definition}[$T_1$]
    Every finite point set is closed. Hausdorff implies $T_1$.
\end{definition}
\begin{theorem*}
    Every finite point set in a Hausdorff space is open.
\end{theorem*}

\subsection*{Continuity and Homeomorphisms}

\begin{definition}[Continuity]
    Let $X$ and $Y$ be topological surfaces. A function $f:X \to Y$ is continuous if for each open set $V$ in $Y$, the set $f^{-1}(V)$ is open in $X$
\end{definition}

\begin{theorem*}
    Let $X,Y$ be topological spaces; let $f: X \to Y$. Then the following are equivalent: (1) $f$ is continuous; (2) $\forall A \subset X$, we have $f(\overline{A}) \subset \overline{f(A)}$. (3) For every closed set $B$ of $Y,$ the set $f^{-1}(B)$ is closed in $X$. (4) For each $x \in X$ and each neighborhood $V$ of $f(x)$, there is a neighborhood $U$ of $x$ such that $f(U) \subset V$.
\end{theorem*}

\begin{definition}[Homeomorphism]
    Let $X,Y$ be topological spaces; let $f: X \to Y$ be a bijection. If both $f$ and $f^{-1}$ are continuous, then $f$ is called a homeomorphism.
\end{definition}

\begin{theorem*}[pasting lemma]
    Let $X = A \cup B$ where $A,B$ are closed in $X$. Let $f: A \to Y$, $g: B \to Y$ be continuous. If $f(x) = g(x)$ when $x \in A \cap B$, then $f$ and $g$ combine to give a continuous function $h: X \to Y$, defined by setting $h(x) = f(x)$ if $x \in A$ and $h(x) = g(x)$ if $x \in B$.
\end{theorem*}

\begin{theorem*}
    Let $f: A \to X \times Y$ be given by the equation $f(a) = (f_1(a),f_2(a))$. Then $f$ is continuous if and only if $f_1: A \to X$ and $f_2: A \to Y$ are continuous.
\end{theorem*}

\subsection*{Useful Results}
\begin{theorem*}[Heine-Borel]
    The following are equivalent:
    \begin{enumerate}
        \item[(1)] $S$ is compact
        \item[(2)] Every open cover of $S$ has a finite subcover
        \item[(3)] $S$ is closed and bounded
    \end{enumerate}
\end{theorem*}
\begin{theorem*}[Cantor's Theorem]
    card($X$) $\leq$ card($P(X)$)
\end{theorem*}
\begin{theorem*}[Bolzano–Weierstrass]
    Every bounded sequence has a convergent subsequence.
\end{theorem*}
\begin{theorem*}[Shroeder-Bernstein]
    If $\exists f: A \rightarrow B$ injective, and $\exists g: B \rightarrow A$ injective, then card($A$) = card($B$).
\end{theorem*}
\begin{theorem*}[N.I.P]
    If $I_{n+1} \subset I_n$, then $\bigcap I_n \neq \emptyset$. This is also true for compact sets.
    
\end{theorem*}
\begin{theorem*}[Set Theory]$ $
    \begin{itemize}
        \item $A \cap (B- C) = (A \cap B) - (A \cap C)$
        \item $A \cap (B \cup C) = (A \cup B) \cap (A \cap C)$
        \item $(\bigcap_{i\in I} A_i)^c = \bigcup_{i \in I} A_i^c$ 
        \item $(\bigcup_{i\in I} A_i)^c = \bigcap_{i \in I} A_i^c$
    \end{itemize}
\end{theorem*}

\subsection*{Good Examples to Know}
\begin{itemize}
    \item $\Q$ - countable, nowhere dense, interior of closure if R, closure of interior is empty, 
    \item $1/n$ - only 1 limit point, limit point is 0, only converges to 0
    \item $(\overline{A})^c = (A^c)^0$
    \item The following function is continuous only at 0:

    $f = \begin{cases}
        x \text{ if } x \in \Q \\
        0 \text{ if } x \notin \Q
    \end{cases}$

    \item Properties of the Cantor set: uncountable, length/measure 0, nowhere dense, perfect.
    \item Let Let $\tt_1$ be the standard topology,$\tt_2$ be the topology $\R_K$, $\tt_3$, the finite complement topology, $\tt_4$ the upper limit topology, and $\tt_5$ the topology having all set $(-\infty,a)$ as a basis.
    $\tt_3/\tt_5 \subset \tt_1 \subset \tt_2 \subset \tt_4$
\end{itemize}


\end{document}