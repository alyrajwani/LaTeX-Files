\documentclass{hmwk}

\hdr{Problem Set 7}{MATH 1530: Abstract Algebra}{Aly Rajwani}
\hwk{7}

\begin{document}
\maketitle

\begin{problem}{Problem 1}

\pre Let $I = [0,1]$ and consider the ring $\mathrm{Fun}(I, R)$ of functions $f: [0,1] \to \R$ under the operations of pointwise addition and multiplication.

\begin{itemize}
    \item[(a)] Does $\mathrm{Fun}(I, R)$ have a multiplicative identity element?
    \item[(b)] What elements of $\mathrm{Fun}(I, R)$ are units?
    \item[(c)] What elements of $\mathrm{Fun}(I, R)$ are zero divisors?
    \item[(d)] Show that $S = \{ f : f(1/2) = 0\}$ is a subring of $\mathrm{Fun}(I, R)$.
\end{itemize}
    
\end{problem}

\begin{solution}
We first note that since pointwise addition and multiplication are commutative, $\mathrm{Fun}(I, R)$ is a commutative ring.

\begin{itemize}
    \item[(a)] A multiplicative identity element is an element such $1$ such that $\forall x \in R, 1 \cdot x = x \cdot 1 = x$. The multiplicative identity element of $\mathrm{Fun}(I, R)$ is $f(x) = 1$, since for any $g(x) \in \mathrm{Fun}(I, R)$, $f(x)g(x) = 1\cdot g(x) = g(x)$.

    \item[(b)] A unit is an element that is invertible. The units of $\mathrm{Fun}(I, R)$ are all the functions $f$ such that $\forall x \in [0, 1], f(x) \neq 0$. The inverse of such an element $f$ is $g(x) = 1/f(x)$, since $f(x)g(x) = f(x) \cdot 1/f(x) = 1$. If there is an $x_0$ such that $f(x_0) = 0$, then $g(x_0)$ is not defined, and so $f$ would have no inverse. 

    \item[(c)] A zero divisor is a non-zero element $a$ such that there exists a non-zero element $b$ satisfying $ab = 0$. The zero divisors of $\mathrm{Fun}(I, R)$ are the non-zero functions $f$ such that there is an $x_0$ satisfying $f(x_0) = 0$. For such an $f$, the function $g$ satisfying $f(x)g(x) = 0$ is defined as follows: 
    $$g(x) = \begin{cases}
        0 \text{ if } f(x) \neq 0 \\
        1 \text{ if } f(x) = 0
    \end{cases}$$

    Then $\forall x \in [0, 1]$ one of $f(x)$ or $g(x)$ is 0, so $f(x)g(x) = 0$. Importantly, since $f$ is non-zero but has an $x_0$ satisfying $f(x_0) = 0$, $g$ is also non-zero and has an $x_1$ satisfying $g(x_1) = 0$. 

    \item[(d)] To show $S$ is a subring, we show that is is non-empty and closed under subtraction and multiplication. 

    \ss 

    To show $S$ is non-empty, we see that $f: [0, 1] \rarr \R$ defined by $f(x) = 0$ is an element of $S$, since $f(1/2) = 0$.

    \ss 

    Now suppose $f, g \in S$. We want to show $f - g \in S$. To do this, we see that $(f - g)(1/2) = f(1/2) - g(1/2) = 0 - 0 = 0$, so $f - g \in S$.

    \ss 

    Now suppose $f, g \in S$. We want to show $fg \in S$. To do this, we see that $(fg)(1/2) = f(1/2)g(1/2) = 0 \cdot 0 = 0,$ so $fg \in S$.

    \ss 

    Since $S$ is non-empty and closed under both subtraction and multiplication, it is a subring.
\end{itemize}
\end{solution}

\begin{problem}{Problem 2}

\pre Which of the following are subrings of $\Q$. (Give a brief justification in each case.)

\begin{itemize}
    \item[(a)] the set of all rational numbers with even denominators (when reduced)
    \item[(b)] the set of all rational numbers with odd denominators (when reduced)
    \item[(c)] the set of all squares of rational numbers
\end{itemize}
    
\end{problem}

\begin{solution}
\begin{itemize}
    \item[(a)] This set is not a subring, since $1/2$ is a rational number with an even denominator when reduced, but $1/2 + 1/2 = 1$, and $1$ is not a rational number with an even denominator. 

    \item[(b)] We denote this set as $S$. To show $S$ is a subring, we show that it is non-empty and closed under subtraction and multiplication. 

    \ss 

    Since $1/3$ is a rational number with an odd denominator when reduced, so $1/3 \in S$.

    \ss

    Let $a, b \in S$, so $a = m/n$ with $n$ odd, and $b = p/q$ when $q$ odd. Then $a - b = (mq - bn)/nq$, and since $n$ and $q$ are odd, $nq$ is odd. Note that even if this fraction needs to be reduced, the denominator will remain odd, since an odd number cannot reduce to an even number, so $a - b \in S$. 

    \ss 

    Let $a, b \in S$, so $a = m/n$ with $n$ odd, and $b = p/q$ when $q$ odd. Then $ab = mp/nq$, and since $n$ and $q$ are odd, $nq$ is odd. Note that even if this fraction needs to be reduced, the denominator will remain odd, since an odd number cannot reduce to an even number, so $ab \in S$.

    \ss

    Since $S$ is non-empty and closed under subtraction and multiplication, it is a subring.

    \item[(c)] This set is not a subring, since $1$ is a square of a rational number, but $1 + 1 = 2$ is not the square of a rational number. 
\end{itemize}
\end{solution}

\begin{problem}{Problem 3}

\pre Consider the power series ring $\Q[[x]]$ that consists of elements of the form $\sum_{i = 0}^\infty a_i x^i$. 
The sum of two power series 
$$\sum_{i = 0}^\infty a_i x^i+ \sum_{j = 0}^\infty b_j x^j = \sum_{k = 0}^\infty (a_k +b_k) x^k.$$
The product of two power series $$\sum_{i = 0}^\infty a_i x^i \sum_{j = 0}^\infty b_j x^j = \sum_{k = 0}^\infty c_k x^k$$ where $c_k = \sum_{\ell = 0}^k a_\ell b_{k- \ell}$.

\begin{itemize}
	\item[(a)] Multiply the power series $1+x$ and $1 - x + x^2 - x^3 + x^4 - x^5 + \dots$, in other words, $\sum_{i = 0}^\infty (-1)^i x^i$.
	\item[(b)] Suppose $A = \sum_{i = 0}^\infty a_i x^i$ has an inverse $B = \sum_{j = 0}^\infty b_j x^j$. Prove that $b_0 = 1/a_0$. Conclude that $A$ has an inverse only if $a_0 \neq 0$.
	\item[(c)] What is the inverse of the power series $1+x$? 
	\item[(d)] Note that the polynomials $\Q[x]$ are a subring of $\Q[[x]]$ but $1+x \in \Q[x]$ is not a unit in $\Q[x]$. Deduce that if $S \subset R$ is a subring of $R$ then the units of $S$ are units of $R$ but not all of the units of $R$ that are contained in $S$ are also units of $S$.
	\item[(e)] Bonus, not graded: show that $A$ is invertible if and only if $a_0 \neq 0$.
\end{itemize}
\end{problem}

\begin{solution}
\begin{itemize}
    \item[(a)] We can represent $1 + x$ as the power series $\sum_{i = 0}^\infty a_ix^i$ where $a_i = 1$ for $i = 0, 1$ and $a_i = 0$ otherwise, and we can represent $1 - x + x^2 - x^3 + \dots$ as the power series $\sum_{j = 0}^\infty b_jx^j$ where $b_j = (-1)^j$. If we compute $c_k = \sum_{\ell = 0}^k a_\ell b_{k - \ell}$, we see that if $k = 0$, $c_k = a_0b_0 = 1$, and if $k \geq 1$, $c_k = a_0b_1 + a_1b_0$, since each term after that has a factor of $a_i$, and is thus equal to 0. But, $a_0b_1 + a_1b_0 = b_1 + b_0 = 0$, since $(-1)^1 + (-1)^0 = 0$. Thus, $c_k = 1$ when $k=0$ and $c_k = 0$ otherwise. Now, we use the formula for the product of two power series to get the following:

    \begin{align*}
        (1 + x)(1 - x + x^2 - x^2 + \dots) &= \left(\sum_{i = 0}^\infty a_ix^i\right)\left(\sum_{j = 0}^\infty (-1)^jx^j\right) \\
        &= \sum_{k=0}^\infty c_k x^k \\
        &= c_0x^0 + \sum_{k=1}^\infty c_k x^k \\
        &= 1 + \sum_{k=1}^\infty 0 x^k \\
        &= 1
    \end{align*}

    \item[(b)] If $A = \sum_{i = 0}^\infty a_ix^i$ has an inverse $B = \sum_{j=0}^\infty b_jx^j$, then $AB = 1$. Using the formula for the product of two power series, we get the following:

    \begin{align*}
        AB &= \left(\sum_{i = 0}^\infty a_ix^i\right)\left(\sum_{j=0}^\infty b_jx^j\right) \\
        &= \sum_{k=0}^\infty c_k x^k \\
        &= c_0x^0 + \sum_{k=1}^\infty c_k x^k \\
        &= a_0b_0 + \sum_{k=1}^\infty c_k x^k \\
        &= 1
    \end{align*}

    Since each term in the sum has a factor of $x$, but $1$ is a constant term, the sum must evaluate to 0, and we are left with $a_0b_0 = 1$. Thus, $b_0 = 1/a_0$. 

    \ss

    So, if $A = \sum_{i = 0}^\infty a_ix^i$ has an inverse $B = \sum_{j=0}^\infty b_jx^j$, it must be that $b_0 = 1/a_0$, which is only defined if $a_0 \neq 0$. Thus, $A$ has an inverse only if $a_0 \neq 0$.

    \item[(c)] As we computed in part (a), the power series $1 - x + x^2 - x^3 + \dots$ multiplied with the power series $1 + x$ evaluates to $1$. Since inverses are unique in rings, the inverse of the power series $1 + x$ is $1 - x + x^2 - x^3 + \dots$.

    \item[(d)] A unit of a subring $S$ is an element which has a multiplicative inverse in $S$. If $a \in S$ is a unit, then there exists a $b \in S$ such that $ab = 1$. But since $S \subset R$, $b \in R$, and so $a$ is a unit in $R$. 

    \ss

    However, as seen in the example of $\Q[[x]]$ and $\Q[x]$, it is not the case that all the units of $R$ that are contained in $S$ are units in $S$, since the inverse of a unit in $R$ might not be contained in $S$.
\end{itemize}
\end{solution}

\begin{problem}{Problem 4}

\pre Prove that $\Q(\sqrt[3]{2}) = \{ a + b\sqrt[3]{2}+c \sqrt[3]{4}  | a,b, c \in \Q\}$ is a ring. Most of the properties should be easy, but the closure under multiplication will require some justification.
    
\end{problem}

\begin{solution}
To show that $\Q(\sqrt[3]{2})$ is a ring, we will show that it has two binary operations, addition and multiplication, which satisfy the following properties for all $a, b, c \in \Q(\sqrt[3]{2})$:

\begin{itemize}
    \item[(1)] $a + b = b + a$ 
    \item[(2)] $(a + b) + c = a + (b + c)$
    \item[(3)] There is an additive identity $0$ such that $a + 0 = a$
    \item[(4)] There is an element $-a \in \Q(\sqrt[3]{2})$ such that $a + (-a) = 0$
    \item[(5)] $(ab)c = a(bc)$
    \item[(6)] $a(b + c) = ab + ac$ and $(b + c)a = ba + ca$
\end{itemize}

First, if $x, y \in \Q(\sqrt[3]{2})$, then $x = a_1 + b_1\sqrt[3]{2} + c_1\sqrt[3]{4}$ and $y = a_2 + b_2\sqrt[3]{2} + c_2\sqrt[3]{4}$. So, $x + y = (a_1 + a_2) + (b_1 + b_2)\sqrt[3]{2} + (c_1 + c_2)\sqrt[3]{4} \in \Q(\sqrt[3]{2})$. As well, $xy = (a_1a_2 + 2b_1c_2 + 2b_2c_1) + (a_1b_2 + a_2b_1 + 2c_1c_2)\sqrt[3]{2} + (a_1c_1 + b_1b_2 + a_2c_1)\sqrt[3]{4} \in \Q(\sqrt[3]{2})$. $\Q(\sqrt[3]{2})$ is therefore closed under both addition an multiplication.

\begin{description}
    \item[(1)] Since any elements $a, b \in \Q(\sqrt[3]{2})$ are elements in $\R$, and addition in $\R$ is commutative, addition in $\Q(\sqrt[3]{2})$ is commutative.
    \item[(2)] Since any elements $a, b \in \Q(\sqrt[3]{2})$ are elements in $\R$, and addition in $\R$ is associative, addition in $\Q(\sqrt[3]{2})$ is associative.
    \item[(3)] The additive identity is $0$, which is the identity for standard addition.
    \item[(4)] The inverse of an element $a + b\sqrt[3]{2} + c\sqrt[3]{4} \in \Q(\sqrt[3]{2})$ is $-a + (-b)\sqrt[3]{2} + (-c)\sqrt[3]{4}$, since $(a + b\sqrt[3]{2} + c\sqrt[3]{4}) + (-a + (-b)\sqrt[3]{2} + (-c)\sqrt[3]{4}) = 0$.
    \item[(5)] Since any elements $a, b \in \Q(\sqrt[3]{2})$ are elements in $\R$, and multiplication in $\R$ is associative, multiplication in $\Q(\sqrt[3]{2})$ is associative.
    \item[(6)] Since any elements $a, b, c \in \Q(\sqrt[3]{2})$ are elements in $\R$, and elements in $\R$ satisfy $a(b+c) = ab+ac$ and $(b+c)a = ba + ca$, elements in $\Q(\sqrt[3]{2})$ satisfy $a(b+c) = ab+ac$ and $(b+c)a = ba + ca$.
\end{description}
\end{solution}

\begin{problem}{Problem 5}

\pre Let $X$ be a set and let $P(X)$ be the power set of $X$, that is, the set of all subsets of $X$.  (For example, if $X = \{ a, b, c\}$ then $P(X) = \{ \emptyset, \{a\}, \{b \}, \{c\}, \{a, b\}, \{a, c\}, \{b, c \}, \{a, b, c\}\}$).  

\begin{itemize}
	\item[(a)]  Show that we can make $P(X)$ into a ring where if $Y,Z \in P(X)$ then $Y + Z = (Y - Z) \cup (Z - Y)$ and $Y \cdot Z = Y \cap Z.$  
	\item[(b)] What is the multiplicative identity? What is the additive identity?  
	\item[(c)] What is $Y \cdot Y$ for any $Y\in P(X)$?
\end{itemize}
\end{problem}

\begin{solution}
\begin{itemize}
    \item[(a)] 
    To show that we can turn $P(X)$ into a ring $R$ with the given operations, we will show that for any $A, B, C \in R$, the operations satisfy:

    \begin{itemize}
        \item[(1)] $A + B = B + A$ 
        \item[(2)] $(A + B) + C = A + (B + C)$
        \item[(3)] There is an additive identity $0$ such that $A + 0 = A$
        \item[(4)] There is an element $-A \in R$ such that $A + (-A) = 0$
        \item[(5)] $(AB)C = A(BC)$
        \item[(6)] $A(B + C) = AB + AC$ and $(B + C)A = BA + CA$
    \end{itemize}

    First, since $A$ and $B$ consist of elements of $X$, $A + B = (A - B) \cup (B - A) \in P(X)$ and $AB = A \cap B \in P(X)$. $R$ is therefore closed under both addition and multiplication.

    We will make use of the fact that set unions are associative and commutative, as is the case for set intersections. As well, $(A \cap B)^c = A^c \cup B^c$.

    \begin{description}
        \item[(1)] \begin{align*}
            A + B &= (A - B) \cup (B - A) \\
            &= (B - A) \cup (A - B) & \text{unions are commutative}\\
            &= B + A
        \end{align*}
        \item[(2)] \begin{align*}
            (A + B) + C &= ((A - B) \cup (B - A)) + C \\
            &= (((A - B) \cup (B - A)) - C) \cup (C - ((A - B) \cup (B - A))) \\
            &= (A \cup B \cup C) \cap (A \cup B^c \cup C^c) \cap (A^c \cap B \cap C^c) \cap (A^c \cup B^c \cup C) \\
            &= (B \cup C \cup A) \cap (B \cup C^c \cup A^c) \cap (B^c \cap C \cap A^c) \cap (B^c \cap C^c \cap A) \\
            &= (B + C) + A \\
            &= A + (B + C) \text{\hspace{8em}addition is commutative}
        \end{align*}

        These equations follow from the definitions of addition and multiplication in $R$ and the aforementioned properties of sets. 
        \item[(3)] The additive identity is $\emptyset$, since $A + \emptyset = (A - \emptyset) \cup (\emptyset - A) = A \cup \emptyset = A$.
        \item[(4)] The inverse of $A$ is $A$ since $A + A = (A - A) \cup (A - A) = \emptyset \cup \emptyset = \emptyset$.
        \item[(5)] \begin{align*}
            (AB)C &= (A \cap B)C \\
            &= (A \cap B) \cap C \\
            &= A \cap (B \cap C) & \text{intersection is associative}\\
            &= A \cap (BC) \\
            &= A(BC)
        \end{align*}
        \item[(6)] We want to show that $A(B+C) = AB + AC$, which we will do by showing they are both equal to $(A \cap (B - C)) \cup (A \cap (C - B))$.

        \begin{align*}
            A(B+C) &= A \cap ((B - C) \cup (C - B)) \\
            &= (A \cap (B - C)) \cup (A \cap (C - B))
        \end{align*}

        As well,
        \begin{align*}
            AB + AC &= (A \cap B) + (A \cap C) \\
            &= ((A \cap B) - (A \cap C)) \cup ((A \cap C) - (A \cap B)) \\
            &= ((A \cap B) \cap (A^c \cup C^c)) \cup ((A \cap C) \cap (A^c \cup B^c)) \\
            &= ((A \cap B \cap A^c) \cup (A \cap B \cap C^c)) \cup ((A \cap C \cap A^c) \cup (A \cap C \cap B^c)) \\
            &= (A \cap B \cap C^c) \cup (A \cap C \cap B^c) \\
            &= (A \cap (B - C)) \cup (A \cap (C - B))
        \end{align*}

        Since both $A(B+C)$ and $AB + AC$ are equal to $(A \cap (B - C)) \cup (A \cap (C - B))$, they are equal to each other. 

        \ss

        We must also show that $(B+C)A = BA + CA$, which is true by the following:

        \begin{align*}
            (B+C)A &= ((B - C) \cup (C - B)) \cap A \\
            &= A \cap ((B - C) \cup (C - B)) & \text{intersection is commutative}\\
            &= A(B+C) \\
            &= AB + AC & \text{just proved}\\
            &= (A \cap B) + (A \cap C) \\
            &= (B \cap A) + (C \cap A) & \text{intersection is commutative}\\
            &= BA + CA
        \end{align*}
    \end{description}
    
    \item[(b)] The multiplicative identity is $X$, since for any $Y \in P(X)$, $Y \cdot X = Y \cap X = Y$ and $X \cdot Y = X \cap Y = Y$. The additive identity is $\emptyset$ since $Y + \emptyset = (Y - \emptyset) \cup (\emptyset - Y) = Y \cup \emptyset = Y$.
    \item[(c)] $Y \cdot Y = Y \cap Y = Y$ for any $Y \in P(X)$
\end{itemize}
\end{solution}

\begin{problem}{Problem 6}

\pre Reflect on your process for working on problem sets.  Here are some questions you can address:

\begin{itemize}
	\item Do you solve them alone or in a group? How much time do you spend thinking about the problems alone versus in office hours or with peers? 
	\item Do you go to office hours? 
	\item How long do you spend on the problem set? 
	\item What do you do when you get stuck?
	\item  What is succeeding about your process, and what do you want to change?
	\item Do you enjoy the problem sets?
\end{itemize}
	 Be honest with yourself. You are receiving full credit just for providing a thoughtful answer here.
    
\end{problem}

\begin{solution}
Some weeks, I solve them in a group, others I do it individually, usually depending on whether my friends and I can find a time to all work together. I like to read the problem sets as soon as they come out, even if I won't necessarily start them immediately, just so that I can think about the problems in the background. When I work with others, we write out the problem and try to all solve it on a white/chalk board. If one of us is not understanding, we try to explain it to the other, which also deepens our own understanding. 

I do not usually go to office hours. 

It varies from problem set to problem set. On this one I spent an hour on problems and another hour or so typing. On previous ones I've spent up to 4 hours solving problems and another hour typing.

I usually take a break, either by doing something completely different or working on a different question. If I am really stuck I may talk to a friend to see if we can figure it out together, or reread the textbook to see if there are any ideas that may be of use. 

I think I am doing well in solving the problems, but I sometimes miss a few details in my proof, or mess something up with my notation, so I should probably spend some more time checking over my answers after I've written them. 

I really enjoy the problem sets because I think they have a great flow to them. In other classes, each question is distinct, but here there is a very clear connection from question to question, which I appreciate.
\end{solution}

\end{document}