\documentclass{hmwk}

\hdr{Problem Set 8}{MATH 1530: Abstract Algebra}{Aly Rajwani}
\hwk{8}

\begin{document}
\maketitle

\begin{problem}{Problem 1}
Basics on ideals.
\begin{itemize}
    \item[(a)] Show that if $I, J \subset R$ are ideals then so is $I \cap J$.
    \item[(b)] Show that if $\phi: R \to S$ is a ring homomorphism and $J$ is an ideal of $S$ then $\phi^{-1} (J) = \{ r \in R | \phi(r) \in J\}$ is an ideal.  If $I$ is an ideal of $R$, is $\phi(I)$ necessarily an ideal of $S$?
\end{itemize}
\end{problem}

\begin{solution}
\begin{itemize}
    \item[(a)] 
    Recall that an ideal is defined as a subring $I$ of a ring $R$ such that for every $r \in R$ and every $a \in I$, we have $ra \in I$. 

    To show  that $I \cap J$ is an ideal, we will apply the ideal test, showing that $a - b \in I \cap J$ whenever $a, b \in I \cap J$ and  that $ra \in I \cap J$ whenever $r \in R$ and $a \in I \cap J$. 

    Let $a, b \in I \cap J$. Then $a, b \in I$, and since $I$ is an ideal, $a - b \in I$. Similarly, $a - b \in J$. Thus, $a - b \in I \cap J$.

    Let $r \in R$ and $a \in I \cap J$. Then $a \in I$, and since $I$ is an ideal, $ra \in I$. Similarly, $ra \in J$. Thus, $ra \in I \cap J$.

    Since $I \cap J$ satisfies these properties, $I \cap J$ is an ideal. 

    \item[(b)] To show that $\phi^{-1}(J)$ is an ideal, we will apply the ideal test, showing that $a - b \in \phi^{-1}(J)$ whenever $a, b \in \phi^{-1}(J)$ and $ra \in \phi^{-1}(J)$ whenever $r \in R$ and $a \in \phi^{-1}(J)$. 

    Let $a, b \in \phi^{-1}(J)$. Then $\phi(a), \phi(b) \in J$. Since $J$ is an ideal, $\phi(a) - \phi(b) \in J$, and since $\phi$ is a ring homomorphism, $\phi(a - b) \in J$. Thus, $a - b \in \phi^{-1}(J)$. 

    Let $r \in R$ and $a \in \phi^{-1}(J)$. Then $\phi(a) \in J$. Since $J$ is an ideal of $S$, and since $\phi(r) \in S$, we have $\phi(r)\phi(a) = \phi(ra) \in J$. Thus, $ra \in \phi^{-1}(J)$.

    Since $\phi^{-1}(J)$ satisfies these properties, $\phi^{-1}(J)$ is an ideal.

    It is not necessarily true that if $I$ is an ideal of $R$, then $\phi(I)$ is an ideal of $S$, since $\phi$ is not necessarily a surjective map. 

    Let $\phi: \Z \rarr \Z$ be defined by $\phi(n) = n \mod 4$. Let $I = 2\Z$, which we know is an ideal of $\Z$. Then $\phi(I) = \{0, 2\}$. However, $3 \in \Z$, and $3 \cdot 2 = 6 \notin \{0, 2\}$. 
\end{itemize}
\end{solution}

\begin{problem}{Problem 2}
    Show that $\{0\}$ is a maximal ideal of a commutative ring $S$ with unity if and only if $S$ is a field.  
\end{problem}

\begin{solution}

\pre If $S$ is a field, the only proper ideal is $\{0\}$, since if $0 \neq a \in I$ for an ideal $I$, then $ba \in I$ for all $b \in S$, and so $x \in S$ implies $(xa^{-1})a = x \in I$, and so $I = S$, and $I$ is not a proper ideal. Since $\{0\}$ is trivially ideal, it must be a maximal ideal.

\pre Suppose $\{0\}$ is a maximal ideal. Let $x \in S$ be a non-unit element. Then $x \in xS$ since $1 \in S$, and $1 \notin xS$ since this would imply $x$ has a multiplicative inverse, but it was supposed to be a non-unit. But this implies $xS$ is a proper ideal of $S$, so $\{0\} \subsetneq xS \subsetneq S$, which contradicts $\{0\}$ being maximal. Thus, every $x \in S$ must be a unit, so $S$ is a field. 
\end{solution}

\begin{problem}{Problem 3}
(6 points) The following pair of propositions are key. Suppose $R$ is a commutative ring with unity. In the following exercises, give an alternative proof than the book that quotienting by a maximal ideal gives a field by considering for an ideal $I \subset R$ the map $\pi: R \to R/I$. We also consider the quotient ring of prime ideals.
\begin{itemize}
    \item[(a)] Prove the following lemma. Let $R$ be a multiplicative ring with identity and $I$ an ideal of $R$. There is a one-to-one correspondence between the ideals of $R$ that contain $I$ and the ideals of the quotient ring $R/I$. The correspondence takes an ideal $J$ such that $I\subset J$ and sends it to $\pi(J) =  \{a + I | a \in J\} \subset R/I$. \\
    Hint: The correspondence has an inverse, namely it sends $J \subset R/I$ to $\pi^{-1}(J)$.  
    \item[(b)]  Let $M$ be an ideal. Show that $R/M$ is a field if and only if $M$ is a maximal ideal. 
    \item[(c)]  Let $P$ be an ideal. Show that $R/P$ is an integral domain if and only if $P$ is prime.
    \item[(d)] Deduce that every maximal ideal is prime, but not every prime ideal is maximal.
\end{itemize}
\end{problem}

\begin{solution}
\begin{itemize}
    \item[(a)] Consider the set $C$ containing the ideals $J$ which contain $I$, and the set $D$ containing the ideal of $R/I$. We define the map $\pi: C \rarr D$ as $\pi(J) = \{a + I : a \in J\}$ and the map $\pi^{-1}: D \rarr C$ as $\pi^{-1}(K) = \{a : a + I \subset K\}$. We will show that $\pi(\pi^{-1}(K)) = K$ and $\pi^{-1}(\pi(J)) = J$, which will prove that $\pi$ is a bijection, and so there is a one-to-one correspondence between the ideals of $R$ containing $I$ and the ideals of the quotient ring $R/I$. 

    $\pi(\pi^{-1}(K)) = \{a + I : a \in \pi^{-1}(K)\} = \{a + I: a + I \in K\} = K$, where all equalities follow from the definitions. 

    $\pi^{-1}(\pi(J)) = \{a : a + I \in \pi(J)\} = \{a : a + I = b + I, b \in J\}$. Clearly $J \subset \pi^{-1}(\pi(J))$ since this holds for any function. Now let $a \in \pi^{-1}(\pi(J))$. Then $a + I = b + I$ for some $b \in J$, so $a \in b + I$. This implies $a - b \in I \subset J$, and so $a \in b + J$ implies $a \in J$. Thus, $\pi^{-1}(\pi(J)) = J$. 

    Thus, $\pi$ is a bijection, and there is a one-to-one correspondence between the ideals of $R$ containing $I$ and the ideals of the quotient ring $R/I$. 

    \item[(b)] Suppose $M$ is a maximal ideal. Then there are no proper ideals of $R$ which strictly contain $M$. By part (a), there is a one-to-one correspondence between the ideals of $R$ which contain $M$ and the ideals of $R/M$. Since the only ideals which contain $M$ if $M$ is maximal are $M$ and $R$, the only ideals of $R/M$ are $R/M$ itself and $\{0\}$. Thus, $R/M$ has no nontrivial ideals, and so it is a field. 

    Suppose $R/M$ is a field. Then it has no ideals other than $\{0\}$ and $R/M$. By part (a), there is a one-to-one correspondence between the ideals of $R$ which contain $M$ and the ideals of $R/M$. Thus, the only ideals of $R$ which contain $M$ are $M$ and $R$, and so $M$ is a maximal ideal. 

    \item[(c)] An integral domain is a commutative ring with identity and no zero divisors. Suppose $R/P$ is an integral domain, so $(a + P)(b + P) = 0 + P$ implies $a + P = P$ or $b + P = P$. To show that $P$ is prime, let $a, b \in R$ and suppose $ab \in P$. Then $(a + P)(b + P) = ab + P = P$. Then either $a + P = P$ or $b + P = P$, so either $a \in P$ or $b \in P$. Thus, by definition, $P$ is a prime ideal. 

    Suppose $P$ is prime, so $a, b \in R$ and $ab \in P$ implies that $a \in P$ or $b \in P$. Suppose $(a + P)(b + P) = 0 + P$. To show $R/P$ is an integral domain, we must show that one of $a + P$ or $b + P$ is $P$. This immediately follows from the fact that $(a + P)(b + P) = ab + P = P$, so $ab \in P$, and since $P$ is prime $a \in P$ or $b \in P$, so $a + P = P$ or $b + P = P$, and so $R/P$ has no zero divisors. 
    
    Thus, $R/P$ is an integral domain if and only if $P$ is prime. 

    \item[(d)] Let $M$ be a maximal ideal. Then $R/M$ is a field by part (b), and so it is an integral domain. Thus, $M$ is prime by part (c). 

    Consider the ideal $\{0\}$ in $\Z$. Then $\{0\}$ is prime, since if $a, b \in \Z$ and $ab \in \{0\}$, then $ab = 0$ and so $a = 0$ or $b = 0$, so $a \in \{0\}$ or $b \in \{0\}$. However, every ideal in $\Z$ contains $\{0\}$, so it is not maximal. 
\end{itemize}
\end{solution}

\begin{problem}{Problem 4}
Ideals, examples. Recall that $I = [0,1]$ and $\mathrm{Fun}(I,\R)$ is the ring of functions from $I \to \R$.
\begin{itemize}
    \item[(a)] Show that $M_p \subset \mathrm{Fun}(I,\R)$ defined as $M_p = \{ f : I \to \R | f(p) = 0\}$ is a maximal ideal of $\mathrm{Fun}(I,\R)$ for each $p \in [0,1]$.
    \item[(b)] What is the ideal $(4, 6)$ generated by the elements $4$ and $6$ in $\Z$ (i.e. what is the smallest ideal containing $4$ and $6$ in $\Z$)?  What about $(2,3)$?  What about $(m,n)$? 
\end{itemize}
\end{problem}

\begin{solution}
\begin{itemize}
    \item[(a)] Consider the ring homomorphism $\phi: \mathrm{Fun}(I, \R) \rarr \R$ defined by $\phi(f) = f(p)$. Then the kernel of $\phi$ is $M_p$, and $\phi$ is a surjective function, since for any $x \in \R$, if $f$ is the constant function equal to $x$, then $\phi(f) = x$. So, we apply the first isomorphism theorem, which shows that $\R \cong \mathrm{Fun}(I, \R)/M_p$. Since $\R$ is a field, $\mathrm{Fun}(I, \R)/M_p$ must also be a field.

    $\funir$ is a commutative ring with identity, since the operations are point-wise addition and multiplication, which are commutative, and the identity is $f = 1$. Thus, we can apply theorem 14.4 from Gallian to say that since $\funir$ is a commutative ring with identity and $\mathrm{Fun}(I, \R)/M_p$ is a field, $M_p$ must be maximal. Since $p$ was arbitrary, $M_p$ is a maximal ideal of $\mathrm{Fun}(I, \R)$ for each $p \in [0, 1]$. 

    \item[(b)] By Bezout's Lemma, an ideal containing both 4 and 6 must contain their greatest common divisor, which is 2. Since it is an ideal, it must contain every multiple of two, so an ideal containing 4 and 6 must at least contain $2\Z$. Since $2\Z$ is itself an ideal as shown in the textbook, it is the smallest ideal which contains 4 and 6. 
    
    Since the greatest common divisor of 2 and 3 is 1, an ideal containing 2 and 3 must contain $\Z$, and so $\Z$ is the smallest ideal containing 2 and 3. 

    An ideal containing $m$ and $n$ also contains $\gcd(m, n)$ and so it contains $\gcd(m, n)\Z$. Since $\gcd(m, n)\Z$ is itself an ideal, it is the smallest ideal which contains $m$ and $n$. 
\end{itemize}
\end{solution}

\begin{problem}{Problem 5}
Let \[J = \{ f \in \mathrm{Fun}(I,\R) | f(1/2) = f(1/3) = 0\} .\] Describe the quotient ring $\mathrm{Fun}(I,\R)/J$.
\end{problem}

\begin{solution}
    
    
    \pre A coset of $\mathrm{Fun}(I, \R)/J$ is a set of functions which have the same value on $1/2$ and $1/3$, that is, a set of the form $\{f: f(1/2) = a, f(1/3) = b\}$ for some $a, b \in \R$. We can therefore define a function $\phi: \mathrm{Fun}(I, \R) \rarr \R \times \R$ defined by $\phi(f) = (f(1/2), f(1/3))$. This is a surjective map with kernel $J$, so by the first isomorphism theorem, $\mathrm{Fun}(I, \R)/J$  is isomorphic to $\R \times \R$.
\end{solution}
\end{document}