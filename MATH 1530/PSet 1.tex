\documentclass{article}
\usepackage{graphicx} % Required for inserting images
\usepackage{amsmath}
\usepackage{amsthm}
\usepackage{amssymb}

\title{Math 1530 PSet 1}
\author{Aly Rajwani}
\date{September 2024}
\begin{document}

\maketitle

\begin{enumerate}
    \item Hi! My name is Aly and I use he/him pronouns. I don't have any classroom experience with proof-writing, but I have written some proofs in my self-studying. I'm taking this course because I like math, this course seems very interesting, I like the professor, and it's also a degree requirement. 

    \item 
        \textbf{Base Case:} $n = 1$
        $$1 = \frac{6}{6} = \frac{1 \cdot (1 + 1) \cdot (2 + 1)}{6}$$
    
        
        \textbf{Induction Hypothesis:}
        
        Suppose that $1^2 + 2^2 + \dots + n^2 = \frac{n\cdot (n+1) \cdot (2n + 1)}{6}$. 
        
        Then $1^2 + 2^2 + \dots + n^2 + (n+1)^2 = \frac{(n+1)\cdot ((n+1)+1) \cdot (2(n+1)+1)}{6}$
        
    
        \textbf{Proof:}
        
        We start by supposing $1^2 + 2^2 + \dots + n^2 = \frac{n\cdot (n+1) \cdot (2n + 1)}{6}$.
    
        Adding $(n+1)^2$ to both sides, we get $1^2 + 2^2 + \dots + n^2 +(n+1)^2= \frac{n\cdot (n+1) \cdot (2n + 1)}{6} + (n+1)^2$.
    
        Then we factor an rearrange terms on the right side of the equation in the following way:
        \begin{align*}
            \frac{n\cdot (n+1) \cdot (2n + 1)}{6} + (n+1)^2 &= \frac{n\cdot (n+1) \cdot (2n + 1) + 6(n+1)^2}{6} \\
            &= \frac{(n+1) \cdot (n \cdot (2n+1) + 6(n+1))}{6} \\
            &= \frac{(n+1) \cdot (2n^2 + 7n + 6)}{6} \\
            &= \frac{(n+1) \cdot (n+2) \cdot (2n+3)}{6} \\
            &= \frac{(n+1)\cdot ((n+1)+1) \cdot (2(n+1)+1)}{6}
        \end{align*}
    
        Thus, $1^2 + 2^2 + \dots + (n+1)^2 = \frac{(n+1)\cdot ((n+1)+1) \cdot (2(n+1)+1)}{6}$, which completes the induction hypothesis. 
    
        By induction, we have shown that $1^2 + 2^2 + \dots + n^2 = \frac{n \cdot (n+1) \cdot (2n + 1)}{6}$
    
        \rightline\qedsymbol

    \item 
        \begin{enumerate}
            \item The equivalence classes are as follows:

            $[0] = \{7n : n \in \mathbb{Z}\}$ \\
            $[1] = \{7n + 1: n \in \mathbb{Z}\}$ \\
            $[2] = \{7n + 2: n \in \mathbb{Z}\}$ \\
            $[3] = \{7n + 3: n \in \mathbb{Z}\}$ \\
            $[4] = \{7n + 4: n \in \mathbb{Z}\}$ \\
            $[5] = \{7n + 5: n \in \mathbb{Z}\}$ \\
            $[6] = \{7n + 6: n \in \mathbb{Z}\}$

            \item 
            This relation is not reflexive, since $1 + 1 = 2$, and 7 does not divide 2.

            This relation is symmetric, since if $a \sim b$, then $a + b = 7k$, so $b + a =7k$, so $b \sim a$ 

            This relation is not transitive, since 3 + 4 = 7, and 4 + 3 = 7, but 3 + 3 = 6, and 7 does not divide 6. 

            \item 
            This relation is not reflexive, since $1 \nless 1$.

            This relation is not symmetric, since $1 < 2$ but $2 \nless 1$. 

            This relation is transitive, since if $a \sim b$ and $b \sim c$ then $a < b$ and $b < c$, so $a < b < c$ which implies $a < c$, so $a \sim c$.

            \item 
            This relation is not reflexive, since $1^2 - 2\cdot 1^2 = -1$, and 7 does not divide $-1$.

            This relation is not symmetric, since $3^2 - 2 \cdot 1^2 = 7$, but $1^2 - 2 \cdot 3^2 = -17$, and 7 does not divide $-17$.

            This relation is not transitive, since $1^2 - 2 \cdot 2^2 = -7$, $2^2 - 2 \cdot 3^2 = -14$, but $1^2 - 2 \cdot 3^2 =-17$, and 7 does not divide $-17$. 
        \end{enumerate}

    \item 
        \begin{enumerate}
            \item To prove this relation is an equivalence relation, we must show that it is reflexive, symmetric, and transitive.

            Since each $S_i$ is non-empty, suppose $a \in S_i$. Clearly $a$ is in the same $S_i$ as itself, and can be in no other set by definition of a disjoint union, so $a \sim a$. This is true of any $a$ and $S_i$, so the relation is reflexive. 

            Suppose $a \sim b$, meaning $a, b \in S_i$. This implies $b, a \in S_i$, so $b \sim a$. This applies to any $a, b,$ and $S_i$, so the relation is symmetric. 

            Suppose $a \sim b$ and $b \sim c$. Then $a, b \in S_i$ and $b, c \in S_j$. Since the sets are disjoint, if $b \in S_i$ and $b \in S_j$, $i = j$, so we have $a, b, c \in S_i$. Thus, $a, c \in S_i$, so $a \sim c$, so the relation is transitive. 

            These 3 properties mean $\sim$ is an equivalence relation.

            \item Consider an equivalence class $[a]$ in the relation $\sim$. By definition of this relation, the members of this equivalence class are exactly the elements that were in the same $S_i$ as $a$, and so $[a] = S_i$. Since equivalence relations partition sets, each element of $S$ belongs to one subset of the partition, and as shown before, each equivalence class is exactly equal to some subset $S_i$ in the original partition. Thus, the equivalence classes of the relation $\sim$ are exactly the disjoint union we started with. 
        \end{enumerate}

    \item 
        \textbf{Base Case: } $n = 2$
        $$1 + \frac{1}{2^2} = \frac{5}{4} < \frac{6}{4} = 2 - \frac{1}{2}$$

        \textbf{Induction Hypothesis:}

        
        Suppose that $1 + \frac{1}{4} + \dots + \frac{1}{n^2} < 2 - \frac{1}{n}$.

        Then $1 + \frac{1}{4} + \dots + \frac{1}{n^2} + \frac{1}{(n+1)^2} < 2 - \frac{1}{n+1}$

    \item Let $S$ be the set of all people. For $a, b \in S$, define $a \sim b$ if the parents of $a$ are the parents of $b$. 

    Clearly this relation is reflexive, since the parents of $a$ are also the parents of $a$, meaning $a \sim a$.

    If $a \sim b$, then the parents of $a$ are the parents of $b$, so the parents of $b$ are the parents of $a$, so $b \sim a$. Thus, this relation is symmetric.

    If $a \sim b$ and $b \sim c$, then the parents of $a$ are the parents of $b$, and the parents of $b$ are the parents of $c$. Since $a$ shares parents with $b$, and those same parents $b$ shares with $c$, then $a$ shares parents with $c$, so $a \sim c$. Thus, the relation is transitive. 

    Since this sibling relation is reflexive, symmetric, and transitive, it is an equivalence relation. 

    \item Feel free to ignore this since we didn't have to prove it, but here are some attempts at proving the statement in problem 5.

    \textbf{Proof 1: Basel Problem}

    Recall that $\sum_{n=1}^\infty \frac{1}{n^2} = \frac{\pi^2}{6} < 1.65$

    For $n \geq 3$, $2 - \frac{1}{n} > 2 - \frac{1}{3} > 1.66 > \sum_{k=1}^\infty \frac{1}{k^2} > \sum_{k=1}^n \frac{1}{k^2}$.

    Thus, for $n \geq 3$, $2 - \frac{1}{n} > \sum_{k=1}^n \frac{1}{k^2}$

    We verified the base case of $n=2$ earlier, so we have that for $n \geq 2$, $2 - \frac{1}{n} > \sum_{k=1}^n \frac{1}{k^2}$
    

    \textbf{Proof 2: Induction}

    We already verified the base case, so now we verify the induction hypothesis. 

    Suppose $1 + \frac{1}{4} + \dots + \frac{1}{n^2} < 2 - \frac{1}{n}$.

    Consider that $\left(2 - \frac{1}{n+1}\right) - \left(2 - \frac{1}{n} \right) = \frac{1}{n(n + 1)} > \frac{1}{(n+1)^2}$.

    We then create a chain of equalities:
    \begin{align*}
        2 - \frac{1}{n+1} &= \frac{1}{n(n+1)} + \left(2 - \frac{1}{n}\right) \\
        &> \frac{1}{(n+1)^2} + \left(2 - \frac{1}{n}\right) \\
        &> \frac{1}{(n+1)^2} + \sum_{k=1}^n \frac{1}{k^2} \\
        &= \sum_{k=1}^{n+1}\frac{1}{k^2}
    \end{align*}

    This is the claim we set out to prove, so the induction hypothesis is complete. Thus, by induction, $1 + \frac{1}{4} + \dots + \frac{1}{n^2} < 2 - \frac{1}{n}$ for $n \geq 2$.

    

    
    
    
\end{enumerate}

\end{document}
