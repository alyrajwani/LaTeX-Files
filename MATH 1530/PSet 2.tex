\documentclass{article}
\usepackage{graphicx} % Required for inserting images
\usepackage{amsfonts}
\usepackage{amsmath}
\usepackage{amssymb}
\usepackage{amsthm}

\title{Math 1530 PSet 2}
\author{Aly Rajwani}
\date{September 2024}

\begin{document}

\maketitle

\begin{enumerate}
    \item There are $n$ rotation elements $\{R_{k\cdot \frac{360^o}{n}}\}_{k=0}^{n-1}$ and $n$ flip elements. If $n$ is odd, each flip is across a line of symmetry from a vertex to its opposing edge, and if $n$ is even, $\frac{n}{2}$ flips are across a line of symmetry from the middle of an edge to its opposing edge, and $\frac{n}{2}$ flips are across a line of symmetry from a vertex to its opposing vertex. Each of these are $\frac{n}{2}$ because they involve using 2 edges/vertices, and there are $n$ edges/vertices in total. Thus, there are $2n$ elements in $D_n$ since there are $n$ rotations and $n$ flips. 

    \item 
        \begin{enumerate}
            \item First we prove the forwards direction: 

            If $a, b \in G$ commute, then $ab = ba$. 
            
            So, $(ab)^2 = (ab)(ab) = a(ba)b = a(ab)b = a^2b^2$.

            Next, we prove the backwards direction:

            If $(ab)^2 = a^2b^2$, then we have $(ab)(ab) = (aa)(bb)$. We can cancel the $a$ on the left and the $b$ on the right to get the equality $ba = ab$, and so $a$ and $b$ commute with each other. 

            \item We will prove this by induction. 

            \textbf{Base Case: } $n = 1$

            
            $(a_1)^{-1} = a_1^{-1}$, and so the base case is verified.

            \textbf{Induction Hypothesis:}

            $\forall n \geq 1$, if $(a_1a_2\dots a_n)^{-1} = a_n^{-1}a_{n-1}^{-1}\dots a_1^{-1}$, then we want to show $(a_1a_2\dots a_na_{n+1})^{-1} = a_{n+1}^{-1}a_n^{-1}\dots a_1^{-1}$

            \begin{align*}
                (a_1a_2\dots a_na_{n+1})^{-1} &= ((a_1a_2\dots a_n)a_{n+1})^{-1}  & \text{(associativity)} \\
                &= a_{n+1}^{-1}(a_1a_2\dots a_{n+1})^{-1} & \text{($(ab)^{-1} = b^{-1}a^{-1}$)}\\
                &= a_{n+1}^{-1}a_n^{-1}\dots a_2^{-1}a_1^{-1} & \text{(induction hypothesis)}
            \end{align*}

            We have arrived at the desired conclusion. By the base case of $n=1$ and the induction hypothesis, we have proved that 
            
            $\forall n \geq 1, (a_1a_2\dots a_n)^{-1} = a_n^{-1}a_{n-1}^{-1}\dots a_1^{-1}$
        \end{enumerate}

    \item 
    We will show results for $ik, jk, ji, kj, \text{ and }, ki$, and then since $-1$ commutes with every element, the rest of the table can be filled in based on these results. 

    \begin{align*}
        ik &= i(ij) & \text{(substitution)} \\
        &= (ii)j & \text{(associativity)} \\
        &= (-1)j \\
        &= -j
    \end{align*}

    \begin{align*}
        jk &= (-ik)k & \text{(substitution)} \\
        &= -i(kk) & \text{(associativity)} \\
        &= (-i)(-1) \\
        &= i
    \end{align*}

    \begin{align*}
        ji &= j(jk) & \text{(substitution)} \\
        &= (jj)k & \text{(associativity)} \\
        &= (-1)k \\
        &= -k
    \end{align*}

    \begin{align*}
        kj &= (ij)j & \text{(substitution)} \\
        &= i(jj) & \text{(associativity)} \\
        &= i(-1) \\
        &= -i
    \end{align*}

    \begin{align*}
        ki &= k(-kj) & \text{(substitution)} \\
        &= k(-1)kj \\
        &= (-1)kkj & \text{(associativity)} \\
        &= (-1)(-1)j \\
        &= j
    \end{align*}

    With these results, we are able to use the fact that $-1$ commutes with every other element to fill in the rest of the table. Here is one example:

    \begin{align*}
        j(-k) &= j(-1)k \\
        &= (-1)jk \\
        &= (-1)i \\
        & =-i
    \end{align*}

    The rest of the calculations follow a similar pattern of isolating the $-1$, using the fact that it commutes, and then using the results previously calculated.

    \begin{center}
        \begin{tabular}{c||c|c|c|c|c|c|c|c}
             & 1 & $-1$ & $i$ & $-i$ & $j$ & $-j$ & $k$ & $-k$\\ \hline\hline
             1 & 1 & $-1$ & $i$ & $-i$ & $j$ & $-j$ & $k$ & $-k$\\ \hline
             $-1$ & $-1$ & 1 & $-i$ & $i$ & $-j$ & $j$ & $-k$ & $k$ \\ \hline
             $i$ & $i$ & $-i$ & $-1$ & 1 & $k$ & $-k$ & $-j$ & $j$ \\ \hline
             $-i$ & $-i$ & $i$ & $1$ & $-1$ & $-k$ & $k$ & $j$ & $-j $\\ \hline
             $j$ & $j$ & $-j$ & $-k$ & $k$ & $-1$ & $1$ & $i$ & $-i$\\ \hline
             $-j$ & $-j$ & $j$ & $k$ & $-k$ & $1$ & $-1$ & $-i$ & $i $\\ \hline
             $k$ & $k$ & $-k$ & $j$ & $-j$ & $-i$ & $i$ & $-1$ & $1$\\ \hline
             $-k$ & $-k$ & $k$ & $-j$ & $j$ & $i$ & $-i$ & $1$ & $-1$\\ \hline
        \end{tabular}
    \end{center}
        
    \item 
        \begin{enumerate}
            \item Suppose the columns of the Cayley table appear in the order $a_1, a_2, \dots, a_n$. Then the row corresponding to $e$ is represented by $ea_1, ea_2, \dots, ea_n$. By definition of the identity, $ea_k = a_k$ for all $k$, and so the row corresponding to $e$ is $a_1, a_2, \dots, a_n$, which is the original order. 

            Similarly, suppose the rows of the Cayley table appear in the order $a_1, a_2, \dots, a_n$. Then the column corresponding to $e$ is represented by $a_1e, a_2e, \dots, a_ne$. By definition of the identity, $a_ke = a_k$ for all $k$, and so the column corresponding to $e$ is $a_1, a_2, \dots, a_n$, which is the original order.

            \item Suppose an element appears twice in a column, which would imply that $ac = bc$, where $c$ is the column and $a$ and $b$ are rows. By right cancellation, we have that $a = b$, and so the rows must be the same. Thus, an element cannot appear twice in a column. 

            Suppose an element appears twice in a row, which would imply that $ab = ac$, where $a$ is the row and $b$ and $c$ are columns. By left cancellation, we have that $b = c$, and so the columns must be the same. Thus, an element cannot appear twice in a row.

            \item If we write the Cayley table rows in the same order as the columns, we will be able to tell if it is Abelian by checking that it is symmetric across the diagonal from top left to bottom right. This is because, given some index in the table $(i, j)$ where $i$ is the row and $j$ is the column, the element in that spot is $a_ia_j$. The symmetric element to that is at index $(j, i)$, and the element in that spot is $a_ja_i$. If these elements are the same, we would have $\forall i \forall j, a_ia_j = a_ja_i$, which would make the group Abelian. If we did not have this symmetric property, it would be the case that $\exists i\exists j$ s.t. $a_ia_j \neq a_ja_i$, and so the group would not be Abelian.
        \end{enumerate}
    \item 

        \begin{enumerate}
            \item Since $\text{det}A = ad-bc$, any matrix in this subset has determinant 2. However, $\text{det}XY = \text{det}X \cdot \text{det}Y$, and so multiplying any 2 matrices in this subset creates a matrix with determinant 4, and so the subset is not closed, and cannot be a subgroup.  

            \item We will prove this is a subgroup using the one-step subgroup test. First, this subset is nonempty, since 
            $\begin{bmatrix}
                1 & 0 \\ 0 & 1 
            \end{bmatrix}$ is a member of the subset.

            Then, suppose $A$ and $B$ are in the subset, and so $\text{det}A, \text{det}B \in \{1, -1\}$. We use another property of determinants, namely that $\text{det}X^{-1} = \frac{1}{\text{det}X}$, and so $\text{det}B^{-1} \in \{1, -1\}$. Thus, $\text{det}AB^{-1} = \text{det}A\text{det}B^{-1} \in \{1, -1\}$, and so this subset is a subgroup. 

            \item We will prove this is a subgroup using the one-step subgroup test. First, this subset is nonempty since $\begin{bmatrix}
                1 & 1 \\ 0 & 1
            \end{bmatrix}$ is a member of the subset. Next, consider two matrices in this subgroup $X = \begin{bmatrix}
                x_1 & x_2 \\ 0 & x_3
            \end{bmatrix}$ and $Y = \begin{bmatrix}
                y_1 & y_2 \\ 0 & y_3
            \end{bmatrix}$. $Y^{-1} = \frac{1}{y_1y_3}\begin{bmatrix}
                y_3 & -y_2 \\ 0 & y_1
            \end{bmatrix}$, and $XY^{-1} = \frac{1}{y_1y_3} \begin{bmatrix}
                x_1y_3 & x_2y_1 - x_1y_2 \\ 0 & x_3y_1
            \end{bmatrix}$, which is an element of the subset. Since this subset is nonempty and satisfies the one-step subgroup test, is is a subgroup. 

            \item Let $A = \begin{bmatrix}
                1 & 1 \\ 1 & 0
            \end{bmatrix}$. Then $A$ is a member of this subset. However, $A^2 = \begin{bmatrix}
                2 & 1 \\ 1 & 1
            \end{bmatrix}$, which is not a member of this subset. Since it is not closed, it is not a subgroup. 

            \item Let $A = \begin{bmatrix}
                1 & 1 \\ 0 & 1
            \end{bmatrix}$. Then $A$ is a member of this subset. For some matrix $X$ in this subset of the form $\begin{bmatrix}
                1 & x \\ 0 & 1
            \end{bmatrix}$, $X^{-1} = \begin{bmatrix}
                1 & -x \\ 0 & 1
            \end{bmatrix}$. Suppose $A$ and $B$ are in the subset. Then $AB^{-1} = \begin{bmatrix}
                 1 & a - b \\ 0 & 1
            \end{bmatrix}$ which is an element of the subset. Thus, by the one-step subgroup test, this subset is a subgroup. 
        \end{enumerate}
    \item If $\forall x \in G, x^2 = e$, then $\forall x \in G, x = x^{-1}$. Consider $a, b \in G$. We saw previously that $(ab)^{-1} = b^{-1}a^{-1}$, but since $ab \in G$, we also have $(ab)^{-1} = ab = a^{-1}b^{-1}$. So, $ba = b^{-1}a^{-1} = (ab)^{-1} = a^{-1}b^{-1} = ab$. Since $\forall a, b \in G, ab=ba, G$ is Abelian.
\end{enumerate}

\end{document}
