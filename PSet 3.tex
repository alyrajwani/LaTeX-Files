\documentclass{article}
\usepackage{graphicx} % Required for inserting images
\usepackage{amsfonts}
\usepackage{amsmath}
\usepackage{amssymb}
\usepackage{amsthm}
\usepackage{tikz}
\newcommand{\T}{\mathcal{T}}
\newcommand{\U}{\mathcal{U}}
\newcommand{\R}{\mathbb{R}}
\newcommand{\Z}{\mathbb{Z}}
\newcommand{\B}{\mathcal{B}}
\newcommand{\C}{\mathcal{C}}
\newcommand{\V}{\mathcal{V}}
\newcommand{\F}{\mathcal{F}}
\newcommand{\sk}{\smallskip}

\title{PSet 3}
\author{Aly Rajwani}
\date{September 2024}

\begin{document}

\maketitle

\begin{enumerate}
    \item \textbf{Exercise 13.7}
    \begin{proof}[Solution.]
        We note that if $\T \subset \T'$ and $\T' \subset \T''$, then $\T \subset \T''$. So, we will create a chain of inclusions to help us come to a final solution. 

        \sk
        
        \underline{$\T_4$ contains $\T_2$} 

        Let $\U \in \T_2$, so $\U = (a, b)$ or $\U = (a, b) \backslash \{\frac{1}{n} : n \in \Z^+\}$. If $\U = (a, b)$, then $\forall x \in \U, x \in (a, x] \subset (a, b)$. If $\U = (a, b) \backslash \{\frac{1}{n} : n \in \Z^+\}$, then $\forall x \in \U, x \in (a, 0]$ or $x \in (1, c]$ where $x < c < b$, or $x \in \left(\frac{1}{n+1}, d\right]$ where $x < d < \frac{1}{n}$. In any case, we can find a set $\U' \in \T_4$ such that $x \in \U' \subset \U$. Thus, $\T_2 \subset \T_4$. 

        \sk

        However, for $\U \in \T_4$, $\U = (a, b]$, and for $b$, there is no basis element $\U'$ in $\T_2$ satisfying $b \in \U' \subset \U$, so $\T_4 \not\subset \T_2$. 

        \sk

        \underline{$\T_2$ contains $\T_1$}

        \sk 

        Let $\U \in \T_1$, so $\U = (a, b)$. Then $\U \in \T_2$. 

        \sk

        However, let $\U \in \T_2$ be equal to $(-1, 1) \backslash \{\frac{1}{n} : n \in \Z^+\}$. Then for $0$, no open interval $\U'$ satisfies $0 \in \U' \subset \U$, so $\T_2 \not\subset \T_1$. 
      
        \underline{$\T_1$ contains $\T_3$ and $\T_5$.} 

        \sk
        
        Let $\U \in \T_3$, so $\U^c$ is finite or $\U = \emptyset$. If $\U = \emptyset$, then $\U \in \T_1$ since $\T_1$ is a topology. Otherwise $\U^c$ is finite, meaning we can represent $\U^c$ as $\bigcup_{n=1}^N x_n$. Then $\U = \bigcup_{n=0}^N (x_n, x_{n+1})$ where $x_0 = -\infty$ and $x_{N+1} = \infty$. Since $\U$ is a union of open intervals, $\U \in \T_1$. Thus, $\T_3 \subset \T_1$. 

        \sk
        
        However, if $\U \in \T_1$, then $\U = (a, b)$, and so $\U^c$ is not finite or all of $\R$, so $\U \notin \T_3$. Thus, $\T_1 \not\subset \T_3$

        \sk

        Let $\U \in \T_5$, so $\U = (-\infty, a)$. Then clearly $\U \in \T_1$, so $\T_5 \subset \T_1$ 

        \sk

        However, if $\U = (0, 1)$, then $\U \notin \T_5$, so $\T_1 \not\subset \T_5$

        \sk

        \underline{$\T_3$ and $\T_5$ are not comparable.}

        \sk

        Let $\U \in \T_5,$ so $\U = (-\infty, a)$. Then $\U^c$ is not finite or all of $\R$, so $\U \notin \T_3$. 
        
        \sk
        
        Let $\U \in \T_3$, so $\U^c$ is finite or $\U = \emptyset$. If $\U = \emptyset, \U \neq (-\infty, a)$ and if $\U^c$ is finite, then $\U \neq (-\infty, a)$.

        We can now create a chain of inclusions:

        \begin{center}
            \begin{tikzpicture}
                \node (t4) at (0, 0) {$\T_4$};
                \node (t2) at (0, -0.75) {$\T_2$};
                \node (t1) at (0, -1.5) {$\T_1$};
                \node (t3) at (-0.75, -2.25) {$\T_3$};
                \node (t5) at (0.75, -2.25) {$\T_5$};
                \draw [black, thick] (t4) -- (t2);
                \draw [black, thick] (t2) -- (t1);
                \draw [black, thick] (t1) -- (t3);
                \draw [black, thick] (t1) -- (t5);
            \end{tikzpicture}
        \end{center}

        where a topology being above another means it contains it. 
    \end{proof}

    \item 
    \textbf{Exercise 13.8}
    \begin{proof}[Solution.]
        \begin{enumerate}
            \item Let $\U$ be open in the standard topology on $\R$ and let $x \in \U$. Then, $x \in (a, b)$ for some $(a', b') \in \U$. Since $\mathbb{Q}$ is dense in $\R$, we can find an $a$ s.t. $a' < a < x$ and a $b$ s.t. $x < b < b'$. Then $x \in (a, b) \subset (a', b') \subset \U$. Lemma 13.2 allows us to conclude from this that $\B$ is a basis that generates the standard topology on $\R$. 

            \item 
            We are asked to show two things. First, that $\C$ is a basis for $\R$, and second, that the topology generated by $\C$ is different than the lower limit topology. 

            \sk

            First, we show that $\C$ meets the two conditions for being a basis. $\forall x \in \R, x \in \left[\lfloor x \rfloor, \lceil x \rceil\right)$, and this is an element of $\C$. As well, if $x \in [a_1, b_1) \cap [a_2, b_2)$, then let $a = \max(a_1, a_2)$ and $b = \min(b_1, b_2)$. Then $x \in [a, b) \subset [a_1, b_1) \cap [a_2, b_2)$, and $[a, b) \in \C$. These two statements show that $\C$ is a basis. 

            \sk

            Second, we show that the topology generated by $\C$ is not the lower limit topology. A topology $\T$ generated by a basis $\B$ satisfies $\forall \U \in \T$ and $\forall x \in \U, \exists B \in \B$ s.t. $x \in \B \subset \U$. We will provide a set $\U$ and element $x$ for which no element $C$ of $\C$ satisfies $x \in C \in \U$. Take $\U = [x, x + 1)$ where $x$ is irrational. Then since $\C$ is defined such that the lower bound is rational, there is no element $C \in \C$ which contains $x$ and is a subset of $\U$. Thus, the topology generated by $\C$ is different than the lower limit topology. 
        \end{enumerate}
    \end{proof}

    \item 
    \textbf{Exercise 16.4}
    \begin{proof}[Solution.]
        Let $\U \times \mathcal{V}$ be open in $X \times Y$ and take $x \times y \in \U\times\mathcal{V}$ Then, there is a basis element $A \times B$ of $X \times Y$ satisfying $x \times y \in A \times B \subset X \times Y$. Such an $A$ is open in $X$, and $x \in A \subset \pi_1\left(A\times B\right) \subset \pi_1(\U \times \mathcal{V})$, which means that $\pi_1(\U\times\mathcal{V})$ is open in $X$, and so $\pi_1$ is an open map. 

        \sk

        Similarly, if $A \times B$ is a basis element satisfying $A \times B \subset \U \times \mathcal{V}$, then $B$ is a basis element open in $Y$, and so $y \in B \subset \pi_2(A \times B) \subset \pi_2(\U \times \mathcal{V})$, which means that $\pi_2(\U \times \mathcal{V})$ is open in $Y$, and so $\pi_2$ is an open map. 
    \end{proof}

    \item 
    \textbf{Exercise 16.8}
    \begin{proof}[Solution.]
        A basis for the topology on $\R_\ell \times \R$ is $\B = \{[a, b) \times (c, d) : a, b, c, d \in \R\}$. The line $L$ is either vertical or has finite slope, and we consider these cases individually. 

        \sk 

        If $L = \{(x, y) : x = x_0\}$, then $L \cap ([a, b) \times (c, d))$ is either $\emptyset$ or $\{x_0\} \times (c, d)$. To prove that this topology is equivalent to the standard topology on $\R$, we will describe a homeomorphism. Let $f: L \cap (\R_\ell \times \R) \rightarrow \R$ be defined by $\{x_0\} \times (c, d) \mapsto (c, d)$. If $(c, d)$ is open in the standard topology, then $\{x_0\} \times (c, d)$ is open in $L \cap (\R_\ell \times \R)$, and vice versa. Thus, we have a bijective, continuous function from $L \cap (\R_\ell \times \R)$ to the standard topology on $\R$, and to the topology a vertical line inherits from $\R_\ell \times \R$ is the standard topology. 

        \sk

        Next, suppose $L$ has finite slope, so $L = \{(x, mx + e)\}$. $L$ either intersects $[a, b) \times (c, d)$ at the closed left edge or it does not. If it does, the intersection of $L$ and $[a, b), \times (c, d)$ is $\{(x, mx+e) : x \in [a, b) \}$, and if it does not, the intersection of $L$ and $[a, b), \times (c, d)$ is $\{(x, mx+e) : x \in (a, b) \}$. 

        \sk

        Define the function $f: L \cap (\R_\ell \times \R) \rightarrow \R_\ell$ as follows:
        \begin{center}$f = 
            \begin{cases}
                (x, mx+e) : x \in [a, b) \mapsto [a, b) \\
                (x, mx+e) : x \in (a, b) \mapsto (a, b)
            \end{cases}$
        \end{center}

        Each $[a, b)$ is an open set in $\R_\ell$, and each $(a, b)$ is open in the standard topology on $\R$, which is contained in $\R_\ell$. 

        \sk

        Thus, this is a bijective function which maps open sets to open sets, and so it is a homeomorphism, and the topology $L$ inherits from $\R_\ell \times \R$ if $L$ has finite slope is $\R_\ell$.

        \sk

        In the case of $\R_\ell \times \R_\ell$, when $L$ is vertical, if it intersects with $[a, b) \times [c, d)$, which is a basis element of $\R_\ell \times \R_\ell$, then it must intersect at the bottom edge, so we have $L \cap (\R_\ell \times \R_\ell) = \{x_0\} \times [c, d)$. Define $f: L \cap (\R_\ell \times \R_\ell) \rightarrow \R_\ell$ as $\{x_0\} \times [c, d) \mapsto [c, d)$. So, $f$ is bijective and maps open sets in $L \cap (\R_\ell \times \R_\ell)$ to open sets in $\R_\ell$, so the topology $L$ inherits is $\R_\ell$.

        \sk

        If $L$ has finite slope, we consider whether the slope is positive or negative. In the case that it is positive, then the intersection of $L \cap (\R_\ell \times \R_\ell)$ will be $(x, mx+e) : x \in [a, b)$, and we have shown in this case that $L$ inherits $\R_\ell$. 

        \sk

        If $L$ has negative slope and intersects $[a, b) \times [c, d)$, then $L$ inherits the discrete topology. Consider $(x, y) \in L$. $L \cap ([x, x + 1) \times [y, y + 1)) = \{(x, y)\}$, so every point on $L$ is open, which is the definition of the discrete topology. 
        
    \end{proof}

    \item
    \textbf{Exercise 17.6}
    \begin{proof}[Solution.]
        \begin{enumerate}
            \item We are given that $A \subset B$, and we want to show that $\overline{A} \subset \overline{B}$. Choose some $x \in \overline{A}$. Then, by definition, for all $\U$ containing $x$, $\U \cap A \neq \emptyset$. Since $A \subset B$, $(\U \cap A) \subset (\U \cap B)$, and so we have that all open sets $\U$ containing $x$ have a non-empty intersection with $B$, and thus $x \in \overline{B}$. So, $x \in \overline{A} \implies x \in \overline{B}$, meaning $\overline{A} \subset \overline{B}$. 

            \item We will prove this by showing that each of $\overline{A \cup B}$ and $\overline{A} \cup \overline{B}$ are subsets of each other. 

            \sk 

            Let $x \in \overline{A \cup B}$. Then for all $\U$ open in $X$ that contain $x$, $\U \cap (A \cup B) \neq \emptyset$. This implies that $\U \cap A \neq \emptyset$ or $\U \cap B \neq \emptyset$, and so $x$ is in at least one of $\overline{A}$ or $\overline{B}$. Thus, $\overline{A \cup B} \subset \overline{A} \cup \overline{B}$.

            \sk

            Let $x \in \overline{A} \cup \overline{B}$. WLOG assume $x \in \overline{A}$. Then for all $\U$ open in $X$ that contain $x$, $\U \cap A \neq \emptyset$. This implies that $(\U \cap A) \cup (\U \cap B) = \U \cap (A \cup B) \neq \emptyset$, so $x \in \overline{A \cup B}$. Thus, $\overline{A} \cup \overline{B} \subset \overline{A \cup B}$.

            \sk

            Since they are subsets of each other, $\overline{A \cup B} = \overline{A} \cup \overline{B}$.

            \item Let $x \in \bigcup \overline{A_\alpha}$. Then, there is at least one $\beta$ such that $x \in \overline{A_\beta}$. By definition, for all $\U$ open in $X$ containing $x$, $\U \cap A_\beta \neq \emptyset$. This implies that $\bigcup \left(\U \cap A_\alpha\right) = \U \cap \left(\bigcup A_\alpha\right)\neq \emptyset$. By definition, this means that $x \in \overline{\bigcup A_\alpha}$. Thus, $\bigcup \overline{A_\alpha} \subset \overline{\bigcup A_\alpha}$.

            \sk

            For an example where equality fails, take $X = \R$, and let $\forall n \in \mathbb{N}, A_n = \left\{\frac{1}{n}\right\}$. Each $A_n$ is a subset of $\R$, and each $A_n$ is closed, since it is the complement of $\left(-\infty, \frac{1}{n}\right) \cup \left(\frac{1}{n}, \infty\right)$, which are both open in the standard topology on $\R$. Thus, each $A_n = \overline{A_n}$, and so $\bigcup \overline{A_n} = \bigcup A_n = \left\{\frac{1}{n} : n \in \mathbb{N}\right\}$. However, $0 \in \overline{\bigcup A_n}$. For any open set $\U$ containing 0, the Archimedean property states that we can find a $\frac{1}{n} \in \U$, and so $\U \cap \bigcup A_n \neq \emptyset$. Thus, $\overline{\bigcup A_\alpha} = \left\{\frac{1}{n} : n \in \mathbb{N}\right\} \cup \{0\}$, and so equality does not hold. 
        \end{enumerate}
    \end{proof}
    
    \item 
    \textbf{Exercise 17.8}
    \begin{proof}[Solution.]
        \begin{enumerate}
            \item Equality does not hold, as demonstrated by the following example:

            Let $X = \R, A = (0, 1),$ and $B = (1, 2)$. Then $\overline{A} = [0, 1]$ and $\overline{B} = [1, 2]$, and so $\overline{A} \cap \overline{B} = \{1\}$. However, $A \cap B = \emptyset$, and so $\overline{A \cap B} = \emptyset$. This suggests that $\overline{A \cap B} \subset \overline{A} \cap \overline{B}$, which we prove below: 

            \sk 

            Let $x \in \overline{A \cap B}$. Then for all open sets $\U$ in $X$ containing $x$, $\U \cap (A \cap B) \neq \emptyset$. This implies that $(\U \cap A) \cap (\U \cap B) \neq \emptyset$, and so $\U \cap A \neq \emptyset$ and $\U \cap B \neq \emptyset$. This means $x \in \overline{A}$ and $x \in \overline{B}$, so $x \in \overline{A} \cap \overline{B}$. Thus, $\overline{A \cap B} \subset \overline{A} \cap \overline{B}$.

            \item Equality does not hold, as demonstrated by the same example as in part (a).

            \sk
            
            
            We perform the following sequence of operations  to show that $\overline{\bigcap A_\alpha} \subset \bigcap \overline{A_\alpha}$.

            \begin{align*}
                x \in \overline{\bigcap A_\alpha} &\Rightarrow \text{ if }x \in \U, \text{ then } \U \cap \left(\bigcap A_\alpha\right) \neq \emptyset\\
                &\Rightarrow \exists y \text{ s.t. }\forall \alpha,  \U \cap A_\alpha \neq \emptyset \\
                &\Rightarrow \forall \alpha, x \in \overline{A_\alpha} \\
                &\Rightarrow x \in \bigcap \overline{A_\alpha}
            \end{align*}

            The first implication is the definition of $x$ being in the closure of $\bigcap A_\alpha$, the second is a property of intersections, the third is the definition of $x$ being in the closure of $A_\alpha$, and the fourth is a property of intersections. 

            \sk

            Since $x \in \overline{\bigcap A_\alpha} \Leftrightarrow x \in \bigcap \overline{A_\alpha}$, $\overline{\bigcap A_\alpha} = \bigcap \overline{A_\alpha}$. 

            \item Equality does not hold, as demonstrated by the following example:

            Let $X = \R, A = (0, 2)$, and $B = (0, 1)$. Then $\overline{A} = [0, 2]$ and $\overline{B} = [0, 1]$. So, $\overline{A} - \overline{B} = (1, 2]$. However, $A - B = [1, 2)$, and so $\overline{A - B} = [1, 2] \neq (1, 2]$. This suggests that $\overline{A} - \overline{B} \subset \overline{A - B}$, which we prove below:

            \sk
            Let $x \in \overline{A} - \overline{B}$. Then $x \in \overline{A}$ and $x \notin \overline{B}$. By definition, we have that for all open sets $\U$ of $X$ containing $x$, $\U \cap A \neq \emptyset$, and there exists an open set $V$ of $X$ containing $x$ such that $\mathcal{V} \cap B = \emptyset$. Let $\mathcal{W} = \U \cap \mathcal{V}$. Since both $\U$ and $\mathcal{V}$ are open, $\mathcal{W}$ is open, and since $x \in \U$ and $x \in \mathcal{V}$, $x \in \mathcal{W}$. So, $\mathcal{W}$ is an open set containing $x$, so it satisfies $\mathcal{W} \cap A \neq \emptyset$. Take $y$ to be an element in this intersection. Since $y \in \mathcal{W}, y \in \mathcal{V}$. This implies that $y \notin B$, since $y \in B$ would imply that $y \in V \cap B$, which we know to be empty. So, $y \in A$ and $y \notin B$ imply that $y \in A - B$. Since $y \in \mathcal{W}, y \in \U$, and so $U \cap (A - B) \neq \emptyset$. Since $\U$ is any arbitrary open set containing $x$, we have that all open sets of $X$ containing $x$ have a non-empty intersection with $A - B$, which by definition means that $x \in \overline{A - B}$. Thus, $\overline{A} - \overline{B} \subset \overline{A - B}$.
        \end{enumerate}
    \end{proof}
    
    \item
    \textbf{Exercise 17.9}
    \begin{proof}[Solution.]
        We are given that $A \subset X$ and $B \subset Y$ and we want to show that in the space $X \times Y$, $\overline{A \times B} = \overline{A} \times \overline{B}$. Let $(x, y) \in \overline{A \times B}$. Then every basis element of $X \times Y$ containing $(x, y)$, namely sets of the form $\U \times \V$ where $\U$ is open in $X$ and $\V$ is open in $Y$, intersects $A \times B$. So, we have the following series of equivalent statements:

        \begin{align*}
            (x, y) \in \overline{A \times B} &\Leftrightarrow \forall \U \times \V \in X \times Y \text{ s.t. } (x, y) \in \U \times \V, (\U \times \V) \cap (A \times B) \neq \emptyset \\
            &\Leftrightarrow (\U \cap A) \times (\V \cap B) \neq \emptyset \\
            &\Leftrightarrow \U \times A \neq \emptyset \text{ and } \V \times B \neq \emptyset \\
            &\Leftrightarrow x \in \overline{A} \text{ and } y \in \overline{B} \\
            &\Leftrightarrow (x, y) \in \overline{A} \times \overline{B}
        \end{align*}

        The first equivalence is the definition of the closure of $\overline{A \times B}$, the second is the property that $(A \times B) \cap (C \times D) = (A \cap C) \times (B \cap D)$, and the third is true because if the Cartesian product is non-empty, each set must be non-empty. The fourth equivalence is true, because we have that any open set $\U$ in $X$ containing $x$ has a non-empty intersection with $A$, and so $x \in \overline{A}$, and by similar logic $y \in \overline{B}$. The last is true by definition. 

        \sk 

        Since $(x, y) \in \overline{A \times B} \Leftrightarrow (x, y) \in \overline{A} \times \overline{B}$, we have that $\overline{A \times B} = \overline{A} \times \overline{B}$
    \end{proof}

    \item 
    \textbf{Exercise 17.14} 
    \begin{proof}[Solution.]
        \begin{enumerate}
            \item We will show that the sequence $\left(\frac{1}{n}\right)$ converges to $x$ for all $x \in \R$. 

            \sk

            A sequence $(x_n) \rightarrow x$ if for all open $\U$ in $X$ containing $x$, $\exists N \in \mathbb{N}$ s.t. $\forall n \geq N, x_n \in \U$. 

            \sk

            In the finite complement topology, this condition is always met if there are infinitely many distinct $x_n$. Consider $\U$ containing $x$ which is open in the finite complement topology, denoted $\T_f$. Then $\U^c$ is finite or $\U^c = \T_f$. Since $x \in U, x \notin \U^c$, but then $\U^c \neq \T_f$. So, $\U^c$ must be finite. This means that only finitely many terms $x_n$ exist in $\U^c$, and so there exists an $N$ such that $\forall n \geq N$, every $x_n \in \U$. This is the definition of convergence to $x$, but since $x$ is any point in $\T_f$, $(x_n)$ converges to every point of $\T_f$. 

            \sk

            Applying this general argument to the specific example, we have that there are infinitely many distinct terms in $\left(\frac{1}{n}\right)$, and so $\left(\frac{1}{n}\right)$ converges to every point of $\R$ in the finite complement topology on $\R$. 

            \item We will show that the sequence $\left(-\frac{1}{n}\right)$ does not converge in the lower limit topology.

            \sk

            Since each term in $(\left(-\frac{1}{n}\right))$ satisfies $-\frac{1}{n} < 0$, we consider two cases for potential limits of the sequence. 

            \sk

            \textbf{Case 1:} $x \geq 0$

            If $x \geq 0$, then take $[x, \infty)$ which is open in $\R_\ell$. No $-\frac{1}{n} \in [x, \infty)$, and so $\exists \U$ open in $\R_\ell$ s.t. $\forall N$, $\exists n \geq N$ s.t. $-\frac{1}{n} \notin \U$. 

            \sk 

            \textbf{Case 2:} $x < 0$

            Since $x < 0$, the Archimedean property states that we can find a $k \in \mathbb{N}$ s.t. $-\frac{1}{k} > x$. Take $x \in [-1, -\frac{1}{k})$. Then $\forall n \geq k, -\frac{1}{n} > \frac{1}{n} \implies x_n \notin [-1, -\frac{1}{k})$. So, $\exists \U$ open in $\R_\ell$ s.t. $\forall N$, $\exists n \geq N$ s.t. $-\frac{1}{n} \notin \U$. 

            \sk

            Since the union of these two cases covers all of $\R$, $\left(-\frac{1}{n}\right)$ does not converge in $\R_\ell$
        \end{enumerate}
    \end{proof}

    \item 
    \textbf{Exercise 17.16}
    \begin{proof}[Solution.]
        
    
        For each topology, we determine the closure of the set $K = \left\{\frac{1}{n} : n \in \mathbb{Z}\right\}$ and whether the topology satisfies the Hausdorff axiom and the $T_1$ axiom. Note that if $\T$ is coarser than $\T'$ and $\T$ satisfies the Hausdorff axiom, then $\T'$ also satisfies the Hausdorff axiom.  
    
        \underline{$\T_3$: Finite complement topology}
        \begin{enumerate}
            \item We proved earlier that in the finite complement topology, the sequence $\left(\frac{1}{n}\right)$ converges to every $x \in \R$, thus every $x \in \R$ is a limit point of $K$, and so $\overline{K} = \R$. 
    
            \item Choose two arbitrary elements $x$ and $y$ of $\R$ and suppose $x < y$. If $\T_3$ satisfies the Hausdorff axiom, we can find two open sets $\U$ and $\V$ satisfying $\U \cap \V = \emptyset$ and $x \in \U, y \in \V$. We will show that this is impossible. 
    
            \sk
            
            Suppose $x \in \U$ and $y \in \V$. Then $\U^c$ and $\V^c$ are finite. Thus, $\V \not\subset \U^c$, since $\V$ is infinite, and so there must exist a $v \in \V$ where $v \notin \U^c$. This implies that $v \in \V$ and $v \in \U$, so $\U \cap \V \neq \emptyset$. Thus, $\T_3$ does not satisfy the Hausdorff axiom.
    
            \sk 
    
            If $\T_3$ satisfies the $T_1$ axiom, then all finite point sets in $\T_3$ are closed. Consider some finite point set $\mathcal{F}$. $\mathcal{F}^c$ is open in $\T_3$ because $(\mathcal{F}^c)^c = \mathcal{F}$, which is finite, and so $\mathcal{F}$ is closed in $\T_3$. Thus, $\T_3$ satisfies the $T_1$ axiom. 
        \end{enumerate}
    
        \underline{$\T_5$: Topology with sets $(-\infty, a)$ as a basis}
        \begin{enumerate}
            \item Any open set in $\T_5$ is the arbitrary union of basis elements, which by construction have the form $(-\infty, a)$ for some $a \in \R$. Thus, a closed set in $\T_5$ is of the form $[a, \infty)$. The closure of $K$ is the intersection of all closed sets containing $K$. Any closed set containing $K$ is of the form $[a, \infty)$ where $a \leq 0$. The intersection of all these sets is $[0, \infty)$, and so $\overline{K} = [0, \infty)$. 
            
            \item Let $x = 0$ and $y = 1$. Then if $\U$ is an open set containing $x$, $\U = (-\infty, a)$ with $a > 0$. So, $\U$ contains $-1$. Similarly, an open set $\V$ containing $y$ also contains $-1$. Thus, $\U \cap \V \neq \emptyset$ and so $\T_5$ does not satisfy the Hausdorff axiom. 
    
            \sk 
    
            Let $\F$ be a finite point set with an order $\F_1, \F_2, \dots, \F_n$. Then $\F^c$ is of the form $(-\infty, \F_1) \cup \dots (\F_2, \F_3) \cup \dots \cup (\F_n, \infty)$. This set is not open in $\T_5$ since $\frac{\F_2 + \F_3}{2} \in \F^c$, but if this implies $\F^c = (-\infty, a)$ for some $a > \frac{\F_2 + \F_3}{2}$. Then, $\F^c$ would contain $\F_1$, since $\F_1 < \frac{\F_2 + \F_3}{2}$, but this is a contradiction. Thus, $\F^c$ is not open in $\T_5$, so $\F$ is not closed in $\T_5$. 
        \end{enumerate}
    
        \underline{$\T_1:$ Standard topology}
        \begin{enumerate}
            \item The only limit point of $K$ is $\{0\}$. Let $\U$ be an open set containing 0, so there is some $(a, b) \subset \U$ where $a < 0 < b$. Take $N > \frac{1}{b}$, and so $\forall n \geq N, \frac{1}{n} \in (a, b)$. No limit point can be greater than $0$, because we can take $\U = (a, b)$ with $0 < a < x < b$, so beyond $N = \frac{1}{a}$, no $\frac{1}{n}$ will be in $(a, b)$. Similarly, no limit point can be less than 0, because we can take $\U = (a, b)$ with $a < x < b < 0$ and so $\frac{1}{n}$ will not be an element of $\U$. Thus, $\overline{K} = K \cup \{0\}$
            \item Choose two arbitrary elements $x$ and $y$ of $\R$ and suppose $x < y$. Take $\epsilon = \frac{x + y}{2}$, and let $\U = (x - \epsilon, x + \epsilon)$ and let $\V = (y - \epsilon, y + \epsilon)$. Then $\U \cap \V = \emptyset$ and $x \in \U, y \in \V$. Thus, $\T_1$ satisfies the Hausdroff axiom. 
    
            \sk
    
            Since $\T_1$ satisfies the Hausdorff axiom, it also satisfies the $T_1$ axiom. 
        \end{enumerate}
    
        \underline{$\T_2$: Topology of $\R_K$}
        \begin{enumerate}
            \item $K^c = \R - K$, which is open in $\R_K$. Since $K^c$ is open, $K$ is closed, and so $\overline{K} = K$.
    
            \item Since $\T_1$ is coarser than $\T_2$ and $\T_1$ satisfies the Hausdorff axiom, $\T_2$ also satisfies the Hausdorff axiom. Since $\T_2$ satisfies the Hausdorff axiom, it satisfies the $T_1$ axiom.
        \end{enumerate}
    
        \underline{$\T_4$: Upper limit topology}
        \begin{enumerate}
            \item Since $\T_2$ is coarser than $\T_4$, $\R - K \in \T_2 \implies \R - K \in \T_4$. So, $K^c = \R - K$ is open in $\T_4$, so $K$ is closed, and $\overline{K} = K$. 
    
            \item Since $\T_2$ is coarser than $\T_4$ and $\T_2$ satisfies the Hausdorff axiom, $\T_4$ also satisfies the Hausdorff axiom. Since $\T_4$ satisfies the Hausdorff axiom, it satisfies the $T_1$ axiom. 
        \end{enumerate}
    \end{proof}

    \item 
    \textbf{Exercise 17.19}
    \begin{proof}[Solution.]
        \begin{enumerate}
            \item We have that $\text{Bd }A = \overline{A} \cap \overline{X - A}$. Suppose that $x \in \text{Int A}$, which means there is an open set $\U$ of $X$ containing $x$ such that $\U$ is contained in $A$. If $\U$ is contained in $A$, then $\U \cap (X - A) = \emptyset$. But, this means that there exists an open set containing $x$ which has an empty intersection with $X - A$, and thus $x$ is not in $\overline{X - A}$, and so $x \notin \text{Bd }A$. Thus, $\text{Int }A \cap \text{Bd }A = \emptyset$. 

            \sk 

            We will show that $\overline{A}$ and $\text{Int }A \cup \text{Bd }A$ are subsets of each other.

            \sk 

            Let $x \in \text{Int }A \cup \text{Bd }A$. If $x \in \text{Int }A$, then since $\text{Int }A \subset A \subset \overline{A}, x \in \overline{A}$. If $x \in \text{Bd }A$, then $x \in \overline{A} \cap \overline{X - A}$, and so $x \in \overline{A}$. In any case, $x \in \text{Int }A \cup \text{Bd }A$ implies $x \in \overline{A}$, so $\text{Int }A \cup \text{Bd }A \subset \overline{A}$. 

            \sk 

            Let $x \in \overline{A}$. If $x \in \text{Int }A$, then $x \in \text{Int }A \cup \text{Bd }A$. If $x \notin \text{Int }A$, then no open set $\U$ of $X$ containing $x$ is contained in $A$, and so all open sets $\U$ have a nonempty intersection with $X - A$. Thus, $\forall \U$, $x \in \U$ implies $\U \cap (X - A) \neq \emptyset$, and so $x \in \overline{X - A}$. Thus, $x \in \overline{A} \cap \overline{X - A}$ and so $x \in \text{Int }A \cup \text{Bd }A$. 

            \sk

            Since $\overline{A}$ and $\text{Int }A \cup \text{Bd }A$ are subsets of each other, $\overline{A} = \text{Int }A \cup \text{Bd }A$ are subsets of each other.

            \item First, we will show that if $A$ is both open and closed, then $\text{Bd }A = \emptyset$. 

            \sk 

            If $A$ is both open an closed, then $\text{Int }A = A = \overline{A}$. Since $\text{Bd }A = \overline{A} \cap \overline{X - A}$, we can substitude to get $\text{Bd }A = \text{Int }A \cap \overline{X - A}$. Now, suppose $x \in \text{Int }A$. Then there is some $\U$ open in $X$ containing $x$ that is contained in $A$. Since $\U$ is contained in $A$, then $\U \cap (X - A) = \emptyset$. But, this means that $x \notin \overline{X - A}$, since we have found an open set that contains $x$ and has an empty intersection with $X - A$. Since $x \in \text{Int }A$ implies $x \notin \overline{X - A}$, $\text{Bd }A = \text{Int }A \cap \overline{X - A} = \emptyset$.

            \sk 

            Next, suppose that $\text{Bd }A = \emptyset$. We have that $\overline{A} = \text{Bd }A \cup \text{Int }A$, and so $\overline{A} = \text{Int }A$. Since $A \subset \overline{A} \subset \text{Int }A$ and $\text{Int }A \subset A$, $A = \text{Int }A$, so $A$ is open. Similarly, since $\overline{A} \subset \text{Int }A \subset A$ and $A \subset \overline{A}$, $A = \overline{A}$, so $A$ is closed. 

            \sk

            Thus, $\text{Bd }A = \emptyset \Leftrightarrow A$ is both open and closed.

            \item We will first show that $\text{Int }\U$ is disjoint with $\text{Bd }\U$. 

            \sk

            Consider $x \in \text{Int }\U$. Then there is some open set $\V$ of $X$ containing $x$ that is completely contained in $\U$. So $\V \cap (X - \U) = \emptyset$. But, this means that $x \notin \overline{X - \U}$, since we have found an open set that contains $x$ and has an empty intersection with $X - \U$. Since $x \notin \overline{X - \U}, x \notin \text{Bd }\U$. Thus, $\text{Int }\U \cap \text{Bd }\U = \emptyset$. 

            \sk

            Since $\overline{\U} = \text{Int }\U \cup \text{Bd }\U$ and since $\text{Int }\U$ and $\text{Bd }\U$ are disjoint, $\text{Bd }\U = \overline{\U} - \text{Int }\U$. But, since $\U$ is open, $\U = \text{Int }\U$, so $\text{Bd }\U = \overline{\U} - \U$.

            \item Consider the finite complement topology on $\R$. The set $\U = \R - \{0\}$ is open since its complement is $\{0\}$, which is finite. Take some $x \in \R$ and let $\V$ be an open set containing $x$. Since $\V$ is open, $\R - \V$ is finite or is $\R$, but we know the second case is not true since $x \in \V$. So $\R - \V$ is finite, and since $\U$ is infinite, $\U \not\subset \R - \V$. Thus, $\U \cap \V \neq \emptyset$. Since $\V$ is an arbitrary open set containing $x$ and it has a non-empty intersection with $\U$, $x \in \overline{\U}$, so $\overline{\U} = \R$. The union of all sets contained in $\R$ is $\R$ itself, so $\text{Int }\R = \R$.

            \sk

            We thus have $\U = \R - \{0\}$, but $\text{Int }\overline{\U} = \R$, so $\U \neq \text{Int }\overline{\U}$
        \end{enumerate}
    \end{proof}
\end{enumerate}

\end{document}
