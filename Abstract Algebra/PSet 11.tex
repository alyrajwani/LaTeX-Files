\documentclass{hmwk}

\hdr{Problem Set 11}{MATH 1530: Abstract Algebra}{Aly Rajwani}
\hwk{11}

\begin{document}

\maketitle

\begin{problem}{Problem 1}
How many ideals does the ring $\Z/30\Z$ have? List all of the ideals, and give a justification for why the list is complete.
\end{problem}

\begin{solution}

\pre Note that $\Z/n\Z$ is a PID, so every ideal of the ring $\Z/30\Z$ is generated by a single element. Consider some ideal $(a)$. Then, by Bezout's Lemma, there exist $x, y \in \Z$ such that $30x + ay = \gcd(a, 30)$, so $\gcd(a, 30) \in (a)$. This implies that $(\gcd(a, 30)) \subset (a)$, but since $\gcd(a, 30) \mid a$, we also have $(a) \subset (\gcd(a, 30))$, meaning the sets are equal. Thus, the number of unique ideals is the number of unique divisors of 30. There are 8 divisors of 30, which are 1, 2, 3, 5, 6, 10, 15, and 30, and so the ideals of the ring $\Z/30\Z$ are $$(1), (2), (3), (5), (6), (10), (15), \text{ and }(30)$$.
\end{solution}

\begin{problem}{Problem 2}
Let $R$ be a commutative ring with 1 and let $a$ and $b$ be nonzero elements of $R$. A \textit{least common multiple} of $a$ and $b$ is an element $e$ of $R$ such that $a \mid e$ and $b \mid e$ and if $a\mid e'$ and $b \mid e'$ then $e \mid e'$
\begin{itemize}
    \item[(a)]  Prove that the least common multiple of $a$ and $b$ (if it exists) is a generator for the ideal $(a) \cap (b)$.
    \item[(b)] Deduce that any two nonzero elements in a Euclidean Domain have a least common multiple which is unique up to multiplication by a unit.
\end{itemize}
\end{problem}

\begin{solution}
\begin{itemize}
    \item[(a)] 
    \pre Let $e$ be the least common multiple of $a$ and $b$. We will show that $(a) \cap (b) = (e)$, which shows that $e$ is a generator for the ideal $(a) \cap (b)$.

    \pre Suppose $x \in (a) \cap (b)$, and so $x \in (a)$. This means that there is some $r_a \in R$ such that $r_aa = x$. Similarly, there is some $r_b \in R$ such that $r_bb = x$. Thus, $a \mid x$ and $b \mid x$, and so by definition of the least common multiple, $e \mid x$. This implies that $x \in (e)$. Since $x$ was arbitrary, every $x \in (a) \cap (b)$ satisfies $x \in (e)$, so $(a) \cap (b) \subset (e)$.
    
    \pre Suppose $x \in (e)$, so $r_ee = x$. Since $e$ is the least common multiple of $a$ and $b$, there is an $r_a$ and an $r_b$ such that $r_aa = e$ and $r_bb = e$. We can thus write $x$ as $r_er_aa$ and as $r_er_bb$. These respectively imply that $x \in (a)$ and $x \in (b)$, and so $x \in (a) \cap (b)$. Thus, $(e) \subset (a) \cap (b)$.
    
    \pre Since both of these sets are subsets of the other, they must be equal, and so $e$ is a generator for $(a) \cap (b)$ when $e$ is the least common multiple of $a$ and $b$.

    \item[(b)] We showed in class that in a Euclidean Domain, two non-zero elements $a$ and $b$ have a greatest common divisor, $\gcd(a, b)$. Let $e$ be the element such that $e\gcd(a, b) = ab$, which we know exists by definition of gcd. Notationally, we will write $e = \frac{ab}{\gcd(a, b)}$. Also, let $a'$ be the element such that $a'\gcd(a, b) = a$ and $b'$ be the element such that $b'\gcd(a, b) = b$.

    \pre First, clearly $a | e$, since $e = ab'$, which is well-defined element of the Euclidean Domain by definition of the gcd. Similarly, $b | e$. 

    \pre Now suppose $e'$ is some multiple of $a$ and $b$, so $a \mid e'$ and $b \mid e'$. Then $ab \mid be'$ and $ab \mid ae'$, and so we have 
    $$ab \mid \gcd(ae', be') = e'\gcd(a, b)$$

    \pre and so $\frac{ab}{\gcd(a, b)} \mid e'$. Thus, $e$ divides every other element that is a multiple of both $a$ and $b$, and so $\frac{ab}{\gcd(a, b)}$is a least common multiple of $a$ and $b$.

    \pre If $e$ and $e'$ are two least common multiples of $a$ and $b$, then, as proved in part a, $(e) = (a) \cap (b) = (e')$. We proved on the previous problem set that if $(e) = (e')$ in an integral domain, then $e$ differs from $e'$ by multiplication by a unit, and so least common multiples are unique up to multiplication by a unit.
\end{itemize}


\end{solution}

\begin{problem}{Problem 3}
Suppose $G$ acts on a set $X$, and $|G| = p^n$, where $p$ is a prime number and $n \geq 1$ is an integer. Prove that the number of fixed points of the action is congruent to the size of X modulo p.    

\noindent Hint: Recall that the orbits of X under the action of G partition X, and add up the
size of these orbits
\end{problem}

\begin{solution}

\pre Recall the orbit-stabilizer theorem, which states that $|\text{Orb}_x| = |G|/|\text{Stab}_x|$, and that a fixed point of an action is a point $x$ such that $g \cdot x = x$ for all $g \in G$ (that is, $\text{Orb}_x = \{x\}$).  

\pre Since the orbits of $X$ under the action of $G$ form a partition of $X$, the size of $X$ is equal to the sum of the sizes of the orbits. If every point is a fixed point, then clearly the number of fixed points is congruent to the size of $X$ modulo $p$. Otherwise, for every $y \in X$ that is not a fixed point, $|\text{Orb}_y| > 1$ by definition, and since $|\text{Orb}_y|$ divides $|G| = p^n$, $p$ being prime necessitates that $|\text{Orb}_y|$ is a multiple of $p$. So, the size of an orbit is either 1, since if $x$ is a fixed point then $\text{Orb}_x = \{x\}$, or it is a multiple of $p$. Thus, the sum of the sizes of the orbits is $|X| = k + jp$, where $k$ is the number of fixed points. Taking both sides modulo $p$, we get that the number of fixed points of the action is congruent to the size of $X$. 
\end{solution}

\begin{problem}{Problem 4}
Let $G$ be a group with $|G| = p^n$, $p$ is a prime, and $n \geq 1$ an integer. Show that $Z(G) \neq \{e\}$ by considering the action of $G$ on $X = G$ under conjugation, and using the previous problem.  
\end{problem}

\begin{solution}

\pre We just established that given all the conditions mentioned in the problem, $|X| = |G|$ is congruent to the number of fixed points modulo $p$. Since $|G| = p^n \equiv 0 \mod p$, the number of fixed points must also be $0 \mod p$. 

\pre Let $G$ act on itself by conjugation, so $\text{Orb}_x = \{gxg^{-1} : g \in G\}$. The fixed points of this action are the points such that $x = gxg^{-1}$ for all $g \in G$. Since this implies $xg = gx$ for all $g \in G$, the fixed points are the points which commute with every other element; that is, all points in the center of $G$. Since the cardinality of $Z(G)$ is thus the number of fixed points, it must be congruent to 0 modulo $p$, and so its cardinality cannot be 1. Therefore, $Z(G) \neq \{e\}$.

\end{solution}

\begin{problem}{Problem 5}
Let $G$ be a group and $Z(G)$ be the center of $G$. Prove that if $G/Z(G)$ is cyclic then $G$ is abelian.

\noindent Hint: Let $gZ(G)$ be a generator for $G/Z(G)$. Prove that the centralizer of $g$ is $G$ and
deduce that $g \in Z(G)$.
\end{problem}

\begin{solution}

\pre Suppose that $G/Z(G)$ is cyclic, so $G/Z(G) = \langle gZ(G) \rangle$ for some $g \in G$. Let $x \in G$, so $xZ(G) = g^nZ(G)$. This implies by properties of cosets that $x = g^nz$ for some $z \in Z(G)$. Then $$gx = gg^nz = g^{n+1}z = zg^{n+1} = zg^ng = g^nzg = xg$$

\pre where the third and fifth equalities are true because $z$ defined to be an element which commutes with everything in $G$. 

\pre Thus, $g$ commutes with every $x \in G$, and so $g \in Z(G)$. 

\pre Since $g \in Z(G)$, we have $gZ(G) = Z(G)$ by properties of cosets. Therefore, $|G/Z(G)| = |\langle Z(G) \rangle | = 1$, since $Z(G)$ is a subgroup and so is closed. Since the order of $G/Z(G)$ is 1, and since $Z(G)$ is a subgroup of $G$, it must be that $G = Z(G)$, and so $G$ is abelian.
\end{solution}

\begin{problem}{Problem 6}
Prove that every group of order $p^2$ is abelian, by using the previous problems.
\end{problem}

\begin{solution}

\pre Let $G$ be a group of order $p^2$. Since $|G/Z(G)| = \frac{|G|}{|Z(G)|}$, and $|G| = p^2$, it must be that $|Z(G)| = 1, p, \text{ or } p^2$. By Problem 4, $Z(G) \neq \{e\}$, so $|Z(G)| \neq 1$. If $|Z(G)| = p$, then $|G/Z(G)| = p$, and a group of prime order is cyclic. By Problem 5, this implies $G$ is abelian (note, however, that this implies $G = Z(G)$, but $|G|$ is supposed to be $p^2$). If $|Z(G)| = p^2$, then $|G/Z(G)| = 1$, and a group with one element is cyclic. By Problem 5, $G$ is Abelian. 
\end{solution}

\begin{problem}{Challenge Problem 1}
Prove that in a Euclidean Domain the least common multiple of $a$ and $b$ is $\frac{ab}{(a,b)}$, where $(a, b)$ is the greatest common divisor of $a$ and $b$.
\end{problem}

\begin{solution}

\pre See Problem 2.
\end{solution}

\begin{problem}{Challenge Problem 2}
Prove that the quotient $\Z[i]/I$ is finite for any nonzero ideal $I$. (Use an argument about norms.)
\end{problem}

\begin{solution}

\pre Note that $\Z[i]$ is a Euclidean domain equipped with norm $N: \Z[i] \rarr \Z^+$, $N(a + bi) = a^2 + b^2$.

\pre Let $I$ be some non-zero ideal, and choose a non-zero element $d \in I$ with minimal norm. Such an element must exist due to the Well-ordering Principle. $I = (d)$, since if there is an $x \in I$ such that $x$ is not a multiple of $d$, then by the division algorithm there is an $r \in \Z[i]$ such that $x = qd + r$ with $0 \neq N(r) < N(d)$, and since $r = x - qd$, $r \in I$, contradicting the minimality of $N(d)$. 

\pre Consider the coset $r + (d)$ of the quotient $\Z[i]/I$. We can find a representative of this coset such that $N(r) < N(d)$ by the division algorithm. Let $r = a + bi$. Then $a^2 + b^2 < N(d)$, and since $N(d)$ is finite and $a, b \in \Z$, there are only finitely many pairs $(a, b)$ which satisfy this inequality. Thus, there are only finitely many cosets, and so the quotient $\Z[i]/I$ is finite for any non-zero ideal $I$.
\end{solution}

\end{document}