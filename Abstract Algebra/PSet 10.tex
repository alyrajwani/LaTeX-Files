\documentclass{hmwk}

\hdr{Problem Set 10}{\textbf{MATH 1530: Abstract Algebra}}{Aly Rajwani}
\hwk{10}

\begin{document}

\maketitle

\begin{problem}{Problem 1}
Let $R$ be an integral domain. Prove that if two elements $d$ and $d'$ of $R$ generate the same principal ideal, that is, $(d) = (d')$ then $d = ud'$ for some unit $u$ in $R$. Deduce that if $d$ and $d'$ are both greatest common divisors for some elements $a$ and $b$ in $R$, then they differ by multiplication by a unit.
\end{problem}

\begin{solution}

\pre Suppose that $(d) = (d')$. If $d = 0$, then $(d) = \{0\} = (d')$, and so $d'$ must be $0$ as well. We can thus represent $d$ as $1d'$, and so $d$ differs from $d'$ by multiplication by a unit. Now assume $d \neq 0$. Then $d \in (d')$, and so $d = sd'$. Similarly, $d' = td$. We can substitute this value of $d'$ to get $d = sd' = std$. We thus have the equality $std = d$, which implies $d(st - 1) = 0$. Since $R$ is an integral domain and $d$ is non-zero, $st - 1 = 0$, and so $s$ is the inverse of $t$, meaning both $s$ and $t$ are units. Since $d = sd'$, $d$ differs from $d'$ by multiplication by a unit. 

\pre Suppose $d$ and $d'$ are both greatest common divisors for $a, b \in R$. Then if $c$ is a divisor of $a$ and $b$, $c$ divides $d$. Since $d'$ is a divisor of $a$ and $b$, $d' \mid d$. So, $(d) \subset (d')$. Similarly, $(d') \subset (d)$. Thus, $(d) = (d')$, and we just proved that this implies $d$ and $d'$ differ by multiplication by a unit.

\end{solution}

\begin{problem}{Problem 2}
Consider $R = \Z[\sqrt{-5}]$ the quadratic integer ring. This is an important ring in number theory, and it has an associated field norm $N(a+b\sqrt{-5}) = a^2+5 b^2$, where $a, b \in \Z$.  Consider the ideal $I = (3, 2+ \sqrt{-5})$. Suppose that $I$ were principal, and generated by some element $c+d\sqrt{-5}$. 
Then we must have that $3= \alpha (c+d\sqrt{-5})$ and $2+\sqrt{-5} = \beta(c+d \sqrt{-5})$ for some $\alpha, \beta \in R$. 
Derive a contradiction and show that $R$ is not a PID.
You may assume the norm is multiplicative, so $N(xy) = N(x)N(y)$ for all $x, y, \in R$, and furthermore that $N(u) = 1$ if and only if $u$ is a unit. 
\\ Hint: first determine what all the units of $R$ are. Then do a case analysis on the possible norms of $c+ \sqrt{-5} d$.  
\end{problem}

\begin{solution}


\pre Per the hint, we first determine the units of $R$ using the fact that $N(u) = 1$ if and only if $u$ is a unit. Every element of $\Z[\sqrt{-5}]$ is of the form $x + y\sqrt{-5}$ for some $x, y \in \Z$. If $N(x + y\sqrt{-5}) = 1$, then $x^2 + 5y^2 = 1$. Since $x^2 \geq 0$ for $x \in \Z$ and since $x^2 + 5y^2 \geq 5$ if $y \neq 0 \in \Z$, the only possible values for $x + y\sqrt{-5}$ are $x = \pm 1$ and $y = 0$. Thus, the only units in $R$ are $\pm 1$. 

\pre Now, by the setup of the problem,  we can represent $3$ as $\alpha(c + d\sqrt{-5})$, so the norms of both are equal. Since $N(3) = 9$, we have that $N(\alpha(c + d\sqrt{-5})) = N(\alpha)N(c+d\sqrt{-5}) = 9$. Similarly, since $2 + \sqrt{-5} = \beta(c + d\sqrt{-5})$ and since $N(2 + \sqrt{-5}) = 9$, we have that $N(\beta(c + d\sqrt{-5})) = N(\beta)N(c+d\sqrt{-5}) = 9$. 

\pre Since $N$ is a map into the non-negative integers, and $N(c+d\sqrt{-5})$ divides 9, the only possible values are 1, 3, and 9. We do a case analysis of each. 

\pre If $N(c+d\sqrt{-5}) = 1$, then $c + d\sqrt{-5}$ is a unit, and so the ideal generated by it is the entirely of $R$. However, $I \neq R$, since $1 \in R$ but $1 \notin I$. This is because if $1 = 3x + (2 + \sqrt{-5})y$, then $y$ must equal 0 to remove the $\sqrt{-5}$ term, so $1 = 3x$, but 3 has no inverse in $\Z$.

\pre If $N(c+d\sqrt{-5}) = 3$, then $c^2 + 5d^2 = 3$. Since $c, d \in \Z$, if $d \neq 0$, $c^2 + 5d^2 \geq 5$, so $d$ must be 0, and $c^2 = 3$ has no solutions in the integers. Thus, this norm is also not possible. 

\pre If $N(c+d\sqrt{-5}) = 9$, then $c^2 + 5d^2 = 9$, which has solutions $c = \pm 3$ and $d = 0$. However, the ideal generated by $3$ or $-3$ does not contain $2 + \sqrt{-5}$, and so it cannot generate $I$.

\pre Since the norm of a potential generator for $I$ must be either 1, 3, or 9, and in all cases we derive a contradiction, there can be no single generator for $I$, and so $\Z[\sqrt{-5}]$ is not a PID. 
\end{solution}

\begin{problem}{Problem 3}
Show that $\Z[2i] = \{ a+bi | a, b\in \Z \text{ and $b$ is even} \}$ is not a UFD.  Note: the norm in this ring is $a^2 + b^2$. Think about small numbers and find a number that factors in two ways, prove that these factorizations are each irreducibles in $\Z[2i]$. (You can again use without proof that the norm is multiplicative , so $N(xy) = N(x)N(y)$ for all $x, y, \in R$, and furthermore that $N(u) = 1$ if and only if $u$ is a unit.)
\end{problem}

\begin{solution}

\pre Consider my $3^\text{rd}$ favorite number, which is $-4$. We can factor $-4$ as $-2 \cdot 2$ as well as $2i \cdot 2i$. We will show that $2, -2,$ and $2i$ are irreducible, since they are all clearly non-zero, non-units, and the below calculations will show they do not factor into non-units.

\pre Notice the the norm of each is 4, since 
\begin{align*}
    N(2 + 0i) &= 2^2 + 0^2 = 4 \\
    N(-2 + 0i) &= (-2)^2 + 0^2 = 4 \\
    N(0 + 2i) &= 0^2 + 2^2 = 4
\end{align*}

\pre So, if any of these elements can be factored into non-units $a$ and $b$, it must be that $$4 = N(ab) = N(a)N(b)$$

\pre Since the norm function on $\Z[2i]$ is a map into the non-negative integers, $N(a)$ must equal 1, 2, or 4. Since $a$ and $b$ were assumed to be non-units, it cannot be that $N(a) = 1$ or $4$, and so $N(a) = N(b) = 2$. 

\pre Let $a = x + yi$, so $2 = N(a) = N(x + yi) = x^2 + y^2$. This has no solutions when $x, y \in \Z$ and $y$ is even, since $x^2 + y^2 \geq 4$ when $y \neq 0$, and there is no $x \in Z$ with $x^2 = 2$. Thus, each of $2, -2$, and $2i$ are irreducible. 

\pre We have factored $-4$ in two unique ways, each of which consists of irreducibles in $\Z[2i]$. Thus, $\Z[2i]$ is not a UFD. 
\end{solution}

\begin{problem}{Problem 4}
From the lecture, we proved that if we are in a PID and we have factorization into irreducibles, then we get uniqueness of factorization. In this next pair of problems, you will prove that every element of a PID has a factorization into irreducibles. First, start by showing we get a single factor. Let $R$ be a PID and $a \neq 0 \in R$ be a non-unit. Prove that $a$ has an irreducible factor.  

\pre Hint: If $a$ is irreducible, then we are done. Suppose $a = a_0$ is reducible and has no irreducible factor. We can pick a nontrivial factor $a_1 | a_0$, and $a_1$ cannot have an irreducible factor either (why?). We get an infinite sequence $ \dots a_2 |a _1 |a_0$ which gives a sequence of ideals (why?) \[ (a_0) \subsetneq (a_1) \subsetneq (a_2) \subsetneq \dots\] Now consider the ideal $I = (a_0, a_1, a_2, \dots)$ generated by all the $a_i$. Using the fact that $R$ is a PID write $I = (b)$ and $b = \sum_{i=0}^n c_ia_i$ for some $n \in \N$, and derive a contradiction.
\end{problem}

\begin{solution}

\pre Let $a \neq 0 \in R$ be a non-unit. If $a$ is irreducible, then it has a factorization into irreducibles, so suppose $a$ is reducible and suppose it has no irreducible factors. Define $a_0 = a$ and recursively define $a_{i+1}$ by taking non-unit, non-zero factors $a_{i+1} \cdot a_{i+1}'$ of $a_i$. This is a valid definition since $a$ is reducible and divides itself. No $a_i$ has an irreducible factor, since such a factor would also divide $a$, contradiction out initial hypothesis. 

\pre By construction, $a_i = a_{i+1}\cdot a_{i+1}'$, and so we have an infinite sequence of divisors $\dots a_2 | a_1 | a_0$. 

\pre For each pair $a_{i + 1} | a_i$, we have that $(a_i) \subsetneq (a_{i+1})$. The subset comes from the fact that if $a_i = a_{i+1} \cdot a_{i + 1}'$, then if $x \in (a_i)$, $x = a_id$, and so $x = a_{i+1}a_{i+1}'d \in (a_{i+1})$. The inequality comes from the fact that $a_{i+1}$ was assumed to be a non-trivial factor of $a_i$, and so $a_{i+1} \notin (a_i)$. This is because if $a_{i+1} \in (a_i)$, then we would have the following contradiction:

\begin{align*}
    a_{i+1} &= ra_i \\
    a_{i+1} &= ra_{i+1}a_{i+1}' \\
    0 &= a_{i+1}(ra_{i+1}' - 1) \\
    1 &= ra_{i+1}'
\end{align*}

\pre The last equality comes from the fact that $a_{i+1}$ is non-zero and $R$ is a PID, and the contradiction is that $a_{i+1}'$ was assumed to be a non-unit. 


\pre We thus get an infinite sequence of ideals $$(a_0) \subsetneq (a_1) \subsetneq (a_2) \subsetneq \dots$$

\pre Consider the union $I = \bigcup_{i=0}^\infty (a_i)$. To show $I$ is an ideal, we show that it is non-empty, is closed under subtraction, and is closed under multiplication from $R$.

\pre Since $a_0 \in I$, $I$ is non-empty. If $x, y \in I$, then $x \in (a_j)$ and $y \in (a_k)$ for some $j, k \in \N$. Without loss of generality, suppose $j \leq k$, and so $(a_j) \subset (a_k)$. Then $x, y \in (a_k)$, and since $(a_k)$ is an ideal, $x - y \in (a_k) \subset I$. Thus, $I$ is closed under subtraction. 

\pre Let $x \in I$ and $r \in R$. Then $x \in (a_j)$ for some $j \in \N$. Since $(a_j)$ is an ideal, $xr \in (a_j) \subset I$, so $I$ is closed under multiplication. 

\pre Since $I$ satisfies all these properties, it is an ideal. Then since $R$ is a PID, $I = (b)$ for some $b \in R$. Since $b \in I$, there must be some $i$ such that $b \in (a_i)$. Then $(b) \subset (a_i) \subset (a_{i + 1}) \subset \dots \subset (b)$, and so $(a_i) = (a_{i + 1}) = (a_{i + 2}) = \dots$, which contradicts that fact that $(a_i) \subsetneq (a_{i + 1}) \subsetneq (a_{i + 2}) \subset \dots$. 
\pre Thus, some $a_i$ must have an irreducible factor in order to prevent the infinite chain of strict subsets, and this implies that $a$ has an irreducible factor. 
\end{solution}

\begin{problem}{Problem 5}
Show that if $R$ is a PID then every nonzero element $a \in R$ has a factorization into irreducible factors (and a unit), i.e. 
\[a = u \prod_{i =0}^n p_i ^{d_i}\] where $u \in R$ is a unit, $p_i \in R$ are irreducible, $n \in \N$, and $d_i \in \N$.  As a corollary, deduce that every PID is a UFD using the theorems from class.
\end{problem}

\begin{solution}


\pre As proved in Problem 4, every non-zero element $a \in R$ has an irreducible factor, so $a = p_1a_1$, where $p_1$ is irreducible. If $a_1$ is irreducible or a unit, then we are done. Otherwise, we can apply this same principle to factor $a_1$ as $p_2a_2$. Since every non-zero element has an irreducible factor, we can factor each $a_i$ into $p_{i + 1}a_{i + 1}$ where $p_{i + 1}$ is irreducible. This factorization process must bottom out at some point, in the sense that some $a_{i}$ is a unit, because if not, we arrive at the infinite ascending chain of ideals which was shown in Problem 4 to be contradictory in a PID. It may be the case that some $p_{i} = p_{j}$, and so we account for multiplicity in our factorization. We have thus arrived at the following factorization:

\begin{align*}
    a &= p_1a_1 \\
    &= p_1p_2a_2 \\
    &= p_1p_2\dots a_k \text{\hspace{10mm}where $a_k$ is a unit}\\
    &= u\prod_{i=0}^n p_i^{d_i}
\end{align*}

\pre As stated at the top of Problem 4, if we are in a PID and we have factorization into irreducibles, we get uniqueness of factorization. Since we have just factored an arbitrary element $a$ into irreducibles, we have uniqueness of factorization for every element $a \in R$. Thus, if $R$ is a PID, it is a UFD.

\end{solution}

\end{document}