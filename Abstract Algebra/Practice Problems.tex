\documentclass{hmwk}

\hdr{Final Exam Review}{MATH 1530: Abstract Algebra}{Fall 2024}

\begin{document}

\section{Ring Theory}

\begin{itemize}
    \item[1.] What are all the group homomorphisms from $\Z/6\Z \rarr \Z/4\Z$? Ring homomorphisms?
\end{itemize}

\pre Let $\phi: \Z/6\Z \rarr \Z/4\Z$ be a group homomorphism. Then $\phi(0) = 0$ and $\phi(6) = 6\phi(1) = 0$, so $\phi(1) = 0$ or $\phi(1) = 2$. Since $\Z/6\Z$ is cyclic and generated by $1$, $\phi$ is determined uniquely by $\phi(1)$, so the only group homomorphisms are $\phi(1) = 0$ and $\phi(1) = 2$.

\pre Let $\phi: \Z/6\Z \rarr \Z/4\Z$ be a ring homomorphism. From above, we know that $\phi(1) = 0$ or 2. But, if $\phi(1) = 2$, then $\phi(3) = \phi(1) + \phi(1) + \phi(1) = 2$ and $\phi(3) = \phi(3)\phi(1) = 2 \cdot 2 = 0$, a contradiction. So the only ring homomorphism is $\phi = 0$. 

\begin{itemize}
    \item[2.] Check whether the map $\Z[x] \rarr \Z \times \Z$ taking $p(x) = a_nx^n + \dots + a_1x + a_0  \mapsto (a_1, a_0)$ is a homomorphism. 
\end{itemize}

\pre Let $p(x) = \dots + a_1x + a_0$ and $q(x) = \dots + b_1x + b_0$. Then $\phi(p(x) + q(x)) = (a_1 + b_1, a_0 + b_0) = (a_1, a_0) + (b_1, b_0) = \phi(p(x) + \phi(q(x))$. However, $\phi(p(x)q(x)) = (a_1b_0 + a_0b_1, a_0b_0) \neq (a_1b_1, a_0b_0) = \phi(p(x))\phi(q(x))$. So this map is a group homomorphism but not a ring homomorphism. 

\begin{itemize}
    \item[3.]  Prove that the set of polynomials in $\Z[x]$ that satisfy the property that the sum of their linear and constant terms are divisible by 5 is NOT an ideal of $\Z[x]$. However, prove that the set of polynomials satisfying the property that the constant term and the linear term are divisible by 5 is an ideal.
\end{itemize}

\pre Let $I$ represent the first set in question. Then $4x + 1 \in I$, $2x + 2 \in \Z[x]$, but $(4x + 1)(2x + 1) \notin I$, so $I$ is not an ideal. 

\pre Let $I$ represent the second set in question. Then $0 \in I$ so $I$ is non-empty. If $a = \dots + a_1x + a_0, b = \dots + b_1x + b_0 \in I$, then $a - b = \dots + (a_1 - b_1)x + (a_0 - b_0)$, and $a_1 - b_1$ and $a_0 - b_0$ are both divisible by 5 since each term individually is. If $a \in I$ and $r \in \Z[x]$, then $ar = \dots + a_1r_1x + a_0r_0$, and $a_1r_1$ and $a_0r_0$ are both divisible by 5 since $a_1$ and $a_0$ are. 

\begin{itemize}
    \item[4.] Decide whether each of the following are prime ideals, maximal ideals, or neither:
    \begin{itemize}
        \item[(a)] $(x - 3) \subset \R[x]$
        \item[(b)] $(x^2 + 4) \subset \C[x]$
        \item[(c)] $(3) \subset \Z/12\Z$
        \item[(d)] $(x^2) \subset \R[x]$
    \end{itemize}
\end{itemize}

\pre Consider the map $\phi: \R[x] \rarr \R$ defined by $\phi(p(x)) = p(x)$. Then $(x - 3)$ is the kernel of this map, so $\R[x]/(x - 3) \cong \R$. Since $\R$ is a field, so is $\R[x]/(x - 3)$, and so $(x - 3)$ is a maximal ideal and therefore also prime. 

\pre $(x + 2i)(x - 2i) = x^2 + 4 \in (x^2 + 4)$, but $x + 2i, x - 2i \notin (x^2 + 4)$ so $(x^2 + 4)$ is not prime and therefore not maximal.

\pre Suppose $(3) \subsetneq (a) \subsetneq \Z/12\Z$. Then there is some $x \in (a)$ that is not a multiple of 3. $\gcd(x, 3) \in (a)$, but since 3 is prime and $x$ is not a multiple of $3$, $\gcd(x, 3) = 1$, so $1 \in (a)$ and $(a) = \Z/12\Z$. Therefore $(3)$ is maximal and therefore prime. 

\pre $x \notin (x^2)$ but $x \cdot x \in (x^2)$, so $(x^2)$ is not prime and therefore not maximal. 

\begin{itemize}
    \item[5.] True or false:
    \begin{itemize}
        \item[(a)] Every pair of elements in an integral domain has a gcd.
        \item[(b)] In an integral domain, if $ab = ac$ then $b = c$, $a\neq 0.$
        \item[(c)] If $r \in R$ is irreducible and $r = abc$ then at least two of $a, b,$ and $c$ have to be units.
        \item[(d)] The preimage of a maximal ideal under a ring homomorphism is maximal.
    \end{itemize}
\end{itemize}

\pre False. In $\Z[\sqrt{-3}]$, $4 = 2 \cdot 2 = (1 - \sqrt{-3})(1 + \sqrt{-3})$. $2$ and $1 + \sqrt{-3}$ are both maximal divisors but they are not associates, so there is no gcd. 

\pre (Assuming this question is asking if $ab = ac \implies b=c$ and $ab = ac \implies a \neq 0$) False. In an integral domain, $ab = ac$ implies that either $a = 0$ or $a \neq 0$ and $b = c$. 

\pre True. Suppose $a$ and $b$ are non-units. Then $r$ factors into $a$ and $bc$, which are non-units, so $r$ is reducible. 

\pre False. Let $\phi: \Z \rarr \Q$ be the inclusion map. Then $\{0\}$ is a maximal ideal in $\Q$ since $\Q$ is a field, but $\phi^{-1}(\{0\}) = \{0\}$, $\{0\}$ is not maximal in $\Z$. This fact is only true if $\phi$ is surjective. 

\begin{itemize}
    \item[6.] Prove that $\Z[x]$ and $\Q[x]$ are not isomorphic as rings. 
\end{itemize}

\pre Since $\Q$ is a field, $\Q[x]$ is a PID, but $\Z[x]$ is not a PID since $(2, x)$ cannot be generated by a single element. 

\begin{itemize}
    \item[7.] Suppose we have surjective ring homomorphisms, $\phi : R \rarr S$ and $\psi : R \rarr T$ such that $\ker \psi \subseteq \ker \phi$. Prove that there exists some ring homomorphism $\alpha : T \rarr S$, such that $\phi = \alpha \circ \psi.$ 
\end{itemize}

\pre Since $\phi$ and $\psi$ are surjective, we have that $R/\ker\phi \cong S$ and $R/\ker\psi \cong T$. Let $\Tilde{\psi}$ be the induced isomorphism from $T$ to $R/\ker\psi$ and $\Tilde{\phi}$ be the induced isomorphism from $R/\ker\phi$ to $S$. Let $\beta: R/\ker\psi \rarr R/\ker\phi$ defined by $\beta(r + \ker\psi) = r + \ker\phi$. Then $\alpha = \Tilde{\phi}\circ\beta\circ\Tilde{\psi}$ is a map such that $\phi = \alpha \circ \psi$.

\begin{itemize}
    \item[8.] Show that if $M \subset S$ is a maximal ideal and $\phi$ is surjective then $\phi^{-1}(M)$ is a maximal ideal of $R$. Give an example to show that this need not be the case when $\phi$ is not surjective. 
\end{itemize}

\pre Let $\psi: S \rarr S/M$. Then $\ker(\psi\circ\phi) = \{x \in R : \psi(\phi(x)) = 0\} = \{x \in R : \phi(x) \in M\} = \phi^{-1}(M)$. So $S/M \cong R/\phi^{-1}(M)$, and since $M$ is maximal, $S/M$ is a field, so $R/\phi^{-1}(M)$ is a field, and $\phi^{-1}(M)$ is maximal. 

\pre See 5d for an example. 

\section{Group Theory}

\begin{itemize}
    \item[1.] Let $G$ be a group and $\phi \in Aut(G)$. Prove that $S := \{g \in G | \phi(g) = g\}$ is a subgroup of $G$. Deduce that if $\phi$ fixes more than half of the elements of $G$ then it is the identity on $G$.
\end{itemize}

\pre $\phi(e) = e$, so $e \in S$. If $a, b \in S$, then $\phi(a) = a$ and $\phi(b) = b$, so $\phi(ab^{-1}) = \phi(a)\phi(b^{-1}) = a\phi(b)^{-1} = ab^{-1}$, and so $ab^{-1} \in S$ and $S$ is a subgroup. If $|S| > |G|/2$, since $|S| \mid |G|$, is must be that $|S| = |G|$, and since $S \subset G$, $S = G$ and $\phi$ must be the identity on $G$. 

\begin{itemize}
    \item[2.] List all abelian groups of order $5^2 \cdot 7$ and $33$ up to isomorphism. 
\end{itemize}

\begin{itemize}
    \item $\Z/25\Z \times \Z/7\Z$
    \item $\Z/5\Z \times \Z/5\Z \times \Z/7\Z$
    \item $\Z/3\Z \times \Z/11\Z$
\end{itemize}

\begin{itemize}
    \item[3.] These results sometimes go under the name “The second isomorphism theorem”) Let $G$ be a group, let $H \subset G$ and $K \subset G$ be subgroups, and assume that $K$ is a normal subgroup of $G$.
    \begin{itemize}
        \item[(a)] Prove that $HK = \{hk : h \in H, k \in K\}$ is a subgroup of $G$.
        \item[(b)] Prove that $H \cap K$ is a normal subgroup of $H$ and that $K$ is a normal subgroup of $HK$.
        \item[(c)] Prove that $HK/K$ is isomorphic to $H/(H \cap K)$. Hint: What is the kernel of the surjective homomorphism $H \rarr HK/K$?
    \end{itemize}
\end{itemize}

\pre $e \in HK$ since $e \in H, e \in K$. Let $a, b \in HK$, so $a = h_1k_1$ and $b = h_2k_2$. Then $ab^{-1} = h_1k_2k_2^{-1}h_2^{-1}$. Since $K$ is a normal subgroup, $k_2k_2^{-1}h_2^{-1} = h_2^{-1}k_3$ for some $k_3 \in K$, so $ab^{-1} = h_1h_2^{-1}k_3 \in HK$ and $HK$ is a subgroup. 

\pre $e \in H \cap K$. Let $a, b \in H \cap K$. Then $a, b \in H$ and $a, b \in K$, so $ab^{-1} \in H$ and $ab^{-1} \in K$, so $ab^{-1} \in H \cap K$. Let $x \in H \cap K$ and $h \in H$. Then $hxh^{-1} \in H$ since 
\end{document}
