\documentclass{article}
\usepackage{graphicx} % Required for inserting images
\usepackage{amsfonts}
\usepackage{amsmath}
\usepackage{amssymb}
\usepackage{amsthm}

\title{Math 1530 PSet 3}
\author{Aly Rajwani}
\date{September 2024}

\begin{document}

\maketitle

\begin{enumerate}
    \item We will prove this is false by counterexample. Consider $U(8)$, which is the group of numbers coprime with 8 modulo 8. We have that $U(8) = \{1, 3, 5, 7\}$, and the proper subgroups of $U(8)$ are $\{1\} = \langle1\rangle, \{1, 3\} = \langle3\rangle, \{1, 5\} = \langle5\rangle,$ and $\{1, 7\} = \langle7\rangle$. Since each element in $U(8)$ is its own inverse, and since the order of $U(8)$ is 4, no element can generate the group, despite every proper subgroup being cyclic. Hence, the statement is false. 

    \item Let $H$ and $K$ be subgroups of $G$. Then, by definition of a subgroup, $e \in H$ and $e \in K$, so $H \cap K \neq \emptyset$. Now, we use the two-step subgroup test, showing that $a, b \in H\cap K$ implies that  $ab \in H\cap K$ and that $a \in H \cap K$ implies that $a^{-1} \in H \cap K$. First, since $a, b \in H \cap K$, $a, b \in H$ and $a, b \in K$. Then, since $H$ and $K$ are subgroups, they are closed under the group operation, so $ab \in H$ and $ab \in K$. These two facts together imply that $ab \in H \cap K$, so $H\cap K$ is closed under the group operation. Next, if $a \in H\cap K$, then $a \in H$ and $a \in K$. Then, since $H$ and $K$ are subgroups, $a^{-1} \in H$ and $a^{-1} \in K$. These two facts together imply that $a^{-1} \in H\cap K$. Since $H \cap K$ contains the identity, is closed under the group operation, and contains inverses, $H \cap K$ is a subgroup of $G$.

    \smallskip

    Next, if $H \not\subset K,$ there is an element of $H$ that is not in $K$. Similarly, if $K \not\subset H, \exists k \in K$ with $k \notin H$. Consider $H \cup K$. Since $h \in H, h \in H \cup K$, and since $k \in K, k \in H \cup K$. Now, we will show that $H \cup K$ being a subgroup leads to a contradiction, and thus $H \cup K$ can never be a subgroup. Suppose $H \cup K$ is a subgroup. Then, since it is closed under the group operation and since $h, k \in H \cup K, hk \in H \cup K$. This implies $hk \in H$ or $hk \in K$. However, $h^{-1} \in H$, so $h^{-1}hk = k \in H$, but this is a contradiction. Similarly, $k^{-1} \in K$, so $hkk^{-1} = h \in K$, but this is a contradiction. Since $H\cup K$ being a subgroup leads to a contradiction, $H \cup K$ can never be a subgroup of $G$. 

    \item A wild biconditional has appeared! Use "prove both directions"!

    \textit{If $G$ is a finite group, it has a finite number of subgroups.}

    Since $G$ is finite, there are $2^{|G|}$ possible subsets of $G$. Since every subgroup is a subset of $G$, the number of subgroups is bounded by the number of subsets, so there are at most $2^{|G|}$ subgroups, which is finite. 

    \textit{If $G$ has a finite number of subgroups, $G$ is finite.}

    If $G$ has a finite number of subgroups, it must also have a finite number of cyclic subgroups. Suppose one of these cyclic subgroups was infinite, namely $\langle g \rangle = \{e, g, g^1, g^2, \dots\}$ where each $g^n$ is distinct. From this subgroup, we can generate other distinct subgroups $\langle g^p \rangle = \{e, g^{p}, g^{2p}, \dots\}$ where $p$ is prime. Since there are infinitely many primes $p$, there are infinitely many subgroups, which is a contradiction. Thus, there can be no cyclic subgroup of infinite order. Since every element of $G$ will generate a cyclic subgroup, we can represent $G$ as the union of all its cyclic subgroups. So, $G = \bigcup_{n=1}^N \langle g_n \rangle$ where $N$ is the number of cyclic subgroups. Since we have represented $G$ as a finite union of finite sets, $G$ itself must be finite. 

    \item 
        \begin{enumerate}
            \item To prove that this is a group under multiplication, we must show that it is closed under the operation, inverses exist, an identity exists, and the group is associative. 

            \smallskip

            Since the group operation is standard multiplication modulo $n$, which is already associative, we have associativity. 

            \smallskip

            As with standard multiplication modulo $n$, 1 is the identity, and 1 $\in U(n)$ because we can take $b=1$, and so $1\cdot b = 1$. 

            \smallskip

            Suppose we have some $a \in U(n)$. Then we want to show that there is an $a^{-1} \in U(n)$ such that $aa^{-1} = a^{-1}a = 1$. Since $a \in U(n)$, we know $\exists b \in \mathbb{Z}/n\mathbb{Z}$ such that $ab = 1$. We will take $b$ to be $a^{-1}$. We know $b \in U(n)$ because $ba = 1$, and $a \in \mathbb{Z}/n\mathbb{Z}$. Thus, each $a$ has an inverse in $U(n)$.

            \smallskip

            Suppose $a, c \in U(n)$. Then $\exists b, d \in \mathbb{Z}/n\mathbb{Z}$ such that $ab = cd = 1$. Consider $ac \mod n \in \mathbb{Z}/n\mathbb{Z}$. Then we have $bd \mod n\in \mathbb{Z}/n\mathbb{Z}$, and $(ac)(bd) = (ab)(cd) = 1 \cdot 1 = 1$, and so $ac \in U(n)$, so $U(n)$ is closed.  

            \smallskip

            Since $U(n)$ satisfies these properties, it is a group under multiplication. 

            \smallskip

            $U(n)$ is not a subgroup of $\mathbb{Z}/n\mathbb{Z}$. This is because $\mathbb{Z}/n\mathbb{Z}$ is only a group when defined with the addition operation, but $U(n)$ uses multiplication. If we attempt to use multiplication for $\mathbb{Z}/n\mathbb{Z}$, we find that any element $k$ satisfying $\gcd(k, n) > 1$ will not have an inverse, and so it would not be a group. Thus, $U(n)$ cannot be a subgroup of $\mathbb{Z}/n\mathbb{Z}$, since they cannot share the same operation. 

            \item The number of elements in $U(n)$ is the number of elements in $\mathbb{Z}/n\mathbb{Z}$ which are coprime with $n$. This number is expressed as $\varphi(1) = 0$ and $\varphi(n) = n \cdot \prod_{p|n}\left(1 - \frac{1}{p}\right)$ where $p$ is a prime. 

            \item We will prove this by contradiction. Suppose that $G$ is cyclic, namely that $G = \langle g\rangle$, and we have two elements $a, b \in G$ satisfying $a \neq b, a^2 = e, b^2 = e$. Since every element in this group can be represented as $g^k$, so we set $a = g^i$ and $b = g^j$, and suppose $i < j$. Then, we have that $g^{2i} = e$ and $g^{2j} = e$, so $g^{2i} = g^{2j}$. If $G$ is infinite, then we have a contradiction, since each element in an infinite cyclic group is distinct (otherwise it would not be an infinite group), and yet we have $g^{2i} = g^{2j}$ but $2i \neq 2j$. If $G$ is finite, let $n = |g|$. Then the only solutions to $g^{2x} = e$ modulo $n$ are $x = 0$ and $x = \frac{n}{2}$. Since $a, b \neq e$, it must be that $i \equiv j \equiv \frac{n}{2} \mod n$. So, we can express $i$ as $\frac{n}{2} + kn$ and $j$ as $\frac{n}{2} + k'n$
            
            We then have the following: $$a =  g^i= g^{\frac{n}{2} + kn} = g^{\frac{n}{2}}g^{kn} = g^{\frac{n}{2}}g^{k'n} = g^{\frac{n}{2} + k'n} = g^{j} = b$$ 
            
            where $g^{kn} = g^{k'n} = e$ because $n$ is the order of $g$. But, this contradicts $a$ and $b$ being distinct. Thus, we cannot have a cyclic group with two distinct elements of order 2. 

            \smallskip

            We will construct two elements of order 2, which shows that $U(2^n)$ is not cyclic by the above argument. Consider $2^n - 1$. For $n \geq 3$, this is an element of $U(2^n)$. We have that $\left(2^n - 1\right)^2 \equiv 2^{2n} - 2^{n+1} + 1 \equiv 1 \mod 2^n$, and so $2^n - 1$ has order 2. Next, consider $2^{n-1} + 1$. For $n \geq 3$, this is an element of $U(2^n)$. For $n \geq 3$, we have that $2n - 2 > n$, and so we have that $\left(2^{n-1} + 1\right)^2 \equiv 2^{2n-2} + 2^n + 1 \equiv 1 \mod 2^n$. So, $2^{n-1} + 1$ also has order 2. For $n \geq 3$, $2^n - 1 \neq 2^{n-1} + 1$. Since we have found two distinct elements of order 2, $U(2^n)$ cannot be cyclic. 

            \item $U(2) = \{1\} = \langle 1 \rangle$

            $U(3) = \{1, 2\} = \langle 2 \rangle$

            $U(5) = \{1, 2, 3, 4\} = \langle 2 \rangle$
        \end{enumerate}
    \item We will prove this by demonstrating that $e$ is in $\langle a \rangle \cap \langle b \rangle$, and then showing that any $g \in \langle a \rangle \cap \langle b \rangle$ must be equal to $e$. Thus, $\langle a \rangle \cap \langle b \rangle$ must be $\{e\}$. First, $e \in \langle a \rangle$ and $e \in \langle b \rangle$, so $e \in \langle a \rangle \cap \langle b \rangle$. Before proceeding, we note that by Bezout's Lemma, we have $\exists s, t \in \mathbb{Z}$ satisfying $ns + mt = 1$, where $n = |\langle a \rangle|$ and $m = |\langle b \rangle|$. This comes from the fact that $n$ and $m$ are coprime. As well, since $n$ is the order of $\langle a \rangle$, $g \in \langle a \rangle$ implies that $g^n = e$. By similar reasoning, $g^m = e$. Now, we have the following:

    $$g = g^{ns+mt} = (g^n)^s(g^m)^t = e^se^t = e$$ 

    So, any $g \in \langle a \rangle \cap \langle b \rangle$ must be equal to $e$, and thus $\langle a \rangle \cap \langle b \rangle = \{e\}$.

    \item We will use the Fundamental Theorem of Cyclic groups, namely that every subgroup of a cyclic group is cyclic. By showing $H$ is a subgroup of a cyclic group, we prove $H$ is cyclic. 

    \smallskip

    Take $g = \gcd(a_1, a_2, \dots, a_n)$ and $l = \text{lcm}(b_1, b_2, \dots, b_n)$, then let $h = \frac{g}{l}$. We will consider the cyclic group $\langle h \rangle$ under the addition operation. 

    \smallskip

    Choose $\frac{a_k}{b_k} \in H$ arbitrarily. Then, by definition of $g$, $gt = a_k$ for some $t \in \mathbb{Z}$. By definition of $l, l = sb_k$ for some $s \in \mathbb{Z}$. Then, $\frac{a_k}{b_k} = \frac{gt}{\frac{l}{s}} = st\cdot h$ where $st \in \mathbb{Z}$. Since $k$ was arbitrary, every element of $H$ is an element of $\langle h \rangle$. Since $H$ is a subgroup of a cyclic group, $H$ is cyclic by the Fundamental Theorem of Cyclic Groups. 
\end{enumerate}

\end{document}
