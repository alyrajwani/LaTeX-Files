\documentclass{article}
\usepackage{graphicx} % Required for inserting images
\usepackage{amsfonts}
\usepackage{amsmath}
\usepackage{amssymb}
\usepackage{amsthm}
\usepackage[margin=1.0in]{geometry}

\newcommand{\sk}{\smallskip}
\newcommand{\Z}{\mathbb{Z}}
\newcommand{\N}{\mathbb{N}}
\newcommand{\R}{\mathbb{R}}
\newcommand{\Q}{\mathbb{Q}}

\title{Math 1530 PSet 6}
\author{Aly Rajwani}
\date{\today}

\begin{document}

\maketitle

\begin{enumerate}

%%%%%%%%%%%%%%%%%%%%%%%%%%%%%%%%%%%%%%%%%%%%%%%%%%%%%%%%%%%%%%
    \item Let $\phi: G \rightarrow G'$.
    \begin{enumerate}
        \item Prove that the image of a subgroup under a homomorphism is a group. i.e. if $H \leq G$, then $\phi(H)$ is a group.

        \begin{proof}[Solution.]
            To show that $\phi(H)$ is a group, we must show that it has an identity, every element has an inverse, the operation is closed, and the operation is associative.

            \sk

            First, since $H$ is a subgroup, it must contain $e_G$, the identity of $G$. Then, since $\phi$ is a homomorphism, $\phi(e_G) = e_{G'}$, which is the identity of $\phi(H)$. This is true because $\phi(H) \subset G'$, and so the identity of $G'$ is the identity of $\phi(H)$.

            \sk

            Second, we show that every element has an inverse. We are given an element $\phi(a) \in \phi(H)$ with $a \in H$ and we want to find $\phi(a)^{-1}$. Since $H$ is a subgroup, $a^{-1} \in H$. Since $\phi$ is a homomorphism, $\phi(a^{-1}) = \phi(a)^{-1}$, and so the inverse of $\phi(a)$ is $\phi(a^{-1})$.

            \sk

            Third, we show that the operation is closed, namely that if $\phi(a), \phi(b) \in \phi(H)$ with $a, b \in H$, then $\phi(a)\phi(b) \in \phi(H)$. Since $\phi$ is a homomorphism, $\phi(a)\phi(b) = \phi(ab)$. Since $H$ is a subgroup, $a, b \in H$ implies that $ab \in H$. Thus, $\phi(a)\phi(b) = \phi(ab) \in \phi(H)$. 

            \sk

            Fourth, we show that the operation is associative. Since $\phi(H) \subset G'$ and since $G'$ is a group, $\phi(H)$ has associativity inherited from $G'$.

            \sk

            Since $\phi(H)$ has an identity, has inverses, and is closed, $\phi(H)$ is a group.
            
        \end{proof}
        
        \item Prove that the inverse image of a subgroup under a homomorphism is a group, i.e. if $H \leq G'$ then $\phi^{-1}(H)$ is a group.

        \begin{proof}[Solution.]
            To show that $\phi^{-1}(H)$ is a group, we must show that it has an identity, every element has an inverse, the operation is closed, and the operation is associative.

            \sk

            First, since $H$ is a subgroup of $G'$, the identity $e_{G'}$ is in $\phi^{-1}(H)$. So, $\phi^{-1}(e_{G'})$ is in $H$. Since $\phi$ is a homomorphism, $\phi^{-1}(e_{G'}) = e_G$, which is the identity of $\phi^{-1}(H)$. This is true because $\phi^{-1}(H) \subset G$, and so the identity of $G$ is the identity of $\phi^{-1}(H)$.

            \sk

            Second, we show that every element has an inverse. We are given an element $a \in \phi^{-1}(H)$ and we want to find $a^{-1}$. Since $a \in \phi^{-1}(H)$, we know $\phi(a) \in H$. Since $H$ is a subgroup, this implies $\phi(a)^{-1} = \phi(a^{-1}) \in H$. We can then apply $\phi$ to both sides to get $a^{-1} \in \phi(H)$. 

            \sk

            Third, we show that the operation is closed, namely that if $a, b \in \phi^{-1}(H)$, then $ab \in \phi^{-1}(H)$. Since $a, b \in \phi^{-1}(H)$, we know that $\phi(a), \phi(b) \in H$. Since $H$ is a subgroup, it is closed, and so $\phi(a)\phi(b) = \phi(ab) \in H$. Then, applying $\phi^{-1}$ to both sides, we get that $ab \in \phi^{-1}(H)$. 

            \sk
            
            Fourth, we show that the operation is associative. Since $\phi^{-1}(H) \subset G$ and since $G$ is a group, $\phi^{-1}(H)$ has associativity inherited from $G$.
            
        \end{proof}
    \end{enumerate}

%%%%%%%%%%%%%%%%%%%%%%%%%%%%%%%%%%%%%%%%%%%%%%%%%%%%%%%%%%%%%%%

    \item Let $H$ be a normal subgroup of $G$. There is a natural homomorphism $\pi: G \rightarrow G/H$ given by $\pi(g) = gH$ for all $g \in G$ (called the \textit{quotient map by }$H$)
    \begin{enumerate}
        \item Let $K$ be a subset of $G$. Prove that $\pi^{-1}\pi(K) = KH$. Deduce that if $K$ is a subgroup of $G$, then $KH$ is a subgroup of $G$.

        \begin{proof}[Solution.]
            We first show that $\pi^{-1}\pi(K) = KH$ by showing that each set is a subset of the other. 

            \sk

            If $K = \emptyset$, then $\pi^{-1}\pi(K) = \emptyset$ since there are no $k \in K$, and $KH = \emptyset$ for the same reason. In the empty case, the equality is true, so we can freely assume $K \neq \emptyset$, and so there exists an $x \in \pi^{-1}\pi(K)$. Since $x \in \pi^{-1}\pi(K)$, there is a $k \in K$ such that $x = \pi^{-1}\pi(k)$. Applying $\pi$ to both sides, $\pi(x) = \pi(k)$, and by definition of $\pi$, this implies $xH = kH$. This in turn implies $x \in kH$, and so $x = kh$ for some $h \in H$. By definition, if $x = kh$ for some $k \in K$ and some $h \in H$, then $x \in KH$. Thus, $\pi^{-1}\pi(K) \subset KH$.

            \sk

            Now suppose $x \in KH$, so $x = kh$ for some $k \in K$ and some $h \in H$. This implies that $xH = khH$. Since $h \in H$, $hH = H$, and so $khH = kH$. Thus, $\pi(x) = xH = kH$. Applying $\pi^{-1}$ to both sides, we get $\pi^{-1}\pi(x) = \pi^{-1}(kH)$. Thus, $\pi^{-1}\pi(x) = k \in K$. Thus, $\pi^{-1}\pi(x) \in K$, and so $x \in \pi^{-1}\pi(K)$. Thus, $KH \subset \pi^{-1}\pi(K)$. 

            \sk

            Since $\pi^{-1}\pi(K) \subset KH$ and $KH \subset \pi^{-1}\pi(K)$, it must be that $\pi^{-1}\pi(K) = KH$.

            \sk

            We now use this fact to deduce that if $K$ is a subgroup of $G$, then $KH$ is a subgroup of $K$. 

            \sk

            Since $K$ is a subgroup of $G$, it is a subset of $G$, and we just proved that for any subset $K$ of $G$, $\pi^{-1}\pi(K) = KH$. Thus, it suffices to show prove that $\pi^{-1}\pi(K)$ is a subgroup.

            \sk

            Recall from question 1 that the image of a subgroup under a homomorphism is a group, as is the inverse image under a homomorphism. We are given that $K$ is a subgroup, and since $\pi$ is a homomorphism, $\pi(K)$ is also a subgroup. Again using that $\pi$ is a homomorphism and that $\pi(K) $ is a subgroup, we know that $\pi^{-1}\pi(K) \subset G$ is a subgroup of $G$. Thus, if $K$ is a subgroup of $G$, $KH$ is a subgroup of $G$.
        \end{proof}
        
        \item Give an example of a group $G$ with two subgroups $K_1, K_2$ for which $K_1K_2$ is not a subgroup of G.

        \begin{proof}[Solution.]
            Let $G = S_3$, $K_1 = \{(1), (12)\}$, and $K_2 = \{(1), (13)\}$. Then $K_1K_2 = \{(1), (12), (13), (132)\}$. However, $K_1K_2$ is not a subgroup, since $(13)(132) = (32) \notin K_1K_2$.            
        \end{proof}
    \end{enumerate}

%%%%%%%%%%%%%%%%%%%%%%%%%%%%%%%%%%%%%%%%%%%%%%%%%%%%%%%%%%%%%%%

    \item Let $H_1$ be a normal subgroup of $G_1$ and $H_2$ a normal subgroup of $G_2$.
    \begin{enumerate}
        \item Show that $H_1 \times H_2$ is a normal subgroup of $G_1 \times G_2$.

        \begin{proof}[Solution.]
            To show that $H_1 \times H_2$ is a normal subgroup of $G_1 \times G_2$, we must show that $H_1 \times H_2$ is a subgroup, and that for all $g_1 \times g_2 \i G_1 \times G_2$, $(g_1 \times g_2)(H_1 \times H_2) = (H_1 \times H_2)(g_1 \times g_2)$.

            \sk

            We start by showing $H_1 \times H_2$ is a subgroup using the two-step subgroup test. First, since each of $H_1$ and $H_2$ is a subgroup, they are non-empty, and so $H_1 \times H_2$ is non-empty. Now, suppose $a = h_1 \times h_2$, $b = h_1' \times h_2\$$, and $a, b \in H_1 \times H_2$. We want to show that $ab \in H_1 \times H_2$. We have that $ab = h_1h_1' \times h_2h_2'$. Since $H_1$ is a subgroup and since $h_1, h_1' \i H_1$, we know from closure that $h_1h_1' \in H_1$. Similarly, $h_2h_2' \in H_2$. Thus, $ab = h_1h_1' \times h_2h_2' \in H_1 \times H_2$. Next, we show that inverses exist in $H_1 \times H_2$. Let $a = h_1 \times h_2$. Since $H_1$ is a subgroup, there exists an $h_1^{-1} \in H_1$, and similarly for $H_2$. Thus, the inverse of $a$ is $h_1^{-1} \times h_2^{-1} \in H_1 \times H_2$. Since $H_1 \times H_2$ is non-empty, has closure, and has inverses, it is a subgroup. 

            \sk

            Now we show that for all $g_1 \times g_2 \in G_1 \times G_2$, $(g_1 \times g_2)(H_1 \times H_2) = (H_1 \times H_2)(g_1 \times g_2)$. Let $h_1 \times h_2$ be an arbitrary element of $H_1 \times H_2$ and let $g_1 \times g_2$ be an element of $G_1 \times G_2$. Then $(g_1 \times g_2)(h_1 \times h_2) = g_1h_1 \times g_2h_2$. Since $H_1$ is a normal subgroup, there exists an $h_1' \in H_1$ such that $g_1h_1 = h_1'g_1$, and since $H_1$ is a normal subgroup, there exists an $h_2' \in H_2$ such that $g_2h_2 = h_2'g_2$. Then, $h_1' \times h_2' \in H_1 \times H_2$, and $(g_1 \times g_2)(h_1 \times h_2) = (h_1' \times h_2')(g_1 \times g_2)$. This shows that $(g_1 \times g_2)(H_1 \times H_2) \subset (H_1 \times H_2)(g_1 \times g_2)$. The proof of the reverse direction is analogous, and so $g_1 \times g_2)(H_1 \times H_2) = (H_1 \times H_2)(g_1 \times g_2)$
        \end{proof}
        
        \item Show that if $A_1 \subset G_1$ and $A_2 \subset G_2$ are cosets of $H_1$ and $H_2$ respectively, then $A_1 \times A_2$ is a coset of $H_1 \times H_2$.

        \begin{proof}[Solution.]
            Without loss of generality, we will show that this statement holds for left cosets. We are given that $A_1$ is a coset of $H_1$, and so $A_1 = g_1H_1$ for some $g_1 \in G_1$. Similarly, $A_2 = g_2H_2$.  We want to show that $A_1 \times A_2 = (g_1 \times g_2)(H_1 \times H_2)$. 
            
            \sk
            
            We first show that $A_1 \times A_2 \subset (g_1 \times g_2)(H_1 \times H_2)$. Let $a_1 \times a_2 \in A_1 \times A_2$. Then $a_1 \in A_1$ and $a_2 \in A_2$, so $a_1 = g_1h_1$ for some $h_1 \in H_1$ and $a_2 = g_2 \times h_2$ for some $h_2 \in H_2$. Then $a_1 \times a_2 = g_1h_2 \times g_2h_2 = (g_1 \times g_2)(h_1 \times h_2)$, and this is an element of $(g_1 \times g_2)(H_1 \times H_2)$. Thus, $A_1 \times A_2 \subset (g_1 \times g_2)(H_1 \times H_2)$.

            \sk

            Now we show that $(g_1 \times g_2)(H_1 \times H_2) \subset A_1 \times A_2$. Suppose that $(g_1 \times g_2)(h_1 \times h_2) = g_1h_1 \times g_2h_2 \in (g_1 \times g_2)(H_1 \times H_2)$ for some $h_1 \in H_1$ and $h_2 \in H_2$. Since $A_1 = g_1H_1$, this means that $g_1h_1 \in A_1$. Since $A_2 = g_2H_2$, this means that $g_2h_2 \in A_2$. Thus, $g_1h_1 \times g_2h_2 \in A_1 \times A_2$, and so $(g_1 \times g_2)(H_1 \times H_2) \subset A_1 \times A_2$.

            \sk

            Since both these sets are subsets of each other, we have that $(g_1 \times g_2)(H_1 \times H_2) = A_1 \times A_2$.
        \end{proof}
        
        \item Prove that every coset of $H_1 \times H_2$ is of this form. 
        \begin{proof}[Solution.]
            To prove that every coset of $H_1 \times H_2$ is of this form, we pick an arbitrary coset $(g_1 \times g_2)(H_1 \times H_2)$ and show that it is equal to $A_1 \times A_2$ where $A_1$ is a coset of $H_1$ and $A_2$ is a coset of $H_2$. Without loss of generality, we will show that this is true of left cosets. 

            \sk

            $(g_1 \times g_2)(H_1 \times H_2)$ is the set $\{g_1h_1 \times g_2h_2 : h_1 \in H_1, h_2 \in H_2\}$. The set $g_1H_1 \times g_2H_2$ is the set $\{g_1h_1 \times g_2h_2 : h_1 \in H_1, h_2 \in H_2\}$. Since these sets have equivalent representations, they are equal, so $(g_1 \times g_2)(H_1 \times H_2) = g_1H_1 \times g_2H_2$. Since $g_1H_1$ is a coset of $H_1$ and $g_2H_2$ is a coset of $H_2$, we have expressed an arbitrary coset of $H_1 \times H_2$ as $A_1 \times A_2$ where $A_1$ is a coset of $H_1$ and $A_2$ is a coset of $H_2$. Thus, every coset of $H_1 \times H_2$ is of this form.
        \end{proof}
        
        \item Prove that $(G_1 / H_1) \times (G_2 / H_2) \cong (G_1 \times G_2)/(H_1 \times H_2)$
        \begin{proof}[Solution.]
            To prove that $G_1 / H_1) \times (G_2 / H_2)$ is isomorphic to $(G_1 \times G_2)/(H_1 \times H_2)$, we will construct an isomorphism between them. 

            \sk

            Let $\phi: (G_1 / H_1) \times (G_2 / H_2) \rightarrow (G_1 \times G_2)/(H_1 \times H_2)$ be defined by $\phi(g_1H_1 \times g_2H_2) = (g_1 \times g_2)(H_1 \times H_2)$. To show that this is indeed an isomorphism, we must show that it is well-defined, injective, surjective, and operation-preserving. 

            \sk

            First, we show that $\phi$ is well-defined, namely that if $g_1H_1 \times g_2H_2 = g_1'H_1 \times g_2'H_2$, then $\phi(g_1H_1 \times g_2H_2) = \phi(g_1'H_1 \times g_2H_2')$. We assume that $g_1H_1 \times g_2H_2 = g_1'H_1 \times g_2'H_2$. This implies that $g_1H_1 = g_1'H_1$ and $g_2H_2 = g_2'H_2$. So, $g_1^{-1}g_1' \in H_1$ and $g_2^{-1}g_2 \in H_2$, and so $g_1^{-1}g_1' \times g_2^{-1}g_2 = (g_1^{-1} \times g_2^{-1})(g_1' \times g_2') \in H_1 \times H_2$. Thus, $g_1' \times g_2' \in (g_1 \times g_2)(H_1 \times H_2)$. This implies that $(g_1' \times g_2')(H_1 \times H_2) = (g_1 \times g_2)(H_1 \times H_2)$, and so $\phi(g_1H_1 \times g_2H_2) = \phi(g_1'H_1 \times g_2'H_2)$, so $\phi$ is well-defined.

            \sk
            
            Second, we show that $\phi$ is injective, namely that if $\phi(g_1H_1 \times g_2H_2) = \phi(g_1'H_1 \times g_2'H_2)$, then $g_1H_1 \times g_2H_2 = g_1'H_1 \times g_2'H_2$. We assume that $\phi(g_1H_1 \times g_2H_2) = \phi(g_1'H_1 \times g_2'H_2)$, which means that $(g_1 \times g_2)(H_1 \times H_2) = (g_1' \times g_2')(H_1 \times H_2)$. This implies that $g_1 \times g_2 \in (g_1' \times g_2')(H_1 \times H_2)$, and so there exists an $h_1 \in H_1$ and an $h_2 \in H_2$ such that $g_1 \times g_2 = g_1'h_1 \times g_2'h_2$. This equality implies equality of the components, so $g_1 = g_1'h_1$ and $g_2 = g_2'h_2$, so $g_1 \in g_1'H_1$ and $g_2 \in g_2'H_2$. Then, $g_1H_1 = g_1'H_1$ and $g_2H_2 = g_2'H_2$. Finally, since these components are equal, their product will be equal, so $g_1H_1 \times g_2H_2 = g_1'H_1 \times g_2'H_2$. Thus, $\phi(g_1H_1 \times g_2H_2) = \phi(g_1'H_1 \times g_2'H_2)$ implies that $g_1H_1 \times g_2H_2 = g_1'H_1 \times g_2'H_2$, and so $\phi$ is injective.

            \sk

            Third, we show that $\phi$ is surjective, namely that for all $(g_1 \times g_2)(H_1 \times H_2) \in (G_1 \times G_2)/(H_1 \times H_2)$, there exists $a \in G_1$ and $b \in G_2$ such that $\phi(aH_1 \times bH_2) = (g_1 \times g_2)(H_1 \times H_2)$. Take $a = g_1$ and $b = g_2$, so that $\phi(aH_1 \times bH_2) = \phi(g_1H_1 \times g_2H_2) = (g_1 \times g_2)(H_1 \times H_2)$.

            \sk

            Fourth, we show that $\phi$ is operation preserving, namely that $\phi(g_1H_1 \times g_2H_2)\phi(g_1'H_1 \times g_2'H_2) = \phi((g_1H_1 \times g_2H_2)(g_1'H_1 \times g_2'H_2))$. By definition, $\phi(g_1H_1 \times g_2H_2)\phi(g_1'H_1 \times g_2'H_2) = (g_1 \times g_2)(H_1 \times H_1) \cdot (g_1' \times g_2')(H_1 \times H_2)$. This is then equal to $(g_1 \times g_2)(g_1' \times g_2')(H_1 \times H_2)$, and since $(g_1 \times g_2)(g_1' \times g_2') = g_1g_1' \times g_2g_2'$, we have that $\phi(g_1H_1 \times g_2H_2)\phi(g_1'H_1 \times g_2'H_2) = (g_1g_1' \times g_2g_2')(H_1 \times H_2)$. Now we consider $\phi((g_1H_1 \times g_2H_2)(g_1'H_1 \times g_2'H_2))$. We first simplify the input to get $\phi(g_1g_1'H_1 \times g_2g_2'H_2)$, and by definition of $\phi$, this is equal to $(g_1g_1' \times g_2g_2')(H_1 \times H_2)$. Since both $\phi((g_1H_1 \times g_2H_2)(g_1'H_1 \times g_2'H_2))$ and $\phi(g_1H_1 \times g_2H_2)\phi(g_1'H_1 \times g_2'H_2)$ are equal to $(g_1g_1' \times g_2g_2')(H_1 \times H_2)$, they must be equal to each other, and so $\phi$ is operation-preserving. 

            \sk

            Since $\phi$ is well-defined, injective, surjective, and operation preserving, $\phi$ is an isomorphism between $(G_1/H_1) \times (G_1/H_2)$ and $(G_1 \times G_2)/(H_1 \times H_2)$, and so the two groups are isomorphic.
            
        \end{proof}
    \end{enumerate}

%%%%%%%%%%%%%%%%%%%%%%%%%%%%%%%%%%%%%%%%%%%%%%%%%%%%%%%%%%%%%%%

    \item Let $p$ be a prime. Consider $(\Z/p\Z)^n$.
    \begin{enumerate}
        \item How many elements of order $p$ are there in this group?
        \begin{proof}[Solution.]
            $(\Z/p\Z)^n = (\Z/p\Z) \times (\Z/p\Z) \times \dots \times (\Z/p\Z)$, so a typical element of $(\Z/p\Z)^n$ is $(a_1, a_2, \dots, a_n)$. This element has order $p$ if $\text{lcm}(|a_1|, |a_1|, \dots, |a_n|) = p$. Since $p$ is prime, it must be that at least one $|a_i|$ is $p$, the orders of the other components are either 1 or $p$. 

            \sk

            Let us focus on just $\Z/p\Z$. In this group, 0 has order 1, and every other element has order $p$. This is because if $a \in \Z/p\Z$, $a \neq 0$, and $an \equiv 0 \mod p$, then $p | an$, so $p | a$ or $p | n$. $p \nmid a$ since $0 < a < p$. So, $p | n$, and since $ap \equiv 0 \mod p$, $p$ must be the order of $a$. So, every non-zero element has order $p$. 

            \sk

            There are $p$ elements in $\Z/p\Z$, and so there are $p^n$ elements in $(\Z/p\Z)^n$. One element is $(0, 0, \dots, 0)$, and every other element contains some non-zero component. As we just established, any non-zero element in $\Z/p\Z$ has order $p$. Thus, when we take $\text{lcm}(|a_1|, |a_1|, \dots, |a_n|)$ where at least one component is non-zero, we get that the least common multiple is $p$. There are $p^n$ total elements, and all except for 1 have order $p$, so the total number of elements of order $p$ in $(\Z/p\Z)^n$ is $p^n - 1$.
        \end{proof}
  
        \item How many subgroups of order $p$ are there?

        \begin{proof}[Solution.]
            We just saw that there are $p^n - 1$ total elements of order $p$ in $(\Z/p\Z)^n$. An element $a \in (\Z/p\Z)^n$ generates the cyclic group $\{0, a, 2a, \dots, (p-1)a\}$, which has order $p$. A cyclic group of order $p$ has $\phi(p) = p-1$ total generators. So, there are $p^n - 1$ total elements of order $p$, each of which generates a group of order $p$, and these groups have $p-1$ distinct generators. So, there are $\frac{p^n - 1}{p-1}$ distinct subgroups of order $p$ in $(\Z/p\Z)^n$.
        \end{proof}
        
        \item How many subgroups of order $p^2$ are there? You can use the fact that the only groups of order $p^2$ are $\Z/p^2\Z$ or $\Z/p\Z \times \Z/p\Z$.

        \begin{proof}[Solution.]
            Since an element in $(\Z/p\Z)^n$ has order $p$ or 1, and there is an element of order $p^2$ in $\Z/p^z\Z$, a subgroup of $(\Z/p\Z)^n$ with order $p^2$ must be isomorphic to $\Z/p\Z \times \Z/p\Z$.

            \sk

            We first count the number of generators for subgroups isomorphic to $\Z/p\Z \times \Z/p\Z$, and then divide by the number of generators for a particular group to find the total number of distinct subgroups of order $p^2$. A subgroup of $(\Z/p\Z)^n$ with order $p^n$ is generated by two distinct elements with order $p$, where neither element is a multiple of the other. So, there are $p^n - 1$ choices for the first element, and $p^n - p$ for the second. The $p^n - 1$ comes from the answer to part (a), and the $p^n - p$ comes from the fact that there are $p$ elements in the subgroup generated by the first element. So, there are $(p^n - 1)(p^n - p)$ choices for generators of a subgroup with order $p^2$.

            \sk

            In each subgroup of order $p^2$, a pair of generators is any two elements with order $p$ which are not multiples of each other. So, there are $p^2 - 1$ choices for the first element, and $p^2 - p$ choices for the second. 

            \sk

            So, there are $(p^n - 1)(p^n - p)$ choices for generators of a subgroup with order $p^2$, and each of those subgroups has $(p^2 - 1)(p^2 - p)$ generators. The total number of subgroups of $(\Z/p\Z)^n$ with order $p^2$ is then $\frac{(p^n - 1)(p^n - p)}{(p^2 - 1)(p^2 - p)}$.
            
        \end{proof}

        \item (For fun, not graded) How many subgroups of order $p^r$ are there?
        \begin{proof}[Solution.]
            $\frac{(p^n - 1)(p^n - p)(p^n - p^2)\dots(p^n - p^{r-1})}{(p^r - 1)(p^r - p)(p^r - p^2)\dots(p^r - p^{r-1})}$
        \end{proof}

        
    \end{enumerate}

%%%%%%%%%%%%%%%%%%%%%%%%%%%%%%%%%%%%%%%%%%%%%%%%%%%%%%%%%%%%%%%

    \item Let $c, d \in \N$ and let $G$ be a group. Establish a one to one correspondence between the set of homomorphisms $\Z/c\Z \times \Z/d\Z \rightarrow G$ and the subset $\{(a, b) \in G \times G : a^c = b^d = e \text{ and } ab = ba\}$.

    \begin{proof}[Solution.]
        We will do this by defining a bijective function $\psi$ which maps from the set of homomorphisms $\Z/c\Z \times \Z/d\Z \rightarrow G$ to the subset $\{(a, b) \in G \times G : a^c = b^d = e \text{ and } ab = ba\}$.

        \sk

        First, we note that $\Z/c\Z \times \Z/d\Z$ is generated by $(1, 0)$ and $(0, 1)$, since any $(x, y) \in Z/c\Z \times \Z/d\Z$ can be expressed as $x(1, 0) + y(0, 1)$. A homomorphism is uniquely determined by its actions on the generators, so every homomorphism from $Z/c\Z \times \Z/d\Z$ to $G$ is determined by $\phi((1, 0)) = a$ and $\phi((0, 1)) = b$. So, any group homomorphism from $\Z/c\Z \times \Z/d\Z$ can be denoted as $\phi_{ab}$, where $\phi_{ab}((x, y)) = a^xb^y$. 

        \sk

        Now, we define a function $\psi: \{\phi_{ab}\} \rightarrow \{(a, b) \in G \times G : a^c = b^d = e \text{ and } ab = ba\}$ by $\psi(\phi_{ab}) = (a, b)$. First, we show that $\psi$ is well-defined, and then we show that it is bijective. This will show that there is a one to one correspondence between the set of homomorphisms $\Z/c\Z \times \Z/d\Z \rightarrow G$ and the subset $\{(a, b) \in G \times G : a^c = b^d = e \text{ and } ab = ba\}$.

        \sk

        To show $\psi$ is well-defined, we suppose $\phi_{ab} = \phi_{a'b'}$ and then show that $\psi(\phi_{ab}) = \psi(\phi_{a'b'})$. If $\phi_{ab} = \phi_{a'b'}$, then $\phi_{ab}((x, y)) = \phi_{a'b'}((x, y))$ for all $(x, y)$. In particular, this is true for $(1, 0)$, and so $\phi_{ab}((1,0)) = \phi_{a'b'}((1, 0))$, meaning $a = a'$. Similarly, $\phi_{ab}((0, 1)) = \phi_{a'b'}((0, 1))$ implies $b = b'$, and $(a, b) = (a', b')$. 
        
        \sk
        
        As well, since $a = a' = \phi((1, ))$, we have $a^c = \phi((1, 0))^c = \phi((1, 0)^c) = \phi((0, 0)) = e_G$. Similarly, since $b = b' = \phi((0, 1))$, $b^d = e_G$. Lastly, $ab = \phi((1, 0))\phi((0, 1)) = \phi((1, 1)) = \phi((0, 1))\phi((1, 0)) = ba$. 
        
        \sk
        
        Thus, $\psi(\phi_{ab}) = (a, b) = (a', b') = \psi(\phi_{a'b'})$, and $(a, b) \in \{(a, b) \in G \times G : a^c = b^d = e \text{ and } ab = ba\}$, so $\psi$ is well-defined. 

        \sk

        To show $\psi$ is injective, we suppose $\psi(\phi_{ab}) = \psi(\phi_{a'b'})$ and then show that $\phi_{ab} = \phi_{a'b'}$. Since $\psi(\phi_{ab}) = \psi(\phi_{a'b'})$, we have that $(a, b) = (a', b')$. Thus, $\phi_{ab}((x, y)) = a^xb^y = a'^xb'^y = \phi_{a'b'}$, and so $\psi$ is injective. 

        \sk

        To show $\psi$ is surjective, we want to show that for any $(a, b) \in \{(a, b) \in G \times G : a^c = b^d = e \text{ and } ab = ba\}$, there is a $\phi$ such that $\psi(\phi) = (a, b)$. Take $\phi_{ab}$. Then $\psi(\phi_{ab})$ is defined to be $(a, b)$, and so $\psi$ is surjective. 

        \sk

        Since we have established a bijective function from the set of homomorphisms $\Z/c\Z \times \Z/d\Z \rightarrow G$ to the subset $\{(a, b) \in G \times G : a^c = b^d = e \text{ and } ab = ba\}$, these sets have a one to one correspondence.
        
    \end{proof}

%%%%%%%%%%%%%%%%%%%%%%%%%%%%%%%%%%%%%%%%%%%%%%%%%%%%%%%%%%%%%%%

    \item Let $G$ and $H$ be groups. Let $\phi: G \rightarrow H$ be a homomorphism of groups. 
    \begin{enumerate}
        \item Show that $\ker \phi$ is a normal subgroup of $G$.

        \begin{proof}[Solution.]
            To show that $\ker \phi$ is a normal subgroup of $G$, we first show that it is a subgroup, and then show that it satisfies $a\ker\phi = \ker\phi a$ for all $a \in G$. 

            \sk

            First, since $\phi$ is a homomorphism, $\phi(e_G) = e_H$, and so $e_G \in ker \phi$, so $\ker \phi$ is non-empty. 

            \sk

            Second, suppose $a, b \in \ker \phi$. We want to show that $a^{-1} \in \ker \phi$ and $ab \in \ker \phi$. Since $a \in \ker \phi$, $\phi(a) = e_h$. Taking the inverse of both sides, we get $\phi(a)^{-1} = e_H^{-1} = e_H$. Since $\phi$ is a homomorphism, $\phi(a)^{-1} = \phi(a^{-1} = e_H$, so $\phi(a) \in \ker \phi$. Then, since $a, b \in \ker \phi$ and since $\phi$ is a homomorphism, $e_H = e_He_H = \phi(a)\phi(b) = \phi(ab)$, and so $ab \in \ker \phi$. Thus, $\ker \phi$ is a subgroup. 

            \sk

            Now, we want to show that $a \ker \phi = \ker \phi a$ for all $a \in G$. Suppose that $ak = \in a \ker \phi$ where $k \in \ker \phi$. We want to show that $ak \in \ker \phi a$, which is equivalent to showing that $aka^{-1} \in \ker \phi$. If we apply $\phi$ to $aka^{-1}$, we get $\phi(aka^{-1}) = \phi(a)\phi(k)\phi(a^{-1}) = \phi(a)\phi(a^{-1}) = e_H$. Thus, $aka^{-1} \in \ker \phi$, and so $ak \in \ker \phi a$, and $a \ker \phi \subset \ker \phi a$. The proof that $\ker \phi a \subset a \ker \phi$ is analogous, and so $a \ker \phi = \ker \phi a$ for all $a \in G$. 

            \sk

            Since $\ker \phi$ is a subgroup and satisfies $a \ker \phi = \ker \phi a$ for all $a \in G$, $\ker \phi$ is a normal subgroup of $G$.
        \end{proof}
        
        \item Show that the image $\text{im} \phi$ is a subgroup of $H$. 

        \begin{proof}[Solution.]

        We will use the two-step subgroup test to show that  $\text{im} \phi$ is a subgroup of $H$. 

        \sk 

        First, we show that $\text{im} \phi$ is non-empty. Since $\phi$ is a homomorphism, $\phi(e_G) = e_H$, and so $e_G \in \text{im}\phi$. Thus, $\text{im}\phi$ is non-empty. 

        \sk

        Second, suppose $h_1, h_2 \in \text{im}\phi$. We want to show that $h_1h_2, \in \text{im}\phi$. Since $h_1, h_2 \in \text{im}\phi$, there exists $g_1, g_2 \in G$ such that $\phi(g_1) = h_1$ and $\phi(g_2) = h_2$. Since $\phi$ is a homomorphism, $h_1h_2 = \phi(g_1)\phi(g_2) = \phi(g_1g_2)$. Since $G$ is a group and $g_1, g_2 \in G$, we know that $g_1g_2 \in G$, and so $h_1h_2 \in \text{im}\phi$. 

        \sk

        Third, suppose $h_1 \in \text{im}\phi$. We want to show that $h_1^{-1} \in \text{im}\phi$. Since $h_1 \in \text{im}\phi$, there exists a $g_1 \in G$ such that $\phi(g_1) = h_1$. Taking the inverse of both sides and using the fact that $\phi$ is a homomorphism, we get $h_1^{-1} = \phi(g_1)^{-1} = \phi(g_1^{-1})$. Since $G$ is a group and $g_1 \in G$, $g_1^{-1} \in G$, and so $h_1^{-1} \in \text{im}\phi$. 

        \sk

        By the two-step subgroup test, since $\text{im}\phi$ is non-empty, it is a subgroup of $H$.

        \end{proof}
        
        \item The \textit{first isomorphism theorem} states that the groups $G/\ker(\phi) \simeq \text{im}(\phi)$ are isomorphic. Consider a cyclic group $G$, and the homomorphisms $\phi: \Z \rightarrow G$. Use the theorem to show that $G$ is isomorphic to $\Z$ or $\Z/m\Z$ for some integer $m \geq 1$. 

        \begin{proof}[Solution.]

        We are given that $G$ is cyclic, which means it is generated by some element $g$. A homomorphism from $\Z$ to $G$ is then $\phi(n) = g^n$, and since $G = \langle g \rangle = \{g^n : n \in \Z\}$, the image of $\phi$ is $G$. 

        \sk

        The kernel of $G$ is either $\{0\}$ if $G$ is infinite, or it is $\{mn : n \in \Z\} = m\Z$ where $m$ is the order of $g$. 

        \sk

        Now, using the first isomorphism theorem, we have two cases. If $\ker \phi = \{0\}$, then $\Z = \Z/\ker \phi \cong \text{im}\phi = G$. If $\ker \phi = k\Z$, then $\Z/m\Z = \Z/\ker\phi \cong \text{im}\phi = G$. Thus, if $G$ is a cyclic group, $G$ is isomorphic to either $\Z$ or $\Z/m\Z$ for some integer $m \geq 1$.
            
        \end{proof}
    \end{enumerate}
\end{enumerate}


\end{document}