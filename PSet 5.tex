\documentclass{article}
\usepackage{graphicx} % Required for inserting images
\usepackage{amsfonts}
\usepackage{amsmath}
\usepackage{amssymb}
\usepackage{amsthm}
\usepackage{tikz}
\usepackage[textwidth=6in,textheight=10in]{geometry}

\newcommand{\T}{\mathcal{T}}
\newcommand{\U}{\mathcal{U}}
\newcommand{\R}{\mathbb{R}}
\newcommand{\N}{\mathbb{N}}
\newcommand{\Z}{\mathbb{Z}}
\newcommand{\C}{\mathbb{C}}
\newcommand{\B}{\mathcal{B}}
\newcommand{\V}{\mathcal{V}}
\newcommand{\F}{\mathcal{F}}
\newcommand{\sk}{\smallskip}

\title{PSet 5}
\author{Aly Rajwani}
\date{\today}

\begin{document}

\maketitle

\begin{enumerate}
    \item \textbf{Exercise 19.6}
    
    Let $x_1, x_2, \dots$ be a sequence of points in the product space $\Pi X_\alpha$. Show that this sequence converges to the point $x$ iff the sequence $\pi_\alpha(x_1), \pi_\alpha(x_2), \dots$ converges to $\pi_\alpha(x)$ for every $\alpha$. Is this fact true if one uses the box topology instead of the product topology?
    
    \begin{proof}[Solution.]
    Recall that a sequence $(x_n)$ converges to $x$ in a topological space if for every neighborhood $\U$ of $x$, there exists an $N$ such that for all $n \geq N$, $x_n \in \U$. 
    
    \sk
    
    Assume that $(x_n) \longrightarrow x$, and fix some neighborhood $\U_\beta$ of $\pi_\beta(x)$. Since any neighborhood $\U$ of $x$ has an $N$ such that for all $n \geq N$, $x_n \in \U$, let $\U_\alpha = \begin{cases}
        X_\alpha \text{ if } \alpha \neq \beta \\
        \U_\beta \text{ otherwise }
    \end{cases}$ and let $\U = \Pi \U_\alpha$. Then, there exists an $N$ such that for all $n \geq N$, $x_n \in \U$, and since $\pi_\beta(x_n) \in \pi_\beta(\U) = \U_\beta$, this implies that there exists an $N$ such that for all $n \geq N$, $\pi_\beta(x_n) \in \U_\beta$. Since $\beta$ was arbitrary, $(\pi_\alpha(x_n)) \longrightarrow \pi_\alpha(x)$ for all $\alpha$.

    \sk

    Now, we prove that if for all $\alpha$, $(\pi_\alpha(x_n)) \longrightarrow \pi_\alpha(x)$, then $(x_n) \longrightarrow x$. Fix some basis element $\U$ containing $x$. In the product topology, $\U$ is equal to $\Pi \U_\alpha$ where $\U_\alpha \neq X_\alpha$ for only finitely many $\alpha$. Since for all $\alpha$, $(\pi_\alpha(x_n)) \longrightarrow \pi_\alpha(x)$, we know that for each $\U_\alpha$, there exists an $N_\alpha$ such that for all $n \geq N_\alpha$, $\pi_\alpha(x_n) \in \U_\alpha$. For each $\alpha$ such that $\U_\alpha = X_\alpha$, $N_\alpha = 1$ is sufficient. Since there are only finitely many other $\alpha$, we can set $N = \max(N_\alpha)$. Then for all $n \geq N$, $x_n \in \U$, since each $x_\alpha \in \U_\alpha$. This implies that $(x_n) \longrightarrow x$ in the product topology.

    \sk

    This is not true for the box topology, since this relied upon us being able to choose $N = \max(N_\alpha)$. We will prove that this statement is not true in the box topology. 

    \sk
    
    Consider the box topology on $\R^\N$. Define $x_n$ to be $\left(\frac{1}{n}, \frac{1}{n}, \dots \right)$. Then for each $\alpha$, $(\pi_\alpha(x_n)) \longrightarrow 0$, since the sequence $\left(\frac{1}{n}\right) \longrightarrow 0$. Consider the open set $\U = \Pi \left(-\frac{1}{n}, \frac{1}{n}\right)$. Clearly $(0, 0, \dots) \in \U$, since $-\frac{1}{n} < 0 < \frac{1}{n}$. However, no $x_n \in \U$, since the $n+1^\text{st}$ term in $x_n$ is not an element of $\left(-\frac{1}{n+1}, \frac{1}{n+1}\right)$. Thus, for every $N$, there exists an $n \geq N$ such that $x_n \notin \U$, and so $(x_n) \not\longrightarrow x$ in the box topology.
        
    \end{proof}

    \item \textbf{Exercise 19.7}

    Let $\R^\infty$ be the subset of $\R^\omega$ consisting of all sequences that are "eventually zero," that is, all sequences $(x_1, x_2, \dots )$ such that $x_i \neq 0$ for only finitely many values of $i$. What is the closure of $\R^\infty$ in the product and box topologies? 

    \begin{proof}[Solution.]
        Recall that $x \in \overline{A}$ if for all open $\U$ containing $x$, $\U \cap A \neq \emptyset$. 

        \sk
        
        In the product topology, the closure of $\R^\infty$ is all of $\R^\omega$. 

        \sk

        Let $x \in \R^\omega$ and let $\U$ be a basis element in the product topology containing $x$. This implies that $\U = \Pi \U_\alpha$ where $\U_\alpha = \R$ for all but finitely many $\alpha$. Let $I$ be an index set containing all $\alpha$ such that $\U_\alpha \neq \R$. 

        \sk

        Let $y$ be a sequence defined as follows: 
        $$y_\alpha = \begin{cases}
            x_\alpha \text{ if } \alpha \in I \\
            0 \text{ otherwise }
        \end{cases}$$ 

        Then $y_\alpha \in \U_\alpha$ for all $\alpha$. This is true because if $\alpha \in I$, then $x_\alpha \in \U_\alpha$, and $y_\alpha = x_\alpha$. If $\alpha \notin I$, then $y_\alpha = 0 \in \R$. Thus, $y \in \U$. As well, since there are only finitely many $\alpha \in I$, $y$ is eventually zero. Thus, $\U \cap \R^\infty \neq \emptyset$. Since $x$ and $\U$ were arbitrary, $x$ is in the closure of $\R^\infty$ by definition. 

        \sk

        This proof required us to have only finitely many $\alpha \in I$. This leads us to the conclusion that the closure of $\R^\infty$ in the box topology is $\R^\infty$ itself. 

        \sk

        We will prove that $\R^\infty = \overline{\R^\infty}$ by showing that if $x \notin \R^\infty$, then $x$ is not a limit point of $\R^\infty$. This implies that $\R^\infty$ already contains its limit points, and so it is closed. 

        \sk

        Suppose $x \notin \R^\infty$, meaning $x_\alpha \neq 0$ for infinitely many values of $\alpha$. We will define an open set $\U = \Pi \U_\alpha$ where $\U_\alpha$ is defined as follows:

        $$\U_\alpha = \begin{cases}
            (-1, 1) \text{ if } x_\alpha = 0 \\
            \left(0, 2x_\alpha \right) \text{ if } x_\alpha > 0 \\
            \left(2x_\alpha, 0\right) \text{ if } x_\alpha < 0
        \end{cases}$$

        Clearly, $\U$ is open in the box topology and $x \in \U$. Since $x_\alpha \neq 0$ for infinitely many $\alpha$, any $y \in \U$ will have infinitely many $y_\alpha$ in either $\left(0, 2x_\alpha \right)$ or $\left(2x_\alpha, 0\right)$. This means that any $y$ will have infinitely many terms not equal to $0$, and so $y \notin \R^\infty$. Thus, we have found an open neighborhood $\U$ of $x$ in the box topology such that $\U \cap \R^\infty = \emptyset$, so $x$ cannot be a limit point of $\R^\infty$. Since $x$ was an arbitrary element not in $\R^\infty$, $\R^\infty$ must contain all its limit points, and so $\overline{\R^\infty} = \R^\infty$ in the box topology. 
    \end{proof}

    \item \textbf{Exercise 21.6}
    
    Define $f_n: [0, 1] \rightarrow \R$ by the equation $f_n(x) = x^n$. Show that the sequence $(f_n(x))$ converges for each $x \in [0, 1]$, but that the sequence $f_n$ does not converge uniformly. 

    \begin{proof}[Solution.]
    We claim that $(f_n(x))$ converges pointwise to the function $$f(x) = \begin{cases}
        1 \text{ if } x = 1 \\
        0 \text{ otherwise }
    \end{cases}$$.

    We prove this in three cases. First, if $x = 1$, then $f_n(x) = 1$ for all $n$, and so $(f_n(x)) \longrightarrow 1 = f(x)$. Second, if $x = 0$, then $f_n(x) = 0$ for all $n$, and so $(f_n(x)) \longrightarrow 0 = f(x)$. Third, if $0 < x < 1$, then we claim $(f_n(x)) \longrightarrow 0 = f(x)$. This sequence is  decreasing and bounded below since $0 < f_2(x) = x^2 < x = f_1(x)$, and inductively: 
    \begin{align*}
        & f_{n+1}(x) < f_n(x) \\
        & x^{n+1} < x^n \\
        & xx^{n+1} < xx^n \\
        & x^{n+2} < x^{n+1} \\
        & f_{n+2}(x) < f_{n+1}(x) 
    \end{align*}
    
    Thus, $(f_n(x))$ converges, so let it converge to $a$. As well, $(f_{n+1}(x))$ must converge to $a$. So, $\lim x^n = a = \lim x^{n+1} = x \lim x^n = ax$. So, $a = ax$, and since $x \neq 1$, $a = 0$. Thus, we have shown that $(f_n(x))$ converges pointwise to $f(x)$.

    \sk

    Now, we show that $(f_n)$ does not converge uniformly. Since $\R$ is a Hausdorff space, limits are unique. Since uniform convergence implies pointwise convergence, if $(f_n)$ does not converge pointwise to $g$, then it does not converge uniformly to $g$. Since $(f_n)$ converges pointwise to $f$, if $g \neq f$, then $(f_n)$ does not converge pointwise to $g$, and so it cannot converge uniformly to $g$. Thus, the only candidate for uniform convergence is $f$. 

    \sk
    
    We know that if $(f_n)$ is a sequence of continuous functions that converges uniformly to $f$, then $f$ is continuous. Taking the contrapositive, if $f$ is not continuous, then $(f_n)$ does not converge to $f$. By elementary calculus, $f_n(x) = x^n$ is continuous when $x \in [0, 1]$ and $\lim_{x \rightarrow 1^-} f(x) = 0 \neq 1 = f(1)$. Thus, $f$ is not continuous, so $(f_n)$ does not converge uniformly. 

    
        
    \end{proof}

    \item \textbf{S1}
    
    Show that $\T_p \subsetneq \T_\infty \subsetneq \T_2 \subset\neq \T_1 \subsetneq \T_b$. 

    \sk

    \begin{proof}[Solution.]

    Recall that for topologies $\T$ and $\T'$ with metrics $d$ and $d'$ respectively, $\T \subset \T'$ iff for all $x \in X$ and for all $\epsilon > 0$, there exists a $\delta$ such that $B_{d'}(x, \epsilon) \subset B_{d}(x, \delta)$. We will use this to prove one topology is finer than another.

    \sk

    We will answer this (rather painstakingly) inequality by inequality. We start with $\T_p \subsetneq \T_\infty$.

    \sk 

    A metric which induces $\T_p$ on $X$ is defined by $d_P = \text{sup}\left\{\frac{\overline{d}(x, y)}{i}\right\}$, where $\overline{d}(x, y) = \min(|x - y|, 1)$. A metric which induces $\T_\infty$ on $X$ is defined by $d_\infty(x, y) = \text{sup}\{|x_i - y_i|\}$. 

    \sk 

    Fix some $x$ and $\epsilon > 0$, and let $\delta = \epsilon$. Then for any $y \in B_{d_\infty}(x, \delta)$, we have that $\text{sup}\left\{|x_i - y_i|\right\} < \delta$. Since $\text{sup}\left\{\frac{\overline{d}(x_i, y_i)}{i}\right\} \leq \text{sup}\{|x_i - y_i|\} < \delta = \epsilon$, if $y \in B_{d_\infty}(x, \delta)$, then $y$ satisfies $\text{sup}\left\{\frac{\overline{d}(x_i, y_i)}{i}\right\} < \epsilon$, and so $y \in B_{d_P}(x, \epsilon)$. Thus, for any $x \in X$ and $\epsilon > 0$, there is a $\delta$ such that $B_{d_\infty}(x, \delta) \subset B_{d_P}(x, \epsilon)$, and so $\T_P \subset \T_\infty$.

    \sk

    Now we will show this inequality is strict by constructing a neighborhood open in $\T_\infty$ that is not open in $\T_p$. Let $x = 0 = (0, 0, \dots)$ and let $\epsilon = 1$. Then $B_{d_\infty}(0, 1) = \{y \in X : \text{sup}\{|y_i|\} < 1\}$. If this set were open in $\T_p$, then we could find a basis element $B_{d_P}(0, \delta)$ of $\T_p$ such that $0 \in B_{d_P}(0, \delta) \subset B_{d_\infty}(0, 1)$. Such a $B_{d_P}(0, \delta)$ is of the form $\Pi \U_i$, where $\U_i = \C$ for all but finitely many $i$. Let $I$ be the set of all $i$ such that $\U_i \neq \C$. Then, $B_{d_P}(0, \delta)$ is of the form $\{y \in X: \forall i \in I, |y_i| < \delta\}$. Define a point $w$ as follows:

    $$w_i = \begin{cases}
        0 \text{ if } i \in I \\
        2 \text{ otherwise }
    \end{cases}$$

    Clearly $w \in B_{d_P}(0, \delta)$, since the only indices that are restricted are in $I$, and $w_i = 0$ for each of these indices. However, for the other indices, $w_i = 2$, and so $\text{sup}\{|w_i|\} = 2$, meaning that that $w \notin B_{d_\infty}(0, 1)$. Thus, $B_{d_P}(0, \delta) \not\subset B_{d_\infty}(0, 1)$ for all $\delta$, and so $B_{d_\infty}(0, 1)$ is not open in $\T_P$. Therefore $\T_P \subsetneq \T_\infty$.

    \sk

    Now we show $\T_\infty \subsetneq \T_2$.

    \sk

    A metric which induces $\T_2$ on $X$ is $d_2(x, y) = \left(\sum |x_i - y_i|^2 \right)^\frac{1}{2}$. We know that $||x||_{\ell^\infty} \leq ||x||_{\ell^2}$ for all $x$. 

    \sk

    Fix some $x \in X$ and $\epsilon > 0$, and let $\delta = \epsilon$. Then for any $y \in B_{d_2}(x, \delta)$, we have that $||x - y||_{\ell^2} < \delta$. But, since $||x-y||_{\ell^\infty} \leq ||x - y||_{\ell^2} < \delta = \epsilon$, $y$ satisfies $||x-y||_{\ell^\infty} < \epsilon$. This implies $y \in B_{d_\infty}(x, \epsilon)$. Thus, for any $x \in X$ and $\epsilon > 0$, there exists a $\delta$ such that $B_{d_2}(x, \delta) \subset B_{d_\infty}(x, \epsilon)$. This implies that $\T_\infty \subset \T_2$. 

    \sk

    Now we will show this inequality is strict by constructing a neighborhood open in $\T_2$ that is not open in $\T_\infty$. Let $x = 0 = (0, 0, \dots)$ and let $\epsilon = 2$. If $B_{d_2}(0, 2)$ were open in $\T_\infty$, then for all $y \in B_{d_2}(0, 2)$, we could find a $\delta$ such that $B_{d_\infty}(0, \delta) \subset B_{d_2}(0, 2)$. Define a point $w$ as $\left(\frac{\delta/2}{1}, \frac{\delta/2}{\sqrt{2}}, \frac{\delta/2}{\sqrt{3}}, \dots \right)$. Then $w \in B_{d_\infty}$ since $\text{sup}\{|w_i|\} = \frac{\delta}{2} < \delta$. But, $w \notin B_{d_2}(0, 2)$ since $\left(\sum \left|\frac{\delta/2}{\sqrt{n}}\right|^2\right)^\frac{1}{2} = \frac{\delta}{2}\left(\sum \frac{1}{n}\right)^\frac{1}{2}$. This sum diverges, so it cannot be in $B_{d_2}(0, 2)$. Thus, $B_{d_\infty}(0, \delta) \not\subset B_{d_2}(0, 2)$ for all $\delta$, and so $B_{d_2}(0, 2)$ is not open in $\T_\infty$. Therefore, $\T_\infty \subsetneq \T_2$. 

    \sk

    Now we show $\T_2 \subsetneq \T_1$. 

    \sk
    A metric which induces $\T_1$ on $X$ is $d_1(x, y) = \sum |x_i - y_i|$. We know that $||x||_{\ell^2} \leq ||x||_{\ell}$ for all $x$. 

    \sk

    Fix some $x \in X$ and $\epsilon > 0$, and let $\delta = \epsilon$. Then for any $y \in B_{d_1}(x, \delta)$, we have that $||x - y||_{\ell} < \delta$. But, since $||x - y||_{\ell^2} \leq ||x - y||_{\ell} < \delta = \epsilon$, $y$ satisfies $||x - y||_{\ell^2} < \epsilon$. This implies $y \in B_{d_2}(x, \epsilon)$. Thus, for any $x \in X$ and $\epsilon > 0$, there exists a $\delta$ such that $B_{d_1}(x, \delta) \subset B_{d_2}(x, \epsilon)$. This implies that $\T_2 \subset \T_1$. 

    \sk

    Now we will show this inequality is strict by constructing a neighborhood open in $\T_1$ that is not open in $\T_2$. Let $x = 0 = (0, 0, \dots)$ and let $\epsilon = 1$. If $B_{d_1}(0, 1)$ were open in $\T_2$, then for all $y \in B_{d_1}(0, 1)$, we could find a $\delta$ such that $B_{d_2}(0, \delta) \subset B_{d_1}(0, 1)$. Define a point $w$ as $\left(\frac{\sqrt{3}\delta}{\pi}\cdot{\frac{1}{1}}, \frac{\sqrt{3}\delta}{\pi}\cdot{\frac{1}{2}}, \frac{\sqrt{3}\delta}{\pi}\cdot{\frac{1}{3}}, \dots \right)$. Then $w \in B_{d_2}(0, \delta)$ be the following calculations:

    \begin{align*}
        \left(\sum |w_i|^2\right)^\frac{1}{2} &= \left(\sum \left|\frac{\sqrt{3}\delta}{\pi}\cdot{\frac{1}{n}}\right|^2\right)^\frac{1}{2} \\
        &= \frac{\sqrt{3}\delta}{\pi}\left(\sum \frac{1}{n^2}\right)^\frac{1}{2} \\
        &= \frac{\sqrt{3}\delta}{\pi} \cdot \frac{\pi}{\sqrt{6}} \\
        &< \delta
    \end{align*}

    However, $w \notin B_{d_1}(0, 1)$ since $\sum |w_i| = \sum \left|\frac{\sqrt{3}\delta}{\pi} \cdot \frac{1}{n}\right| = \frac{\sqrt{3}\delta}{\pi} \sum \frac{1}{n}$, which diverges. Thus, $B_{d_2}(0, \delta) \not\subset B_{d_1}(0, 1)$ for all $\delta$, and so $B_{d_1}(0, 1)$ is not open in $\T_2$. Therefore, $\T_2 \subsetneq \T_1$. 

    \sk
    
    Now we show $\T_1 \subsetneq \T_B$.

    \sk

    Fix some $x \in X$ and $\epsilon > 0$, and let $B_{d_1}(x, \epsilon)$ be a basis element of $\T_1$. We want to construct an open set in the box topology that is contained in $B_{d_1}(x, \epsilon)$. Let $\U = X \cap \Pi B_\C(x_i, \frac{\epsilon}{2^i})$. Let $y \in \U$. Then we have the following: 
    
    \begin{align*}
        d_{\ell}(x, y) &= \sum |x_i - y_i| \\
        &< \sum \frac{\epsilon}{2^i} \\
        &= \epsilon
    \end{align*}

    Thus, if $y \in \U$, $d_{\ell}(x, y) < \epsilon$, and so $y \in B_{d_1}(x, \epsilon)$. Thus, for all $x \in X$ and $\epsilon > 0$, there is a neighborhood $\U$ in $\T_B$ such that $\U \in B_{d_1}(x, \epsilon)$. 

    \sk

    Now we will show this inequality is strict by constructing a neighborhood open in $\T_B$ that is not open in $\T_1$. Let $x = 0 = (0, 0, \dots)$ and let $\U = X \cap \Pi B_\C(x, \frac{1}{n})$. We will show that no $B_{d_1}(0, \epsilon) \subset \U$. Let $\epsilon$ be arbitrary, and pick $N$ such that $\frac{1}{N} < \epsilon$. Let $w = (0, 0, \dots, \frac{1}{N}, 0, 0, \dots)$, where the $\frac{1}{N}$ occurs at the $N^\text{th}$ index in $w$. Then clearly $w \in B_{d_1}(0, \epsilon)$, since $\sum |w_i| = \frac{1}{N} < \epsilon$. However, $w \notin \U$, since that would imply $\frac{1}{N} \in B_\C(0, \frac{1}{N})$, but this is false. Therefore, there is no $\epsilon$ such that $B_{d_1}(0, \epsilon) \subset \U$, and so $\U$ is not open in $\T_1$. Therefore, $\T_1 \subsetneq \T_B$.  

    \end{proof}

    \item \textbf{S2}
        \begin{enumerate}
            \item Let $x_{nm} = \begin{cases}
                1 \text{ if } m = n \\
                0 \text{ otherwise }
            \end{cases}$. Show that $(x_n) \longrightarrow 0$ in $X_P$, but that it does not converge in $X_\infty$.

            \begin{proof}[Solution.]
                
            
            First, we show that $(x_n) \longrightarrow 0$ in $X_p$. In the product topology on $\C^\N$, $(x_n) \longrightarrow x$ if and only if $(\pi_m(x_n)) \longrightarrow \pi_m(x)$ for all $m$. Fix some $\epsilon > 0$ and some $m \in \N$ and let $N = m + 1$. Then $\forall n \geq N$, $x_{nm} = 0$, so $|x_{nm} - 0| < \epsilon$. Thus, $(\pi_m(x_n)) \longrightarrow 0$, and so $(x_n) \longrightarrow 0$.

            \sk

            Now we show that $(x_n)$ does not converge in $X_\infty$. Convergence in $X_\infty$ means that $\lim \text{sup}|x_{nm} - x_{m}| = 0$. Suppose $(x_n) \longrightarrow x$ for some $x = (x_1, x_2, x_3, \dots)$. For any $m$, we can pick an $n' > m$ and find an $x_{n'm'} = 1$. For such an $m'$, $x_m'$ must equal 1 in order to have that $\lim \text{sup} |x_{nm} - x_m| = 0$. As well, since $x_{n'm'+1} = 0$, $x_{m'+1}$ must equal 0. However, $x_{n'+1m'+1} = 1$, and so in order to maintain $\lim \text{sup}|x_{nm} - x_m| = 0$, we must have that $x_{m'+1} = 1$. This contradicts our earlier claim that $x_{m'+1} = 0$. Since this logic applies to any $x$, $(x_n)$ cannot converge in $X_\infty$.  

            \end{proof}

            \item Let $x_{nm} = \begin{cases}
                \frac{1}{\sqrt{m}} \text{ if } 2^{n-1} \leq m < 2^n \\
                0 \text{ otherwise }
            \end{cases}$. Show that $(x_n) \longrightarrow 0$ in $X_\infty$, but that it does not converge in $X_2$.

            \begin{proof}[Solution.]

            
            First, we show that $(x_n) \longrightarrow 0$ in $X_\infty$. We want to show that $\forall \epsilon > 0$, $\exists N$ such that $\forall n \geq N$, $\text{sup}|x_{nm} - x_{m}| = \text{sup}|x_{nm}|< \epsilon$. For any $x_n$, each $x_{nm} = 0$ except for $m$ satisfying $2^{n-1} \leq m < 2^n$. This inequality implies that $\frac{1}{2^{(n-1)/2}} \geq \frac{1}{\sqrt{m}}$. So, for any particular $n$, $\text{sup}\{x_{nm}\} = \frac{1}{2^{(n-1)/2}}$. Now, for any $\epsilon > 0$, let $N > \frac{\log 2 - 2 \log \epsilon}{\log 2}$. Then, $\forall n \geq N$, $\epsilon > \frac{1}{2^{(n-1)/2}} = \text{sup}|x_{nm}|$. Thus, $(x_n) \longrightarrow 0$ in $X_\infty$.

            \sk

            Now we show that $(x_n)$ does not converge in $X_2$. Convergence to $x$ in $X_2$ means that $\lim \left(\sum |x_{nm} - x_m|^2\right)^\frac{1}{2} = 0$. Fix some $N$. Then $\forall n \geq N$, $\sum_{m=2^{n-1}}^{2^n} |x_{nm} - x_m| = \sum_{m=2^{n-1}}^{2^n} \left|\frac{1}{\sqrt{m}} - x_m\right|> 0$. For a particular range $2^{n-1} \leq m < 2^n$, if we want $\left(\sum|x_{nm} - x_{m}|^2\right)^\frac{1}{2} = 0$, then we must have $x_m = \frac{1}{\sqrt{m}}$. But since this applies to every such range, $x_{nm} = \frac{1}{\sqrt{m}}$ for every $m$. If this was not that case, then for every $N$, we could find an $n \geq N$ such that $x_{nm} - x_m > \epsilon$, and thus the sequence would not converge. However, if we let $x = \left(\frac{1}{1}, \frac{1}{\sqrt{2}}, \frac{1}{\sqrt{3}}, \dots\right)$, then $x \notin X_2$. This is because $\left(\sum \left|\frac{1}{\sqrt{m}}\right|^2\right)^\frac{1}{2} = \left(\sum \frac{1}{m}\right)^\frac{1}{2} = \infty$. Thus, $(x_n)$ cannot converge in $X_2$. 

            \end{proof}

            \item Let $x_{nm} = \begin{cases}
                \frac{1}{m} \text{ if } 2^{n-1} \leq m < 2^n \\
                0 \text{ otherwise }
            \end{cases}$. Show that $(x_n) \longrightarrow 0$ in $X_2$, but that it does not converge in $X_1$.

            \begin{proof}[Solution.]
                
            
            First, we show that $(x_n) \longrightarrow 0$ in $X_2$. We want to show that $\forall \epsilon > 0$, $\exists N$ such that $\forall n \geq N$, $\left(\sum|x_{nm}|^2\right)^\frac{1}{2} < \epsilon$. We note that $\left(\sum_{m=2^{n-1}}^{2^n-1} \frac{1}{m^2}\right)^\frac{1}{2} < \left(2^{n-1} \cdot \frac{1}{2^{2(n-1)}}\right)^\frac{1}{2} = 2^{(1-n)/2}$. Then, choose $N$ such that $N > 1 - \frac{2\log \epsilon}{\log 2}$. Then, for all $n \geq N$, $2^{(1-n)/2} < \epsilon$, and so $\left(\sum |x_{nm}|\right)^\frac{1}{2} = \left(\sum_{m=2^{n-1}}^{2^n-1} \frac{1}{m^2}\right)^\frac{1}{2}< 2^{(1-n)/2} < \epsilon$. Thus, $(x_n) \longrightarrow 0$ in $X_2$.

            \sk

            Now we show that $(x_n)$ does not converge in $X_1$. Convergence to $x$ in $X_1$ means that $\lim \sum |x_{nm} - x_{m}| = 0$. Fix some $N$. Then $\forall n \geq N$, $\sum_{m=2^{n-1}}^{2^n-1} \approx \log 2$. For a particular range $2^{n-1} \leq m < 2^n$, if we want $\sum|x_{nm} - x_{m}| = 0$, then we must have $x_m = \frac{1}{m}$. But since this applies to every such range, $x_{nm} = \frac{1}{m}$ for every $m$. If this was not that case, then for every $N$, we could find an $n \geq N$ such that $x_{nm} - x_m > 0$, and thus the sequence would not converge. However, if we let $x = \left(1, \frac{1}{2}, \frac{1}{3}, \dots \right)$, then $x \notin X_1$. This is because $\sum \frac{1}{m} = \infty$. Thus, $(x_n)$ cannot converge in $X_1$. 

            \end{proof}
            
            \item Let $x_{nm} = \begin{cases}
                \frac{1}{m^2} \text{ if } m \geq n \\
                0 \text{ otherwise }
            \end{cases}$. Show that $(x_n) \longrightarrow 0$ in $X_1$, but that it does not converge in $X_B$.

            \begin{proof}[Solution.]
            
            First, we show that $(x_n) \longrightarrow 0$ in $X_1$. We want to show that $\forall \epsilon > 0$, $\exists N$ such that $\forall n \geq N$, $\sum |x_{nm}| < \epsilon$. We note that the series $\sum_{k=1}^\infty \frac{1}{k^2}$ converges, which means that for any $\epsilon > 0$, we can find an $N$ such that $\forall n \geq N$, $\sum_{k=n}^\infty \frac{1}{k^2} < \epsilon$. We will take this $N$ as our choice. Then, for all $n\geq N$, $\sum |x_{nm}| = \sum_{k=n}^\infty \frac{1}{k^2} < \epsilon$. Thus, $(x_n) \longrightarrow 0$ in $X_1$.

            \sk

            Now we show that $(x_n)$ does not converge in $X_B$. If $(x_n)$ converges to $x$ in the box topology, then there exist $M, N \geq 1$ such that $\forall n \geq N$ and $m \geq M$, $x_{nm} - x_{m} = 0$. However, given any $N$ and $M$, can can take $n \geq N$ and $m \geq n$, and so $x_{nm} = \frac{1}{m^2}$. But, we can take an $n' \neq n$ with $n' > m$ such that $x_{n'm} = 0$. Then $x_{n'm} - x_m \neq 0$, and so $(x_n)$ does not converge in the box topology. 
            
            \end{proof}
        \end{enumerate}

    \item \textbf{S3}

    Prove that $(x_n) \longrightarrow 0 $ in $X_B$ iff there exists $N \geq 1$ and $M \geq 1$ such that $x_{nm} = 0$ for all $n \geq N, m \geq M$ and for each $m$, $1 \leq m \leq M - 1$, $\lim x_{nm} = 0$.

    \begin{proof}[Solution.]

    Suppose that $(x_n) \longrightarrow 0$ in $X_B$. Let $s_{nm} = \sup_{r\geq n}|x_{rm}| \geq 0$. Suppose $\forall n$ there are infinitely many $m$ for which $s_{nm} > 0$. This will lead to a contradiction. Let $m: \N \rightarrow \N$ be defined as follows: let $m(1)$ be the first index $k \geq 1$ such that $s_{1k} > 0$, let $m(2)$ be the first index $k > m(1)$ such that $s_{2k} > 0$, etc. Now let $\U = B_1 \times B_2 \times \dots $ where $B_m = \{x \in \C : |x| < \frac{s_{nm(n)}}{2}\}$ if $m(n) = m$ for some $n$, and $B_m = \C$ if there is no such $n$. Then $\U \cap X_B$ is a neighborhood of 0, but $\forall N, \exists n \geq N$ such that $x_n \notin \U \cap X_B$.

    \sk

    First, $\U \cap X_B$ is a neighborhood of $0$ since $0 \in \C$, and $|0| < \frac{s_{nm(n)}}{2}$, so $0_m \in \B_m$ for every $m$.

    \sk

    Now we show the contradiction. Let $N$ be arbitrary, and suppose that $\forall n \geq N$, $x_n \in \U \cap X_B$. This implies that $x_{nm} \in B_m$ for all $m$. Let $n' \geq n$ such that $m(n') = m'$. By definition, this means that $s_{n'm'} > 0$. Since $s_{n'm'}$ is a supremum, $\forall \epsilon > 0$, there exists an $r \geq n$ such that $s_{n'm'} - \epsilon < |x_{rm'}|$. Let $\epsilon = \frac{s_{n'm'}}{2}$. Then there exists an $r$ such that $|x_{rm'}| > \frac{s_{n'm'}}{2}$. But, this implies that $x_{rm'} \notin B_{m'}$, and so $x_r \notin \U$. Thus, we have found an $r \geq N$ such that $x_r \notin \U \cap X_B$. Since $N$ was arbitrary, this applies to all $N$, and so $(x_n)$ does not converge to $0$. This contradicts our initial assumption. 

    \sk

    The cause of this contradiction was the ability to choose an $n'$ for which $s_{n'm'} > 0$, which is only possible if there are infinitely many $m'$. Thus, it must be that there are only finitely many $m$ for which $s_{nm} > 0$. This means that there must be an $N$ and $M$ beyond which $s_{nm} = 0$. This statement directly implies that there exists an $N$ and an $M$ such that $\forall n \geq N$ and $m \geq M$, $x_{nm} = 0$ and for $1 \leq m \leq M - 1$, $\lim x_{nm} = 0$. 

    \sk 

    Now suppose that there exists an $N$ and an $M$ such that $\forall n \geq N$ and $m \geq M$, $x_{nm} = 0$ and for $1 \leq m \leq M - 1$, $\lim x_{nm} = 0$. We want to show that $(x_n) \longrightarrow 0$ in $X_B$, that is, if $\U$ is a neighborhood of $0$, then there exists an $N$ such that for all $n \geq N$, $x_n \in \U$. If $\U$ is open in $X_B$, then there is a basis element $B \subset \U$ of the form $B = B_\C(x_1, \epsilon_1) \times B_\C(x_2, \epsilon_2) \times \dots$ where $0 \in B$. 

    \sk

    For any such $B$, let $N$ be the $N$ such that $\forall n \geq N$ and $m \geq M$, $x_{nm} = 0$. There are now two cases. If $m \geq M$, then $x_{nm} = 0$, and so $x_{nm} \in B_m$. Otherwise, if $1 \leq m \leq M - 1$, then we are given $\lim x_{nm} = 0$. This means that we can find an $N_m$ such that $|x_{nm}| < \epsilon$ for $n \geq N_m$. Thus, for $n \geq N_m$, $x_{nm} \in B_m$. Let $\epsilon > 0$ and let $N'' = \max(N, N_1, N_2, \dots)$, which is valid because there are finitely many integers $m \in [1, M-1]$. For all $n \geq N''$, we have that $|x_{nm}| < \epsilon$, and so $(x_n) \longrightarrow 0$. 
        
    \end{proof}

    \item \textbf{S4}
    
    On $\R^2$, consider the $\ell^1$ and $\ell^\infty$ norms. Show that neither of these norms arises from an inner product. 

    \begin{proof}[Solution.]
        We want to show that $||x||_{\ell^1} \neq \sqrt{\langle x, x\rangle}$. We prove this by contradiction. Suppose $||x||_{\ell^1} = \sqrt{\langle x, x\rangle}$. Then we would have $||x+y||^2_{\ell^1} + ||x-y||^2_{\ell^1} = 2||x||_{\ell^1}^2 + 2||y||_{\ell^1}^2$. This equality should be true for $x = (1, 0), y = (0, 1)$. However, substituting values, we see that $||x+y||^2_{\ell^1} + ||x-y||^2_{\ell^1} = 2^2 + 2^2 = 8$, and $2||x||_{\ell^1}^2 + 2||y||_{\ell^1}^2 = 2 + 2 = 4$. Since $8$ is unfortunately not equal to $4$, this equality does not hold, and so $||x||_{\ell^1}$ does not arise from an inner product. 

        \sk

        Similarly, we will show $||x||_{\ell^\infty}$ does not arise from an inner product by contradiction. If it did, then $||x+y||^2_{\ell^\infty} + ||x-y||^2_{\ell^\infty} = 2||x||_{\ell^\infty}^2 + 2||y||_{\ell^\infty}^2$. We can again substitute $x = (1, 0), y = (0, 1)$. Then $||x+y||^2_{\ell^\infty} + ||x-y||^2_{\ell^\infty} = 1 + 1 = 2$, and $2||x||_{\ell^\infty}^2 + 2||y||_{\ell^\infty}^2 = 2 + 2 = 4$. Since $4 \neq 2$, this equality does not hold, and so $||x||_{\ell^\infty}$ does not arise from an inner product.
    \end{proof}

    \item \textbf{S5}

    If $X$ is a topological space, let $C^o(X) = \{f: X \rightarrow \C : f \text{ is continuous and bounded}\}$. $||f||_{C^o(X)} = \sup |f(X)|$. This is a normed vector space. Let $C^o_c(X)$ be the subspace consisting of consisting of continuous functions with compact support. Take $X = \R^n$. Show that $\overline{C^o_c(X)}$ is $\{f: X \rightarrow \C : f \text{ is continuous and} \lim f(x) = 0\}$. Here $\lim f(x) = 0$ means $\forall \epsilon > 0, \exists R > 0$ such that $\forall x \in R^n, |x| > R \implies |f(x)| < \epsilon$.

    \begin{proof}[Solution.]
        First, we show that if $f \in \overline{C^o_c(X)}$, then $f \in \{f: X \rightarrow \C : f \text{ is continuous and} \lim f(x) = 0\}$. 

        \sk

        If $f \in \overline{C^o_c(X)}$, then there is a sequence of functions $(f_n)$ that converges uniformly to $f$, that is, $\forall \epsilon > 0, \exists N$ such that $\forall n \geq N$, $\sup |f(x) - f_n(x)| < \epsilon$ for all $x \in \R^n$. We want to show that this $f$ is continuous and that $\lim f(x) = 0$. 

        \sk

        By the uniform limit theorem, since each $f_n$ is in $C^o(X)$ and therefore is continuous, and since $(f_n)$ uniformly converges to $f$, $f$ is continuous. 

        \sk
        
        Since $f_n \in C^o(X)$, $f_n$ has compact support. This means that there exists a $K$ such that $\forall x \notin K$, $f_n(x) = 0$, and $K$ is compact. Since $K$ is compact, it is bounded by $R_K$, and so if $|x| > R_K$, then $f_n(x) = 0$. Let $\epsilon > 0$ and choose $N = R_K$. Then $\forall |x| > R$, $|f(x)| = |f(x) - 0| = |f(x) - f_n(x)|$. Since $(f_n)$ converges uniformly to $f$, $|f(x) - f_n(x)| < \epsilon$, and so $|f(x)| < \epsilon$. Thus, $\lim f(x) = 0$. Since $f$ is both continuous and satisfies $\lim f(x) = 0$, $f \in \{f: X \rightarrow \C : f \text{ is continuous and} \lim f(x) = 0\}$.

        \sk

        Second, we show that if $f \in \{f: X \rightarrow \C : f \text{ is continuous and} \lim f(x) = 0\}$, then $f \in \overline{C^o_c(X)}$. 

        \sk

        $f$ is continuous and $\lim f(x) = 0$, that is, $\forall \epsilon > 0, \exists R > 0$ such that $|x| > R \implies |f(x)| < \epsilon$. We will construct a sequence of functions in $C^o(X)$ that uniformly converge to $f$, which will show that $f \in \overline{C^o_c(X)}$. Let


        $$f_R = \begin{cases}
            f \text{ if } |x| \leq R \\
            0 \text{ otherwise} 
        \end{cases}$$.

        Clearly, $\lim f_R(x) = 0$, so we only need to show that $f_R$ is continuous. Since $f$ is continuous, $f_R$ is continuous for $|x| \leq R$, and since $0$ is continuous, $f_R$ is continuous for $|x| > R$. The only point left to show is that there is no discontinuity occurring at $x = |R|$. Let $|x_k| \rightarrow R$. We want to show that $|f_R(x_k) - f_R(R)| < \epsilon$. If $|x_k| \leq R$, then $|f_R(x_k) - f_R(R)| = |f(x_k) - f(R)| < \epsilon$, which is true by the continuity of $f$. If $|x_k| > R$, then $f_R(x_k) = 0$, and so $|f_R(x_k) - f_R(R)| = |f(R)| < \epsilon$, which is true since $\lim f(x) = 0$. Thus, $f_R$ is continuous for all $R$. This implies that $f_R \in C^o(X)$. 

        \sk

        Now we show that $\{f_R\}$ converges uniformly to $f$. For $|x| \leq R$, $\sup |f(x) - f_R(x)| = \sup |0| = 0$. For $|x| > R$, $\sup |f(x) - f_R(x)| = \sup |f(x)|$. But, since $\lim f(x) = 0$, we can find an $R$ such that if $|x| > R$, then $|f(x)| < \frac{\epsilon}{2}$. Thus, $\sup |f(x)| < \epsilon$. So, we have shown that $\forall \epsilon > 0$, $\exists N$ such that $\forall R \geq N, \sup |f(x) - f_R(x)| < \epsilon$ and this holds for all  $x \in \R^n$. Thus, $(f_R)$ converges uniformly to $f$. Since $(f_R) \rightarrow f$, $f \in \overline{C^o_c(X)}$. 

        \sk

        Since we showed both that $\{f: X \rightarrow \C : f \text{ is continuous and} \lim f(x) = 0\} \subset \overline{C^o_c(X)}$ and that $\overline{C^o_c(X)} \subset \{f: X \rightarrow \C : f \text{ is continuous and} \lim f(x) = 0\},$ these two sets must be equal, which is what we set out to prove.
    \end{proof}
\end{enumerate}

\end{document}

